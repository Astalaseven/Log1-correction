\chapter*{Correction des exercices 6.6}

\vspace{-2cm}

\begin{Emphase}{Structure -- Moment}
\cadre{
\begin{pseudo}
\Struct{Moment}
    \Decl heure~:~entier
    \Decl minute~:~entier
    \Decl seconde~:~entier
\EndStruct
\end{pseudo}
}
\end{Emphase}

\begin{Emphase}{Exercice 1 -- Conversion moment-secondes}
\cadre{
\begin{pseudo}

\Module{secondeMinuit}{moment\In~:~Moment}{entier}
    
    \Return moment.heure * 3600 + moment.minute * 60 + moment.seconde

\EndModule
\end{pseudo}
}
\end{Emphase}

\begin{Emphase}{Exercice 2 -- Conversion secondes-moment}
\cadre{
\begin{pseudo}
\Module{secondesVersMoments}{secondes\In~:~entier}{Moment}

    \Decl moment~:~Moment
    \Let moment \Gets $\displaystyle \left\{ \dfrac{\text{secondes}}{3600},
    \dfrac{\text{secondes~MOD~} 3600}{60}, \text{secondes~MOD~} 60 \right\}$
    \Return moment

\EndModule
\end{pseudo}
}
\end{Emphase}

\begin{Emphase}{Exercice 3 -- Temps écoulé entre 2 moments}
\cadre{
\begin{pseudo}
\Module{écartEntreMoments}{moment1, moment2~:~Moments}{entier}

    \Return{SecondesMinuit(moment1) - SecondesMinuit(moment2)}

\EndModule

\end{pseudo}
}
\end{Emphase}

\begin{Emphase}{Exercice 4 -- Milieu de deux points}
\cadre{
\begin{pseudo}
\Module{MilieuSegment}{a, b~:~Points}{Point}

    \Return $\displaystyle \left\{ \dfrac{a.x + b.x}{2} , \dfrac{a.y + b.y}{2}
    \right\}$

\EndModule

\end{pseudo}
}
\end{Emphase}

\begin{Emphase}{Exercice 5 -- Distance entre deux points}
\cadre{
\begin{pseudo}
\Module{LongueurSegment}{a, b~:~Points}{entier}

    \Return $\sqrt{(b.x - a.x)^2 + (b.y - a.y)^2}$

\EndModule

\end{pseudo}
}
\end{Emphase}

\begin{Emphase}{Exercice 6 -- Un cercle}
\cadre{
\begin{pseudo}
\Struct{Cercle}
    
    \Decl centre~:~Point
    \Decl rayon~:~réel

\EndStruct
\\
\Module{SurfaceCercle}{cercle\In~:~Cercle}{réel}

    \Return $\pi$ * (cercle.rayon)$^2$

\EndModule
\\
\Module{CréeCercle}{a, b~:~Points}{Cercle}

    \Decl cercle~:~Cercle
    \Let cercle.centre \Gets a
    \Let cercle.rayon \Gets LongueurSegment(a,b) / 2
    \Return cercle

\EndModule
\\
\Module{pointDansCercle}{point~:~Point, cercle~:~Cercle}{booléen}

    \Return LongueurSegment(point, cercle.centre) < cercle.rayon

\EndModule
\\
\Module{IntersectionCercle}{cercle1, cercle2~:~Cercles}{booléen}

    \Decl distanCentre~:~réel
    \Let distanceCentre \Gets LongueurSegment(cercle1.centre, cercle2.centre)
    \Return distanceCentre - (cercle1.rayon * 2 - cercle2.rayon * 2) $\le$ 0

\EndModule

\end{pseudo}
}
\end{Emphase}

\newpage

\begin{Emphase}{Exercice 7 -- Un rectangle}
\cadre{
\begin{pseudo}

\Struct{Rectangle}

    \Decl bg~:~Point
    \Decl hd~:~Point

\EndStruct
\\
\Module{périmètreRectangle}{rect~:~Rectangle}{réel}

    \Decl base, hauteur~:~réels
    \Let base \Gets LongueurSegment(rect.bg.x, rect.hd.x) 
    \Let hauteur \Gets LongueurSegment(rect.bg.y, rect.hd.y) 
    \Return 2 * base + 2 * hauteur

\EndModule
\\
\Module{aireRectangle}{rect\In~:~Rectangle}{réel}
    
    \Decl base, hauteur~:~réels
    \Let base \Gets LongueurSegment(rect.bg.x, rect.hd.x) 
    \Let hauteur \Gets LongueurSegment(rect.bg.y, rect.hd.y) 

    \Return base * hauteur

\EndModule
\\
\Module{pointDansRectangle}{rect~:~Rectangle, point~:~Point}{booléen}

    \Return (rect.bg.x $\le$ point.x $\le$ rect.hd.x) ET (rect.bg.y $\le$ point.y
    $\le$ rect.hd.y)

\EndModule
\\
\Module{pointSurBordRectangle}{rect~:~Rectangle, point~:~Point}{booléen}

\Decl surBaseBas, surBaseHaut, surHauteurGauche, surHauteurDroit~:~booléens

\Let surBaseBas \Gets rect.bg.x $\le$ point.x $\le$ rect.hd.x ET (rect.bg.y = point.y)
\Let surBaseHaut \Gets rect.bd.x $\le$ point.x $\le$ rect.hd.x ET (rect.hd.y = point.y)
\Let surCotéGauche \Gets rect.bg.y $\le$ point.y $\le$ rect.hd.y ET (rect.bg.x = point.x)
\Let surCotéDroit \Gets rect.bg.y $\le$ point.y $\le$ rect.hd.y ET (rect.hd.x = point.x)

\Return surBaseBas OU surBaseHaut OU surCotéGauche OU surCotéDroit

\EndModule


\end{pseudo}
}
\end{Emphase}   
