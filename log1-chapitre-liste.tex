%=================
\chapter{La liste}
%=================


\marginicon{objectif}

Imaginons qu’on désire manipuler par programme une liste de contacts ou
encore une liste de rendez-vous. Cette liste va varier ; sa taille
n’est donc pas fixée. Utiliser un tableau à cet effet n’est pas
l’idéal. En effet, la taille d’un tableau, qu’il soit statique ou
dynamique, ne peut plus changer une fois le tableau créé. Il faudrait
le sur-dimensionner, ce qui n’est pas économe.

Il serait intéressant de disposer d’une structure qui offre toutes les
facilités d’un tableau tout en pouvant «~grandir~» si nécessaire.
Construisons une telle structure de données et appelons-la «~Liste~»
pour rester en phase avec son appellation commune en Java.


%========================
\section{La classe Liste}
%=========================

Nous verrons plus loin comment la réaliser en pratique mais nous pouvons
déjà définir le comportement qu’on en attend (les méthodes qu’elle doit
fournir)

\cadre{
\begin{pseudo}
	\Class{Liste de T}
		\RComment T est un type quelconque
		\Private
			\LComment sera complété plus tard	
		\Public
			\ConstrSign{Liste de T}{}			
				\RComment construit une liste vide
			\MethodSign{get}{pos : entier}{T}
				\RComment donne un élément en position pos
			\MethodSign{set}{pos : entier, valeur : T}{}
				\RComment modifie un élément en position pos
			\MethodSign{taille}{}{entier}
				\RComment donne le nombre d’éléments
			\MethodSign{ajouter}{valeur : T}{}
				\RComment ajoute un élément en fin de liste
			\MethodSign{insérer}{pos : entier, valeur : T}{}
				\RComment insère un élément en position pos
			\MethodSign{supprimer}{}{}
				\RComment supprime le dernier élément
			\MethodSign{supprimerPos}{pos : entier}{}
				\RComment supprime l'élément en position pos
			\MethodSign{supprimer}{valeur : T}{booléen}
				\RComment supprime l'élément de valeur donnée
			\MethodSign{vider}{}{}
				\RComment vide la liste
			\MethodSign{estVide}{}{booléen}
				\RComment la liste est-elle vide ?
			\MethodSign{existe}{valeur \In : T, poso \Out : entier}{}
				\RComment recherche un élément
		\EndClass
\end{pseudo}
}

\bigskip

Quelques précisions s’imposent :

\begin{liste}
	\item 
		Comme les tableaux, les listes peuvent contenir des éléments de
		n’importe quel type tout en restant uniforme au sein d’une même liste
		(on pourra manipuler une liste d’entiers, une liste de contacts, ...
		mais pas mélanger). Il serait rédhibitoire de devoir définir une Liste
		pour chaque type d’éléments. On utilise dès lors la possibilité en OO
		d’écrire un code \textbf{générique}\footnote{\ on parle aussi de
		«~template~».}. Le «~T~» dans la définition de la classe indique le
		type des éléments qui sera spécifié lors de l’utilisation de la classe.
		On écrira par exemple «~\textstyleCodeInsr{Liste d’entiers}~» pour
		utiliser une liste d’entiers.
	\item 
		Les méthodes «\textstyleCodeInsr{~get~}» et «\textstyleCodeInsr{~set~}»
		permettent de connaitre ou modifier un élément de la liste. On
		considère que le premier élément de la liste est en position 1.
	\item 
		«\textstyleCodeInsr{~ajouter~}» ajoute un élément en fin de liste (elle
		grandit donc d’une unité)
	\item 
		«\textstyleCodeInsr{~insérer~}» insère un élément à une position donnée
		(entre 1 et taille+1). L’élément qui s’y trouvait est décalé
		d'une position ainsi que tous les éléments suivants.
	\item 
		La version de «\textstyleCodeInsr{~supprimer~}» avec une position en
		paramètre supprime un élément d'une position donnée en
		décalant les éléments suivants. On pourrait imaginer une technique plus
		rapide consistant à placer le dernier élément à la place de l’élément
		supprimé mais ce faisant on changerait l’ordre relatif des éléments ce
		qui va à l’encontre de l’idée intuitive qu’on se fait d’une liste.
		Cette amélioration pourrait plutôt s’envisager dans une structure de
		type \textbf{ensemble} pour lequel il n’y a pas d’ordre relatif entre
		les éléments.
	\item 
		La version de «\textstyleCodeInsr{~supprimer~}» avec une valeur en
		paramètre enlève un élément de valeur donnée. Elle retourne un booléen
		indiquant si la suppression a pu se faire ou pas (ce qui sera le cas si
		la valeur n’est pas présente dans la liste). Si la valeur existe en
		plusieurs exemplaires, la méthode n’en supprime que la première
		occurrence.
	\item 
		La méthode «\textstyleCodeInsr{~existe~}» permet de savoir si un élément
		donné existe dans la liste. 
	\item 
		si c’est le cas, elle donne aussi sa position.
	\item 
		si l’élément n’existe pas, le paramètre ‘\textstyleCodeInsr{pos}’ est
		indéterminé 
	\item 
		si l’élément est présent en plusieurs exemplaires, la méthode donne la
		position de la première occurrence.
	\item 
		En pratique, il serait intéressant de chercher un élément à partir d’une
		partie de l’information qu’elle contient mais c’est difficile à
		exprimer de façon générique c'est-à-dire lorsque le
		type n'est pas connu à priori.
\end{liste}

\bigskip

{\sffamily\bfseries\scshape
Exemple : recherche du minimum}

Dans le chapitre sur les tableaux, vous avez fait un exercice consistant
à afficher tous les indices où se trouve le minimum d’un tableau.
Reprenons-le et modifions-le afin qu’il retourne la liste des indices
où se trouvent les différentes occurrences du minimum. On pourrait
l’écrire ainsi :

\cadre{
\begin{pseudo}
\footnotesize
	\Module{indicesMinimum}{tab : tableau [1 à n] d'entiers}{Liste d’entiers}
		\Decl min : entier
		\Decl indicesMin : Liste d'entiers
		\Let min \Gets tab[1]
		\Let indicesMin \Gets \K{nouvelle} Liste d'entiers()
		\Stmt indicesMin.ajouter( 1 )
		\For{i \K{de} 2 \K{à} n}
			\Switch{} 
				\Case{tab[i] = min}
					\Stmt indicesMin.ajouter( i )
				\Case{tab[i] < min}
					\Stmt indicesMin.vider() 
					\Stmt indicesMin.ajouter( i )
					\Let min \Gets tab[i]
				\Case{tab[i] > min}
					\LComment rien à faire dans ce cas	
			\EndSwitch
		\EndFor
		\Return indicesMin
	\EndModule
\end{pseudo}
}

\bigskip

%===================================
\section{Comment implémenter l’état}
%===================================

Cette liste est bien utile mais comment la réaliser en pratique ?
Comment représenter une liste variable d’éléments ? Pour
l'instant, la seule structure qui peut accueillir
plusieurs éléments de même type est le tableau. Nous allons donc
prendre comme attribut principal de la liste, un tableau que nous
appellerons \textstyleCodeInsr{éléments}. Comment, dès lors, contourner
le problème de la limitation de la taille de ce tableau ?

Repartons donc de la notion de tableau et tentons de comprendre sa
limitation. Lors de sa création, un tableau se voit attribuer un espace
bien précis et contigu en mémoire. Il se peut très bien que
l'espace «~juste après~» soit occupé par une autre
variable ce qui l'empêche de grandir. La parade est
claire : si un \ tableau s’avère trop petit lors de son utilisation, il
suffit d’en créer un autre plus grand ailleurs en mémoire et d’y
recopier tous les éléments du premier. Évidemment, cette opération est
coûteuse en temps et on cherchera à l’effectuer le moins souvent
possible.

\textbf{Quelle taille donner au nouveau tableau} ? L’idée qui vient
immédiatement est d’augmenter la taille d’une unité afin d’accueillir
le nouvel élément mais cette approche implique de fréquents
agrandissements. Il est plus efficace d’augmenter la taille
proportionnellement, par exemple en la multipliant par un facteur 2.

\begin{center}
\tablehead{}
\begin{supertabular}{|m{0.259cm}|m{0.259cm}|m{0.259cm}|m{0.087999985cm}m{0.46000004cm}m{0.087999985cm}|m{0.25300002cm}|m{0.259cm}|m{0.259cm}|m{0.15299998cm}|m{0.15299998cm}|m{0.17cm}|}
\hhline{---~~~------}
 1 &
 5 &
 7 &
~
 &
 ${\Rightarrow}$ &
~
 &
 1 &
 5 &
 7 &
 . &
 . &
 .\\\hhline{---~~~------}
\end{supertabular}
\end{center}


\textbf{Taille logique et taille physique}. À tout moment, le tableau
aura une et une seule taille même si celle-ci pourra changer au cours
du temps. Puisqu’on multipliera la taille du tableau par 2 pour des
raisons d’efficacité, il y aura toutefois une différence entre la
\textbf{taille physique} d’un tableau et sa \textbf{taille logique}. La
taille physique est le nombre de cases réservées pour le tableau alors
que la taille logique est le nombre de cases effectivement occupées.
Dans ce qui suit, on s'arrangera pour que les cases
occupées soient groupées à gauche du tableau (il n'y a
pas de trou). Pour l’utilisateur, seule la taille logique a un sens (on
lui cache les détails d’implémentation).


\textbf{Exemple} : pour le tableau suivant, la taille logique est de 6
(c’est cette taille qui a du sens pour l’utilisateur de la liste) et la
taille physique est de 8.

\begin{center}
\tablehead{}
\begin{supertabular}{|m{0.506cm}|m{0.506cm}|m{0.506cm}|m{0.506cm}|m{0.506cm}|m{0.506cm}|m{0.506cm}|m{0.523cm}|}
\hline
 2 &
 5 &
 4 &
 8 &
 3 &
 12 &
\centering  . &
\centering\arraybslash  .\\\hline
\end{supertabular}
\end{center}

Quand il faut insérer un élément (en position valide) ou en ajouter un
en fin de liste, deux cas se présentent :

\begin{liste}
	\item 
		Si la taille logique est plus petite que la taille physique, il suffit
		d’ajouter l’élément dans le tableau et d’adapter la taille logique.
	\item 
		Par contre, si la taille logique est égale à la taille physique, il faut
		procéder à un agrandissement du tableau.
\end{liste}

Présentons les attributs nécessaires et l'algorithme
d’agrandissement du tableau.

\cadre{
\begin{pseudo}
\footnotesize
	\Class{Liste de T}
		\Private
			\Decl éléments : \K{tableau} de T
			\Decl tailleLogique : entier
			\Decl taillePhysique : entier
		\Private
			\Method {agrandir}{}{}
				\Decl i : entier
				\Decl nouveauTab : \K{tableau} de T
				\Let taillePhysique \Gets taillePhysique * 2
				\Let nouveauTab \Gets \K{nouveau} \K{tableau} [ 1 à taillePhysique ] de T
				\For{i \K{de} 1 \K{à} tailleLogique}
					\Let nouveauTab[ i ] \Gets éléments[ i ]
				\EndFor
				\Let éléments \Gets nouveauTab
			\EndMethod
		\EndClass
	\end{pseudo}
}

\bigskip

\textbf{Réduction du tableau}. Tout comme on agrandit le tableau si
nécessaire, on pourrait le réduire lorsque des suppressions d’éléments
le rendent sous-utilisé (par exemple lorsque la taille logique devient
inférieure au tiers de la taille physique). Nous n’abordons toutefois
pas cette problématique ici.


%=======================================
\section{Implémentation du comportement}
%=======================================

Nous avons à présent toutes les cartes en main pour écrire les méthodes
publiques de la classe.

\cadre{
\begin{pseudo}
	\Constr{Liste de T()}{}
		\Let tailleLogique \Gets 0
		\RComment La liste est vide au départ
		\Let taillePhysique \Gets 32
		\RComment une bonne valeur pour commencer
		\Let éléments \Gets \K{nouveau} \K{tableau} [ 1 à taillePhysique ] de T
	\EndConstr
\end{pseudo}
}

\bigskip

\cadre{
\begin{pseudo}
	\Method{get}{pos : entier}{T}
		\If{ pos < 1 OU pos > tailleLogique}
			\K{erreur} position invalide
		\EndIf
		\Return éléments[ pos ]
	\EndMethod
\end{pseudo}
}

\bigskip

\cadre{
\begin{pseudo}
	\Method{set}{pos : entier, valeur : T}{}
		\If{ pos < 1 OU pos > tailleLogique}
			\K{erreur} position invalide
		\EndIf
		\Let éléments[ pos ] \Gets valeur
	\EndMethod
\end{pseudo}
}

\bigskip

\cadre{
\begin{pseudo}
	\Method{taille}{}{entier}
		\Return tailleLogique
		\RComment Et pas la taille physique !
	\EndMethod
\end{pseudo}
}

\bigskip

\cadre{
\begin{pseudo}
	\Method{ajouter}{valeur : T}{}
		\If{tailleLogique = taillePhysique}
			\Stmt agrandir()
			\RComment méthode privée détaillée supra
		\EndIf
		\Let tailleLogique \Gets tailleLogique + 1
		\Let éléments[ tailleLogique ] \Gets valeur
	\EndMethod
\end{pseudo}
}

\bigskip

\cadre{
\begin{pseudo}
	\Method{insérer}{pos : entier, valeur : T}{}
		\If{ pos < 1 OU pos > tailleLogique+1}
			\K{erreur} position invalide
		\EndIf
		\If{tailleLogique = taillePhysique}
			\Stmt agrandir()
		\EndIf
		\Stmt décalerDroite( pos )
		\RComment voir ci-dessous
		\Let tailleLogique \Gets tailleLogique + 1
		\Let éléments[ pos ] \Gets valeur
	\EndMethod
\end{pseudo}
}

\bigskip

\cadre{
\begin{pseudo}
	\Method{supprimer}{}{}
		\If{tailleLogique = 0}
			\K{erreur} liste vide
		\EndIf
		\Let tailleLogique \Gets tailleLogique - 1
	\EndMethod
\end{pseudo}
}

\bigskip

\cadre{
\begin{pseudo}
	\Method{supprimerPos}{pos : entier}{}
		\If{ pos < 1 OU pos > tailleLogique}
			\K{erreur} position invalide
		\EndIf
		\Stmt décalerGauche( pos + 1 )
		\Let tailleLogique\Gets tailleLogique - 1
	\EndMethod
\end{pseudo}
}

\bigskip

\cadre{
\begin{pseudo}
	\Method{supprimer}{valeur: T}{booléen}
		\Decl existe : booléen
		\Decl pos : entier
		\Let existe \Gets existe(valeur, pos)
		\If{existe}
			\Stmt supprimer( pos )
		\EndIf
		\Return existe
	\EndMethod
\end{pseudo}
}

\bigskip

\cadre{
\begin{pseudo}
	\Method{vider}{}{}
		\Let tailleLogique \Gets 0 
		\RComment Les éléments ne sont pas effacés mais sont ignorés
	\EndMethod
\end{pseudo}
}

\bigskip

\cadre{
\begin{pseudo}
	\Method{estVide}{}{booléen}
		\Return tailleLogique = 0
	\EndMethod
\end{pseudo}
}

\bigskip

\cadre{
\begin{pseudo}
	\Method{existe}{valeur\In : T, pos\Out : entier}{booléen}
		\Let pos \Gets 1
		\LComment Rq : le ET ci-dessous est une évaluation 
		court-circuitée (cf. chapitre 3)
		\While{ pos{${\leq}$}tailleLogique ET éléments[ pos ]{${\neq}$}valeur}
			\Let pos \Gets pos + 1
		\EndWhile
		\Return pos{${\leq}$}tailleLogique
	\EndMethod
\end{pseudo}
}

\bigskip

\cadre{
\begin{pseudo}
	\LComment Ces méthodes-ci sont privées
	\bigskip

	\Method{décalerDroite}{début : entier}{}
		\LComment Décale tous les éléments d'une position vers
		la droite à partir de début
		\Decl i : entier
		\For{i \K{de} tailleLogique \K{à} début \K{par} –1}
			\Let éléments[ i + 1 ] \Gets éléments[ i ]
		\EndFor
	\EndMethod

	\bigskip
	
	\Method {décalerGauche}{début : entier}{}
		\LComment Décale toutes les éléments d'une position vers
		la gauche à partir de début ; 
		\LComment ce paramètre vaut toujours au moins 2.
		\Decl i : entier
		\For{i \K{de} début \K{à} tailleLogique}
			\Let éléments[ i - 1 ] \Gets éléments[ i ]
		\EndFor
	\EndMethod
\end{pseudo}
}

\bigskip

\marginicon{attention}
\textstyleMotCl{La recherche se fait sur un élément complet. }


Prenons comme exemple une liste de contacts.
Lors d'une recherche, on doit fournir
\textstyleMotCl{tout} le contact à
rechercher. Il s'agit juste de savoir
s'il est présent et où. Une autre méthode intéressante
serait de retrouver un contact à partir d'une partie
de l'information, par exemple son nom. Cette méthode
est fort proche de notre méthode de recherche mais il serait très
difficile de l'écrire génériquement. On vous demandera
d'écrire explicitement une telle méthode de recherche
en cas de besoin.


%====================================
\section{Et sans tableau dynamique ?}
%=====================================

Certains langages (c’est le cas de Cobol) ne permettent pas de créer
dynamiquement un nouveau tableau. Il vous faudra travailler avec un
tableau classique en le créant suffisamment grand.


Les algorithmes d’ajout/suppression/recherche vus pour la liste peuvent
être appliqués tels quels à un tableau statique à une modification près
: lors d’un ajout dans un tableau plein, on ne peut pas l’agrandir; il
faut générer une erreur.


%==================
\section{Exercices}
%===================

\begin{Exercice}{Liste des premiers entiers}
	Écrire un module qui reçoit un entier n en paramètre et retournant la
	liste contenant les entiers de 1 à n dans l'ordre
	décroissant. On peut supposer que n est positif.
\end{Exercice}
	
\begin{Exercice}{	Liste des premiers entiers}
	Écrire un module qui reçoit une liste de chaines en paramètre et
	supprime de cette liste tous les éléments de valeur donnée en
	paramètre. L'algorithme retournera le nombre de
	suppressions effectuées
\end{Exercice}
	
\begin{Exercice}{Somme d'une liste}
	Écrire un module qui calcule la somme des éléments d’une liste
	d’entiers.
\end{Exercice}

\begin{Exercice}{Les extrêmes}
		Écrire un module qui supprime le minimum et le maximum des éléments
		d’une liste d’entiers. On peut supposer que le maximum et le minimum
		sont uniques.
\end{Exercice}

\begin{Exercice}{Anniversaires}
		Écrire un module qui reçoit une liste de Personne (nom + prénom + date
		de naissance ; cf. exercice dans le chapitre OO) et retourne la liste
		de ceux qui sont nés durant un mois passé en paramètre (entier entre 1
		et 12)
\end{Exercice}
	
\begin{Exercice}{Concaténation de deux listes}
		Écrire un module qui reçoit 2 listes et ajoute
		à la suite de la première les éléments de la seconde; la seconde liste
		n'est pas modifiée par cette opération.
\end{Exercice}

\begin{Exercice}{Fusion de deux listes}
	
		Soit deux listes \textbf{ordonnées}
		d'entiers (redondances possibles). Écrire un module
		qui les fusionne. Le résultat est une liste encore ordonnée contenant
		tous les entiers des deux listes de départ (qu'on
		laisse inchangées).

		Exemple : Si les 2 listes sont (1, 3, 7, 7) et (3, 9), 
		le résultat est (1, 3, 3, 7, 7, 9).
\end{Exercice}

\begin{Exercice}{Éliminer les doublons d'une liste}
		Soit une liste \textbf{ordonnée} 
		d'entiers avec de possibles redondances. Écrire un
		module qui enlève les redondances de la liste.
				
		Exemple : Si la liste est (1, 3, 3, 7, 8, 8, 8),
		le résultat est (1, 3, 7, 8).

		\begin{enumerate}[label=\alph*)]
			\item 
				Faites l'exercice en créant une \textbf{nouvelle
				liste} (la liste de départ reste inchangée)
			\item 
				Refaites l'exercice en \textbf{modifiant}
				la liste de départ (pas de nouvelle liste)
		\end{enumerate}
\end{Exercice}

\begin{Exercice}{La chaine}
	Nous utilisons déjà un type chaine.
	Faisons-en une classe et dotons-la de
	méthodes souvent utiles.

	\begin{center}
	\tablehead{}
	\begin{supertabular}{|m{11.597cm}|}
		\hline
		\centering\arraybslash  Chaine\\\hline
		 {}- listeCar : Liste de caractères\\\hline
		{ + Chaine( )\ \ // crée une chaine vide}

		{ {+ }Chaine( tab : tableau [1 à n] de caractères )
		\ \ // crée une chaine initialisée au contenu du tableau}

		{ {+ getCaractère ( i : entier ) }\textsf{→} {caractère}}

		{ + concatène( autre : Chaine ) // ajoute autre à
		la fin de la chaine}

		{ {+ sousChaine( début,
		taille : entiers ) }\textsf{→} {Chaine
		\ \ // donne la chaine de taille donnée commençant à la
		\ \ // position début. La chaine n'est pas modifiée}}

		{ {+ existe(
		chaineCherchée : Chaine, position~}\textsf{↑}
		:{ entier ) }\textsf{→}
		{booléen
		\ \ // indique si la chaineCherchée est une sous-chaine;
		\ \ // Si oui, donne la position de la sous-chaine
		\ \ // Si la sous-chaine existe en plusieurs exemplaires, on
		\ \ // donne la première occurrence.}}

		{ {+ remplacer(
		chaineARemplacer, par : Chaine ) }\textsf{→}
		{booléen
		\ \ // retourne un booléen indiquant si au moins un
		\ \ // remplacement a eu lieu (ce qui ne sera pas le cas si la}}

		{ \ \ // chaine à remplacer n’était pas
		présente)}

		 {+ taille() }\textsf{→}
		{entier \ // nb de caractères de la
		chaine}\\\hline
	\end{supertabular}
	\end{center}
	
	Par facilité, on supposera qu’écrire une chaine littéralement revient à
	créer un objet Chaine initialisé de la même façon. (ex : «~Bonjour~»
	est équivalent à «~nouveau Chaine (tab)~» où tab est un tableau
	contenant les caractères 'B',
	'o',
	'n',
	'j',
	'o',
	'u',
	'r'.

	\textbf{Utilisation de la chaine.}

	\begin{enumerate}[label=\alph*)]
		\item 
			Écrire un module qui reçoit un nom complet
			sous la forme «~nom, prénom~» (le nom et le prénom sont donc séparés
			par une virgule et un espace) et dissocie le nom du prénom. 
		\item 
			Écrire un module qui remplace dans une chaine
			tous les «~HEB-ESI~» par «~ESI~»
		\item 
			Idem mais pour le changement inverse 
			(remplacer tous les «~ESI~» par «~HEB-ESI~»)
	\end{enumerate}
\end{Exercice}

	
\begin{Exercice}{La liste ordonnée}
		Une recherche dans une liste implique un parcours complet de la liste en
		cas de recherche infructueuse. La recherche pourrait être plus rapide
		si la liste était ordonnée (en utilisant la recherche dichotomique). La
		contrainte principale est qu'il faudra maintenir le
		caractère ordonné de la liste (notamment en cas
		d'ajout). Écrivez les modules suivants, de façon à ce
		que la liste reste ordonnée.

		\cadre{
		\begin{pseudo}
			\ModuleSign{ajouterOrdonné}{liste : Liste de T, valeur : T}{}
			\ModuleSign{enleverOrdonné}{liste : Liste de T, valeur : T}{booléen}
			\LComment retourne faux si valeur pas présente. 
			\LComment Si la valeur est présente en plusieurs exemplaire, en enlève une.
			\ModuleSign{existeOrdonné}{liste\In : Liste de T, valeur\In : T, pos\Out: entier}{booléen}
			\LComment si la valeur n'est pas trouvée, pos donne la position où elle aurait dû être.
		\end{pseudo}
		}

\end{Exercice}

\begin{Exercice}{Trier des mots}
		Écrivez un algorithme qui lit une série de mots (se terminant par une
		chaine vide) sans aucun ordre et les affiche dans l’ordre alphabétique.
		Vous pouvez utiliser les modules écrits lors de
		l'exercice précédent.
		
\end{Exercice}

\begin{Exercice}{Éviter les doublons}
		Modifiez l’exemple ci-dessus pour que deux mots identiques ne soient
		introduits qu’une seule fois dans la liste.
\end{Exercice}

\begin{Exercice}{L'ensemble}
		La notion d’ensemble fini est une notion qui vous est déjà 
		familière pour l’avoir rencontrée dans plusieurs cours. Nous rappelons
		certaines de ses propriétés et opérations. 
				
		Étant donnés deux ensembles
		finis \textbf{S} et \textbf{T} ainsi qu’un élément \textbf{x} :

		\begin{liste}
		\item 
			\textbf{x} {${\in}$} \textbf{S} signifie que l’élément \textbf{x}
			est un élément de l’ensemble \textbf{S}.
		\item 
			L’ensemble vide, noté \textbf{${\emptyset}$} 
			est l’ensemble qui n’a pas d’élément 
			(\textbf{x} {${\in}$} \textbf{${\emptyset}$} 
			est faux quel que soit \textbf{x} ).
		\item 
			L’ordre des éléments dans un ensemble n’a
			aucune signification, l’ensemble \{1,2\} est
			identique à \{2,1\}.
		\item 
			Un élément \textbf{x} ne peut
			pas être plus d’une fois élément d’un même ensemble 
			(pas de répétition).
		\item 
			L’union \textbf{S ${\cup}$ T} 
			est l’ensemble contenant les éléments qui sont dans 
			\textbf{S} ou (non exclusif) dans \textbf{T}.
		\item 
			L’intersection \textbf{S ${\cap}$ T} 
			est l’ensemble des éléments qui sont à la fois 
			dans \textbf{S} et dans \textbf{T}.
		\item 
			La différence \textbf{S {\textbackslash} T} 
			est l’ensemble des éléments qui sont 
			dans \textbf{S} mais pas dans \textbf{T}.
		\end{liste}
		
		Créez la classe \textstyleCodeInsr{Ensemble}
		décrite ci-dessous.
		
		\cadre{
		\begin{pseudo}
			\Class{Ensemble de T}{} 
			\RComment T est le type des éléments de l'ensemble
			
				\Public
				\ConstrSign{Ensemble de T}{} 
				\RComment construit un ensemble vide
				\MethodSign{ajouter}{élt : T}{}
				\RComment ajoute l'élément à l'ensemble
				\MethodSign{enlever}{élt : T}{})
				\RComment enlève un élément de l'ensemble
				\MethodSign{contient}{élt : T}{booléen}
				\RComment dit si l'élément est présent
				\MethodSign{estVide}{}{booléen}
				\RComment dit si l'ensemble est vide
				\MethodSign{taille}{}{entier}
				\RComment donne la taille de l'ensemble
				\MethodSign{union}{autreEnsemble : Ensemble de T}{Ensemble de T}
				\MethodSign{intersection}{autreEnsemble : Ensemble de T}{Ensemble de T}
				\MethodSign{moins}{autreEnsemble : Ensemble de T}{Ensemble de T}
				\MethodSign{liste}{}{Liste de T}
				\RComment conversion en liste
		\EndClass
	\end{pseudo}
	}
	\bigskip

	\textstylePolicepardfaut{Quelques remarques :}

	\begin{liste}
		\item 
			La méthode d'ajout (resp. de suppression) n'a
			pas d'effet si l'élément est déjà
			(resp. n'est pas) dans l'ensemble.
		\item 
			Les méthodes \textstyleCodeInsr{union()}, 
			\textstyleCodeInsr{intersection()} et 
			\textstyleCodeInsr{moins()} retournent un troisième ensemble, 
			résultat des 2 premiers sans toucher
			à ces 2 ensembles. On aurait pu envisager des méthodes modifiant
			l'ensemble sur lequel on les appelle.
		\item 
			La méthode \textstyleCodeInsr{liste()}
			est nécessaire si on veut parcourir les éléments de
			l'ensemble (par exemple pour les afficher).
	\end{liste}
\end{Exercice}

\begin{Exercice}{Autres opérations ensemblistes}
	Nous avons défini des opérations ensemblistes ne touchant pas aux
	ensembles de départ. Que deviennent-elles si on considère
	qu'elles \textbf{modifient}
	l'ensemble sur lequel elles sont appliquées ?
\end{Exercice}


\begin{Exercice}{Rendez-vous}
	Soit la structure «\textstyleCodeInsr{~RendezVous~}» composée d’une date
	et d’un motif de rencontre. Écrire un module qui reçoit une liste de
	rendez-vous et la met à jour en supprimant tous ceux qui sont désormais
	passés. 

	\textstyleCodeInsr{\textrm{Rappel}}\textstyleCodeInsr{\textrm{ : la date
	du jour s'obtient par le constructeur de
	}}\textstyleCodeInsr{Date}\textstyleCodeInsr{\textrm{ sans
	paramètre.}}
\end{Exercice}
