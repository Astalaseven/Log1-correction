
\chapter[Les traitements de rupture]{
Les traitements de rupture}
{
Dans ce chapitre, on étudie une classe de problèmes qui peuvent tous se
résoudre avec un même type d'algorithme :
l'algorithme de rupture. À la fin du chapitre vous
devrez être capable de :}

\begin{center}
 [Warning: Image ignored] % Unhandled or unsupported graphics:
%\includegraphics[width=1.268cm,height=1.268cm]{log1-img/log1-img158}

\end{center}
\liststyleListv
\begin{itemize}
\item {
Détecter qu'on se trouve bien face à un problème qui
peut entrer dans le cadre d'un algorithme de rupture.}
\item {
Adapter le squelette général de l'algorithme de rupture
au problème donné.}
\end{itemize}
\section{Le classement complexe}
{
Dans le chapitre sur les tris, nous avons abordé naturellement la notion
du classement des données. Néanmoins, les données concernées étaient
dans ces cas des variables dites «~simples~» : nombres (entiers ou
réels) ou chaines, variables pour lesquelles la relation d’ordre est
évidente (ordre numérique ou alphabétique). Les algorithmes mis en
œuvre peuvent facilement s’adapter pour d’autre types, par exemple des
objets Date, où l’opérateur de comparaison est remplacé par la méthode
«~estAntérieure()~».}

{
Noter que dans la plupart des cas, les données composées de plusieurs
champs ou attributs, telles les variables structurées, ne possèdent pas
de relation d’ordre naturelle ; par exemple les points d’un espace à
deux ou trois dimensions, ou une structure reprenant toutes les
informations figurant sur une carte d’identité. Si, dans ce dernier
cas, on veut ordonner une série de telles données, il faudra choisir un
premier critère de classement (par exemple le nom ou la date de
naissance) et en cas d’égalité sur le premier critère (deux personnes
peuvent avoir un même nom ou être nées le même jour), il faudra
départager sur un second critère, et éventuellement sur un troisième,
et ainsi de suite.}

{
Ces critères de classement sont bien entendu arbitraires, et dépendent
de l’information qu’on veut retirer de l’ensemble des données. Notons
aussi que l’ordre de classement peut être pour chaque critère croissant
ou décroissant.}

{
Prenons l’exemple d’une structure Etudiant, contenant les champs
matricule, nom, prénom, dateNais (date de naissance) et option (G, I ou
R). Pour l’exemple, considérons une liste de 6 étudiants :}

{
\ \ \textsf{29845\ \ \ \ Durant \ \ \ \ Kevin\ \ \ \ 20/01/84\ \ R}}

{\sffamily
\ \ 30125\ \ \ \ Dupont \ \ \ \ Fabrice\ \ \ \ 13/06/84\ \ G}

{\sffamily
\ \ 30351\ \ \ \ Simon\ \ \ \ André\ \ \ \ 18/11/84\ \ G}

{\sffamily
\ \ 30597\ \ \ \ Dupont \ \ \ \ Charles\ \ \ \ 9/07/84\ \ \ \ G}

{\sffamily
\ \ 31857\ \ \ \ Guilmant \ \ Léon\ \ \ \ 17/03/86\ \ R}

{\sffamily
\ \ 31886\ \ \ \ Durant \ \ \ \ Sam\ \ \ \ 30/05/84\ \ I}

{
Cette liste est classée sur le numéro de matricule. C’est un classement
simple réalisé sur un seul champ des données. Le numéro de matricule
étant dans ce cas-ci un \textbf{\textcolor{black}{identifiant}} des
données, le problème de devoir départager ne se pose pas.}

{
Si nous désirons à présent classer sur l’ordre alphabétique des noms, il
faut décider de départager en cas de noms identiques sur un autre
champ, de façon naturelle sur celui des prénoms. Ceci donnerait le
classement double suivant, en \textbf{majeur} sur le nom et en
\textbf{mineur} sur le prénom :}

{\sffamily
\ \ 30597\ \ \ \ Dupont \ \ \ \ Charles\ \ \ \ 9/07/84\ \ \ \ G}

{\sffamily
\ \ 30125\ \ \ \ Dupont \ \ \ \ Fabrice\ \ \ \ 13/06/84\ \ G}

{\sffamily
\ \ 29845\ \ \ \ Durant \ \ \ \ Kevin\ \ \ \ 20/01/84\ \ R}

{\sffamily
\ \ 31886\ \ \ \ Durant \ \ \ \ Sam\ \ \ \ 30/05/84\ \ I}

{\sffamily
\ \ 31857\ \ \ \ Guilmant \ \ Léon\ \ \ \ 17/03/86\ \ R}

{\sffamily
\ \ 30351\ \ \ \ Simon\ \ \ \ André\ \ \ \ 18/11/84\ \ G}

{
Supposons enfin que nous voulions grouper les étudiants par sections,
nous devons alors classer prioritairement sur l’option, départager sur
les noms et ensuite sur les prénoms. C’est alors un classement triple :
en \textbf{majeur} sur l’option, en \textbf{médian} sur le nom et en
\textbf{mineur} sur le prénom : }

{\sffamily
\ \ 30597\ \ \ \ Dupont \ \ \ \ Charles\ \ \ \ 9/07/84\ \ \ \ G}

{\sffamily
\ \ 30125\ \ \ \ Dupont \ \ \ \ Fabrice\ \ \ \ 13/06/84\ \ G}

{\sffamily
\ \ 30351\ \ \ \ Simon\ \ \ \ André\ \ \ \ 18/11/84\ \ G}

{\sffamily
\ \ 31886\ \ \ \ Durant \ \ \ \ Sam\ \ \ \ 30/05/84\ \ I}

{\sffamily
\ \ 29845\ \ \ \ Durant \ \ \ \ Kevin\ \ \ \ 20/01/84\ \ R}

{\sffamily
\ \ 31857\ \ \ \ Guilmant \ \ Léon\ \ \ \ 17/03/86\ \ R}

{
Remarque : le classement sur l’option n’est pas forcément un classement
alphabétique des trois lettres G, I, R ! Toute autre permutation de ces
trois lettres serait une possibilité~acceptable.}

{
En résumé, les exemples ci-dessus constituent des exemples de
\textbf{classements complexes}. On dira que des données sont classées
sur la \textbf{clé complexe} champ1 – champ 2 – … – champ n (où «~champ
i~» est un champ de la structure des données) si le classement se fait
prioritairement depuis le champ 1 jusqu’au champ n. Autrement dit, si
deux données ont tous leurs champs 1, 2,…,i égaux (i {\textless} n), le
classement se fait en départageant sur le champ i+1. L’indice du champ
correspond au \textbf{niveau} du classement complexe.}

{
Par exemple, le dernier classement ci dessus est un classement sur la
clé complexe option – nom – prénom. La terminologie «~majeur~»,
«~médian~» et «~mineur~» sont couramment utilisées lorsqu’au plus trois
champs interviennent dans le classement complexe.}

\section{La notion de rupture}
{
Nous nous plaçons dans le cadre d’un \textbf{ensemble logique} composé
d’\textbf{éléments} de structure identique (variables simples ou
structurées, enregistrements d’un fichier, éléments d’un tableau, d’une
liste ordonnée ou non) et qui sont l’objet d’un \textbf{traitement
itératif} où chaque élément de l’ensemble est parcouru. }

{
Nous parlons de \textbf{rupture} lorsque dans ce traitement itératif, on
constate que l’information courante que l’on souhaite traiter
n’appartient plus à l’ensemble (ou au sous-ensemble) des informations
déjà traitées précédemment. }

{
Les ruptures sont intimement liées au classement complexe des données,
et elles ne se qualifient que par rapport aux champs des données qui
interviennent dans ce classement. On parle donc de \textit{rupture sur
un champ} des données.}

{
Par exemple, dans le dernier classement des étudiants ci-dessus, il y a
rupture sur l’option au niveau de Durant Sam et de Durant Kevin. En
effet, ces deux étudiants délimitent les sous-ensembles d’étudiants
partageant une même option.}

{
Par contre, cela n’aurait pas de sens de parler de rupture sur l’option
dans les deux classements précédents. Ce champ n’appartenant pas à la
clé complexe pour le classement de ces données, la succession des
lettres G, I, R doit être considérée comme aléatoire dans ces deux cas
et sans signification pour le traitement itératif.}

{
Dans le 2\textsuperscript{ème} classement, nous pouvons parler de
rupture sur les noms : l’étudiant Durant Kevin met fin au sous-ensemble
des Dupont, et l’étudiant Guilmant met fin au sous-ensemble des Durant.
}

{
Dans le 1\textsuperscript{er} classement, qui est un classement simple
sur le numéro de matricule, on peut considérer qu’il n’y a qu’un
ensemble de données d’un seul tenant sans ruptures, ou alors qu’il y a
rupture à chaque étudiant, puisque chaque étudiant forme un
sous-ensemble isolé par son numéro de matricule, vu qu’ils sont
obligatoirement distincts.}

{
Notez que toute rupture à un niveau i de la clé de classement complexe
entraine une rupture aux niveaux supérieurs à i. Ainsi, dans le
classement triple, le passage à Durant Kevin entraine une rupture sur
l’option mais aussi sur le nom. Le fait qu’il y ait dans ce classement
deux Durant qui se succèdent doit être considéré comme une coïncidence.
Bien qu’ayant le même nom, ces deux étudiants appartiennent à deux
sous-ensembles différents lorsqu’on prend l’option en considération.}

{
Notons encore que d’un point de vue théorique, deux éléments ont un rôle
spécial et se singularisent quelque peu des autres : il s’agit du
\textbf{premier} et du \textbf{dernier} élément qui ont pour rôle de
délimiter l’ensemble logique. Le premier est singulier dans la mesure
où il n’est précédé par aucun autre élément, et de plus sa présence
détermine l’existence de l’ensemble (par opposition à un ensemble
vide). Le dernier est particulier parce qu’il termine l’ensemble des
données. Remarquez que RIEN dans ces deux éléments ne signale ces
particularités! Autrement dit, il n’y a pas, à l’intérieur de ces
éléments une information particulière signifiant \textit{je suis le
premier élément} ou \textit{je suis le dernier}.}

\section[Traitement des ruptures dans un fichier]{Traitement des
ruptures dans un fichier}
{
Quel que soit l’ordre de tri des données d’un fichier, celui-ci contient
toujours une information importante signalant la fin des données, il
s’agit de la marque de fin de fichier, sa détection ayant pour effet de
faire passer EOF à vrai. Cette marque de fin de fichier constitue donc
la rupture principale d’un fichier, celle signalant la fin du parcours
itératif. }

{
Sans le savoir, nous avons donc déjà traité la rupture générale d’un
ensemble de données dans un fichier (c’est la rupture de «~niveau 0~»,
car elle n’est pas liée à un champ des données, et est naturellement
prioritaire sur ces champs). Pour illustrer cela, reprenons l’exemple
de la liste d’étudiants, que nous imaginons contenue dans un fichier
nommé Student dont les enregistrements sont de type Etudiant. Le
parcours de base de ce fichier est le suivant :}

{\sffamily
\textstyleMotCl{module} RuptureNiveau0(Student↓~: FichierEntrée
d’Etudiant)}

{\sffamily
\ \ enr : Etudiant}

{\sffamily
\textit{\ \ }Student.ouvrir()}

{\sffamily
\ \ enr \textrm{[F0DF?]} Student.lire()}

{\sffamily
\ \ \textstyleMotCl{tant} \textstyleMotCl{que} NON Student.EOF()
\textstyleMotCl{faire}\ \ \ \ \ \ \ \ }

{\sffamily
\ \ \ \ // traitement de l’enregistrement}

{\sffamily
\ \ \ \ enr \textrm{[F0DF?]} Student.lire()}

{\sffamily
\ \ \textstyleMotCl{fin} \textstyleMotCl{tant} \textstyleMotCl{que} }

{\sffamily
\ \ Student.fermer()}

{\sffamily
\textstyleMotCl{fin} \textstyleMotCl{module}}

{
Si nous voulons faire des statistiques globales sur l’ensemble des
étudiants (par ex. simplement les compter), le traitement de
l’information consiste à incrémenter un compteur, et l’algorithme
ci-dessus peut fonctionner quel que soit l’ordre de classement choisi.}

{
Passons à présent au «~niveau 1~» ; c’est-à-dire un traitement de
rupture correspondant à un classement complexe sur un champ. Imaginons
que nous voulions savoir quel est le nombre d’étudiants dans chaque
option. Une solution consisterait à avoir 3 compteurs, un par option.
On peut imaginer une façon plus judicieuse de faire à partir du dernier
classement~qui contient précisément les étudiants déjà groupés par
option : à chaque fois qu’il y a rupture sur l’option, on connait alors
le total d’étudiants dans l’option qui vient d’être parcourue. Ceci ne
nécessite qu’un seul compteur remis à 0 à chaque fois qu’une nouvelle
option est rencontrée, c’est-à-dire à chaque rupture. De plus
l’algorithme serait aussi fonctionnel quelle que soit le nombre
d’options. Voici une solution : }

{\sffamily
\textstyleMotCl{module} RuptureNiveau1(Student↓~: FichierEntrée
d’Etudiant)}

{\sffamily
\textstyleMotCl{\textmd{\ \ // on suppose les données classées en majeur
sur l’option}}}

{\sffamily
\ \ enr : Etudiant}

{\sffamily
\ \ saveOption : caractère}

{\sffamily
\ \ cpt : entier}

{\sffamily
\ \ Student.ouvrir()}

{\sffamily
\ \ enr \textrm{[F0DF?]} Student.lire()}

{\sffamily
\ \ \textstyleMotCl{tant} \textstyleMotCl{que} NON Student.EOF()
\textstyleMotCl{faire}\ \ }

{\sffamily
\ \ \ \ saveOption \textrm{[F0DF?]} enr.option\ \ \ \ }

{\sffamily
\ \ \ \ cpt \textrm{[F0DF?]} 0}

{\sffamily
\textstyleMotCl{\ \ \ \ tant} \textstyleMotCl{que} NON Student.EOF() ET
saveOption = enr.option \textstyleMotCl{faire}\ \ }

{\sffamily
\ \ \ \ \ \ // traitement de l’enregistrement}

{\sffamily
\ \ \ \ \ \ cpt \textrm{[F0DF?]} cpt + 1}

{\sffamily
\ \ \ \ \ \ enr \textrm{[F0DF?]} Student.lire()}

{\sffamily\bfseries
\ \ \ \ fin tant que}

{\sffamily
\textbf{\ \ \ \ écrire} cpt, «~étudiant dans l’option~», saveOption}

{\sffamily
\ \ \textstyleMotCl{fin} \textstyleMotCl{tant} \textstyleMotCl{que} }

{\sffamily
\ \ Student.fermer()}

{\sffamily
\textstyleMotCl{fin} \textstyleMotCl{module}}

{
Questions de réflexion : }

\liststyleListv
\begin{itemize}
\item {
pourquoi la condition \textstyleCodeInsr{NON Student.EOF()}
apparait-elle une 2\textsuperscript{ème} fois dans la boucle intérieure
?}
\item {
pourquoi l’ordre de lecture est-il dans la boucle centrale et pas
ailleurs ?}
\item {
pourquoi dans l’instruction d’écriture, il apparait
\textstyleCodeInsr{saveOption} plutôt que
\textstyleCodeInsr{enr.option} ?}
\item {
l’ordre des conditions apparaissant dans le 2\textsuperscript{ème} tant
que est-il important ?}
\end{itemize}
{
L’algorithme ci-dessus se généralise facilement si on ajoute davantage
de niveaux de rupture. Pour illustrer le «~niveau 2~», prenons encore
l’exemple suivant. On veut connaitre pour chaque groupe le nombre
d’étudiants nés dans les différentes années de naissance. L’algorithme
correspondant s’écrit facilement et fonctionne lorsque le fichier est
cette fois-ci classé en majeur sur le groupe et en mineur sur la date
de naissance~(ou encore classement double sur la clé complexe option –
dateNais) :}

{\sffamily
\textstyleMotCl{module} RuptureNiveau2(Student↓~: FichierEntrée
d’Etudiant)}

{\sffamily
\textstyleMotCl{\textmd{\ \ // on suppose les données classées en majeur
sur l’option}}}

{\sffamily
\textstyleMotCl{\textmd{\ \ // et en mineur sur la date de naissance
(ordre chronologique)}}}

{\sffamily
\ \ enr : Etudiant}

{\sffamily
\ \ saveOption : caractère}

{\sffamily
\ \ saveAnneNaissance : entier}

{\sffamily
\ \ cpt : entier}

{\sffamily
\ \ Student.ouvrir()}

{\sffamily
\ \ enr \textrm{[F0DF?]} Student.lire()}

{\sffamily
\ \ \textstyleMotCl{tant} \textstyleMotCl{que} NON Student.EOF()
\textstyleMotCl{faire}\ \ }

{\sffamily
\ \ \ \ saveOption \textrm{[F0DF?]} enr.option\ \ \ \ }

{\sffamily
\ \ \ \ \textbf{t}\textstyleMotCl{ant} \textstyleMotCl{que} NON
Student.EOF() ET saveOption = enr.option \textstyleMotCl{faire}}

{\sffamily
\ \ \ \ \ \ saveAnneNaissance \textrm{[F0DF?]} enr.dateNais.getAnnée()}

{\sffamily
\textstyleMotCl{\textmd{\ \ \ \ \ \ cpt
}}\textstyleMotCl{\textrm{\textmd{[F0DF?]}}}\textstyleMotCl{\textmd{
0}}\textbf{\ \ }}

{\sffamily
\ \ \ \ \ \ \ \textbf{t}\textstyleMotCl{ant} \textstyleMotCl{que} NON
Student.EOF() ET saveOption = enr.option }

{\sffamily
\textbf{\ \ \ \ \ \ \ \ }ET saveAnneNaissance =
enr.dateNais.getAnnée()\textstyleMotCl{faire}}

{\sffamily
\ \ \ \ \ \ \ \ // traitement de l’enregistrement}

{\sffamily
\ \ \ \ \ \ \ \ cpt \textrm{[F0DF?]} cpt + 1}

{\sffamily
\ \ \ \ \ \ \ \ enr \textrm{[F0DF?]} Student.lire()}

{\sffamily\bfseries
\ \ \ \ \ \ fin tant que}

{\sffamily
\textbf{\ \ \ \ \ \ }ECRIRE cpt, «~étudiant dans l’option~», saveOption,
}

{\sffamily
\ \ \ \ \ \ \ \ \ \ \ \ «~sont nés en~», saveAnneeNaissance\ \ }

{\sffamily\bfseries
\ \ \ \ fin tant que}

{\sffamily
\ \ \textstyleMotCl{fin} \textstyleMotCl{tant} \textstyleMotCl{que} }

{\sffamily
\ \ Student.fermer()}

{\sffamily
\textstyleMotCl{fin} \textstyleMotCl{module}}

{
Ces exemples montrent que l’algorithme de rupture et le classement du
fichier sont étroitement liés. La structure de l’algorithme épouse le
schéma de la clé complexe du classement des données. À un classement
déterminé correspondra un algorithme bien précis, et le choix d’un
classement d’un fichier peut être fait précisément en vue qu’un certain
algorithme donne un résultat bien déterminé.}

\section{Traitements de clôture et d’initialisation}
{
Chaque rupture du traitement itératif des éléments d’un ensemble
entraine un \textbf{traitement de clôture} sur cet ensemble. Comme une
rupture à un niveau implique des ruptures en cascade sur tous les
niveaux d’ordre plus grands, un traitement de clôture d’un ensemble ne
pourra se faire que lorsque le dernier sous-ensemble de cet ensemble
sera clôturé.}

{
De la même manière, l’arrivée d’un élément appartenant à un nouvel
ensemble nécessite un \textbf{traitement d’initialisation} de ce nouvel
ensemble.}

{
En fait, il ne s’agit que de généraliser ce qui se fait lors du
traitement d’un fichier (travaux d’initialisation consistant par
exemple à mettre des totalisateurs ou compteurs à zéro et travaux de
clôture consistant par exemple à imprimer des résultats totaux
particuliers) à tous les ensembles et sous-ensembles composant ce
fichier !}

\section{Généralisation}
{
L’algorithme de rupture vu ci-dessus s’adapte aussi à d’autres
structures que les fichiers. S’il a été présenté avec des fichiers
c’est parce qu’il a été développé à l’origine pour calculer certains
résultats sur des grands ensembles de données sans nécessiter l’usage
de tableaux. }

{
Montrons simplement comment s’adapterait le modèle de rupture de niveau
1 ci-dessus si les données étaient contenues dans un tableau plutôt
qu’un fichier, ce tableau étant bien entendu ordonné de la même façon
que précédemment.}

{\sffamily
\textstyleMotCl{module} RuptureNiveau1dansTableau(Student↓~: tableau[1 à
n] d’Etudiant)}

{\sffamily
\textstyleMotCl{\textmd{\ \ // on suppose les données classées en majeur
sur l’option}}}

{\sffamily
\ \ élément : Etudiant}

{\sffamily
\ \ saveOption : caractère}

{\sffamily
\ \ cpt, i : entier}

{\sffamily
\ \ i \textrm{[F0DF?]} 1}

{\sffamily
\textbf{\ \ tant que} i ${\leq}$ n \textstyleMotCl{faire}\ \ }

{\sffamily
\ \ \ \ saveOption \textrm{[F0DF?]} tab[i].option\ \ \ \ }

{\sffamily
\ \ \ \ cpt \textrm{[F0DF?]} 0}

{\sffamily
\textstyleMotCl{\ \ \ \ tant} \textstyleMotCl{que} i ${\leq}$ n ET
saveOption = tab[i].option \textstyleMotCl{faire}\ \ }

{\sffamily
\ \ \ \ \ \ // traitement de l’élément}

{\sffamily
\ \ \ \ \ \ cpt \textrm{[F0DF?]} cpt + 1}

{\sffamily
\ \ \ \ \ \ {i
}\textrm{[F0DF?]}{ i + 1}}

{\sffamily
{\textbf{\ \ \ \ }}\textbf{fin tant que}}

{\sffamily
\textbf{\ \ \ \ écrire} cpt, «~étudiant dans l’option~», saveOption}

{\sffamily
\ \ \textstyleMotCl{fin} \textbf{tant que} }

{\sffamily
\textstyleMotCl{fin} \textstyleMotCl{module}}

{
Observer comment dans cet exemple se transforme la progression dans
l’ensemble des données, la constatation de rupture à l’issue de chaque
sous-groupe et à la fin de l’ensemble.}

\section{Exercices}

\bigskip

\liststyleExercice
\begin{enumerate}
\item {\sffamily\bfseries
La chasse au gaspi.}
\end{enumerate}
{
À l’ESI, les quantités de feuilles imprimées et photocopiées par les
professeurs et les étudiants sont enregistrées dans un fichier
séquentiel IMPRESSION, créé au début de l’année en cours et mis à jour
tous les soirs. Le service technique désirant facturer les
«~exagérations~», a demandé à pouvoir utiliser ce fichier de manière à
établir des statistiques du nombre de feuilles imprimées par
utilisateur.}

{
Le fichier IMPRESSION présente la structure d’enregistrement
\textbf{Enr} suivante :}

{
login\ \ \ \ chaine\ \ \ \ login de l’utilisateur}

{
date\ \ \ \ \textbf{Date}\ \ \ \ date de la session d’impression }

{
moment\ \ \textbf{Moment}\ \ moment de la session}

{
nombre\ \ entier\ \ \ \ nombre de feuilles imprimées ou photocopiées}

{
De plus, il est ordonné croissant en majeur sur le champ login et en
mineur sur la date. }

\liststyleNumberingv
\begin{enumerate}
\item {
Écrire un algorithme permettant d'écrire une ligne par
utilisateur dont le nombre total de feuilles imprimées dépasse une
valeur limite entrée en paramètre. Cette ligne contiendra le login et
le nombre. }
\item {
Le fichier étant mis à jour tous les soirs, il est plus probable
qu'il soit trié en majeur sur la date et en mineur sur
le login. Décrivez l'allure générale de
l'algorithme dans ce cas : variables nécessaires et
grandes étapes de la solution.}
\end{enumerate}
\liststyleExercice
\begin{enumerate}
\item {\sffamily\bfseries
Statistiques de ventes de voitures.}
\end{enumerate}
{
Un grand quotidien dispose d’un fichier VENTES regroupant les ventes de
voitures neuves pendant l’année dernière. Chaque demande
d’immatriculation a donné naissance à un enregistrement dans ce
fichier. Ce fichier est ordonné croissant en majeur sur la marque de
voiture et en mineur sur le type. Chaque enregistrement de ce fichier
(structure \textbf{Voiture}) comprend les champs suivants, chacun de
type chaine sauf le dernier :}

{
plaque : \ \ numéro d’immatriculation}

{
marque :\ \ marque de la voiture (par ex. «~Citroën~»)}

{
type : \ \ \ \ type de modèle dans la marque (par ex. «~Berlingo~»)}

{
nom : \ \ \ \ nom du propriétaire}

{
adresse : \ \ adresse du propriétaire}

{
date:\ \ \ \ date de la demande d’immatriculation}

{
Afin de préparer le travail des journalistes, il a été demandé au
service informatique de préparer un affichage qui globalise les ventes
de voiture par marque et pour chaque marque, par type. Cet affichage
contiendra les renseignements suivants :}

\liststyleListv
\begin{itemize}
\item {
un titre général «~Ventes de voitures neuves en ****~» où les étoiles
seront remplacées par l’année concernée.}
\item {
pour chaque marque :}
\item {
le nom de la marque}
\item {
pour chaque type de modèle}
\item {
le nom de ce type et le nombre de voitures neuves vendues}
\item {
le nombre total de voitures vendues pour cette marque}
\item {
enfin, le total global du nombre de voitures vendues toutes marques
confondues}
\end{itemize}
{
À la fin, on souhaite également avoir le palmarès des ventes globales
par marque donné en ordre décroissant des ventes (en cas d’égalité de
ventes pour des marques différentes, on respectera l’ordre alphabétique
pour la parution dans ce palmarès). On devra donc écrire d’abord la
marque pour laquelle il y a eu le plus de ventes, et ainsi de suite
jusque la marque pour laquelle il y en a eu le moins. Notez également
qu’une marque pour laquelle il n’y a pas eu la moindre vente,
n’apparaitra pas dans ce palmarès !}

{
Écrivez un algorithme produisant l'affichage décrit.}

\liststyleExercice
\begin{enumerate}
\item {\sffamily\bfseries
Les fanas d'info.}
\end{enumerate}
{
Une grande société d’informatique a organisé durant les douze derniers
mois une multitude de concours ouverts aux membres de clubs
d’informatique. Elle souhaiterait récompenser le club qui aura été le
plus «~méritant~» durant cette période au point de vue de la
participation des mineurs. Chaque résultat individuel des participants
(y compris des majeurs) a été enregistré dans un fichier RESULTATS dont
la structure d’un enregistrement \textbf{Enr} est :}

{
nom: \ \ \ \ nom et prénom du participant}

{
âge: \ \ \ \ âge du participant au moment du concours (entier)}

{
référence: \ \ référence du club auquel appartient ce participant}

{
numéro: \ \ numéro du concours auquel il a participé}

{
résultat: \ \ résultat obtenu lors de ce concours (entier sur 100)}

{
Sachant que ce fichier est ordonné en majeur sur le champ la référence
et en mineur sur le nom, on demande d’écrire l’algorithme qui écrit les
informations suivantes :}

{
pour chaque club :}

\liststyleListv
\begin{itemize}
\item {
sa référence}
\item {
pour chaque membre mineur de ce club :}
\item {
son nom et prénom}
\item {
la cote moyenne sur 100 des concours auquel ce membre a participé}
\item {
le nombre total de participations des membres mineurs}
\end{itemize}
{
\textbf{N.B. : }un membre mineur qui s’est inscrit à un concours = une
participation. Un club qui n’aura eu aucun membre mineur participant
figurera quand même dans le résultat avec la mention «~Pas de
participation de membre mineur~». Par contre, un club dont aucun membre
n’a participé au moindre concours ne sera pas écrit.}

{
À la fin, on affichera la référence du meilleur club, à savoir celui qui
a eu le plus de cotes moyennes des membres mineurs supérieures ou
égales à 80\%. En cas d’égalité, on départagera sur le club qui a eu le
plus grand nombre de participations des mineurs (et on suppose qu’il
n’y a pas d’ex-aequo sur ce point). }

\liststyleExercice
\begin{enumerate}
\item {\sffamily\bfseries
Degré ou de force.}
\end{enumerate}
{
Soit le fichier séquentiel RELEVES reprenant par site où un capteur de
température est installé, les relevés réalisés au cours des douze
derniers mois. Deux relevés sur un même site sont distanciés au minimum
de 30 secondes et au maximum de 10 minutes. Chaque enregistrement du
fichier a la structure \textbf{Relevé} suivante :}

{
\ \ lieu: \ \ \ \ site d’un relevé (chaine)}

{
date:\ \ \ \ \textbf{Date}}

{
moment :\ \ \textbf{Moment}}

{
température :\ \ température relevée (réel)}

{
Ce fichier est ordonné croissant en majeur sur le lieu, en médian sur la
date et en mineur sur le moment. On demande d’écrire l’algorithme
permettant de~créer :}

\liststyleListv
\begin{itemize}
\item {
le fichier QUOTIDIEN qui reprend les températures maximale et minimale
pour chaque site et chaque jour ; la structure d’un enregistrement sera
LIEU – DATE – MAX – MIN}
\item {
le fichier MENSUEL qui reprend par site les températures maximale et
minimale par mois ainsi que le nombre moyen de relevés quotidiens ; la
structure d’un enregistrement sera LIEU – MOIS – MAX – MIN – NOMBRE}
\end{itemize}

\bigskip


\bigskip

\liststyleExercice
\begin{enumerate}
\item {\sffamily\bfseries
NBA actions.}
\end{enumerate}
{
Soit le fichier RESULTATS reprenant l’ensemble des points marqués et
concédés pour chaque équipe inscrite dans le championnat NBA de
basket-ball. Chaque rencontre jouée du championnat donne naissance à
deux enregistrements, un pour chacune des deux équipes disputant ce
match. La structure d’un enregistrement de type \textbf{Résultat} est
:}

{
équipe :\ \ chaine contenant le nom de l’équipe}

{
ptsGagnés : \ \ entier représentant les points marqués par l’équipe}

{
ptsPerdus : \ \ entier représentant les points encaissés par l’équipe}

{
date : \ \ \ \ Date donnant la date de la rencontre.}

{
Le fichier est ordonné croissant en majeur sur l'équipe
et en mineur sur la date. On demande d’écrire l’algorithme qui
permet~d’écrire :}

\liststyleListv
\begin{itemize}
\item {
un titre général : «~STATISTIQUES NBA~»}
\item {
pour chaque équipe : }
\item {
son nom \ }
\item {
le total des points marqués}
\item {
le total des points encaissés}
\item {
le plus grand écart de points lors d’une rencontre de cette équipe}
\item {
en fin de traitement :}
\item {
le nombre total de points marqués pour l’ensemble des équipes.}
\item {
le mois pendant lequel le plus de points ont été marqués}
\item {
le total de points pour le match le plus spectaculaire (plus grand total
de points pour et contre lors d’un match)}
\end{itemize}
{
De plus, cet algorithme enregistrera dans le fichier PLAYOFF le nom des
équipes dont le pourcentage des victoires est au moins 50\% (une
victoire se traduit par un enregistrement dans lequel le nombre de
points marqués est supérieur au nombre de points encaissés, sachant que
le match nul n’existe pas)}

\liststyleExercice
\begin{enumerate}
\item {\sffamily\bfseries
Le meilleur site.}
\end{enumerate}
{
Pendant le mois de novembre dernier, le fournisseur d’accès ESINET a
proposé aux internautes du monde entier d’élire le plus beau site du
\textit{World Wide Web}. Pour ce faire, ESINET a enregistré dans un
fichier séquentiel VOTES les votes émis par e-mail durant ce mois. Le
règlement du concours était simple: on pouvait voter pour plusieurs
sites, mais pas plusieurs fois pour le même site. Si cela arrivait
quand même, les votes émis à partir d’une même adresse e-mail pour un
même site étaient quand même enregistrés dans le fichier VOTES mais ne
pouvaient compter que pour \textbf{un seul} vote valable. Sachant que
ce fichier a la structure d’enregistrement suivante :}

{
URL : \ \ adresse du site choisi (ex. : «~http://www.heb.be~»)}

{
email : adresse e-mail du votant}

{
jour : \ \ jour du vote (de 1 à 30)}

{
et que le fichier est ordonné croissant en majeur sur
l'URL et en mineur sur l'email,
écrire un algorithme qui en une seule lecture du fichier écrit les
informations suivantes :}


\bigskip

\liststyleListv
\begin{itemize}
\item {
les adresses des sites pour lesquels il y a eu au moins un vote, ainsi
que le nombre de votes valables recueillis }
\item {
le nom du site gagnant, c’est-à-dire celui qui a reçu le plus de votes
valables, ainsi que le nombre de ces votes (on peut supposer qu’il n’y
a pas eu d’ex-æquo)}
\item {
la date du jour qui a connu le plus d’activité, c’est-à-dire le plus
grand nombre de votes (valables ou non).}
\end{itemize}
\liststyleExercice
\begin{enumerate}
\item {\sffamily\bfseries
Quoi de neuf, doc ?}
\end{enumerate}
{
Vous n’êtes pas sans savoir que les médicaments, dans la plupart des
cas, doivent faire l’objet d’une prescription médicale pour pouvoir
être délivrés. Sur la prescription de médicaments, appelée également
l’ordonnance, on trouve les médicaments prescrits, le cachet du
prescripteur ou médecin, avec son nom et son numéro d’agrément. }

{
Chaque ordonnance, lorsqu’elle est présentée au pharmacien, est scannée.
L’image ainsi obtenue passe dans un programme de traitement
informatique qui génère autant d’enregistrements dans un fichier qu’il
y a de médicaments différents prescrits. Les ordinateurs des pharmacies
sont reliés à un ordinateur central situé à Bruxelles. Ainsi, les
enregistrements créés au niveau des pharmacies sont directement ajoutés
à un fichier central annuel appelé PRESCRIPTIONS. }

{
La structure \textbf{Prescription} des enregistrements de ce fichier est
la suivante :}

{
date : \ \ \ \ date de la prescription}

{
numéro : \ \ numéro d’agrément du médecin}

{
nom : \ \ \ \ nom du médicament}

{
qté : \ \ \ \ quantité de ce médicament prescrite sur cette ordonnance}

{
En supposant complété le fichier PRESCRIPTIONS pour l’année écoulée, on
demande d’écrire l’algorithme d’une application qui permettra d’après
ce fichier de compter le nombre de médecins différents qui ont prescrit
des médicaments et d’afficher le résultat. }

{
Pour réaliser cela, expliquez brièvement de quelle façon ce fichier
devrait être ordonné.}

\liststyleExercice
\setcounter{saveenum}{\value{enumi}}
\begin{enumerate}
\setcounter{enumi}{\value{saveenum}}
\item {\sffamily\bfseries
Vos papiers, SVP !}
\end{enumerate}
{
La gendarmerie a enregistré dans un fichier PV toutes les infractions au
code de la route commises par des automobilistes ayant une plaque belge
durant l’année dernière. La structure \textbf{Infraction} d’un
enregistrement comprend les champs :}

{
plaque:\ \ \ \ chaine}

{
date :\ \ \ \ \textbf{Date}}

{
moment:\ \ \textbf{Moment}}

{
type:\ \ chaine (code associé à une infraction donnée)}

{
Le fichier est ordonné croissant sur la plaque et ne contient aucune
erreur. Il peut bien entendu y avoir dans le fichier plusieurs
enregistrements ayant le même numéro de plaque.}

{
La gendarmerie dispose également d’une table \textstyleCodeInsr{amendes}
de 60 éléments structurés de type \textbf{Amende} comprenant les champs
:}

{
type:\ \ \ \ comme ci-dessus}

{
libellé:\ \ \ \ chaine décrivant l’infraction (ex. «~excès de vitesse~»,
etc.)}

{
montant:\ \ montant de l’amende en €.}

{
Écrire l’algorithme qui, en une seule lecture du fichier PV, calcule et
écrit dans le fichier STATS, pour chaque plaque apparaissant dans le
fichier PV, une ligne contenant le numéro de plaque du véhicule et le
montant total annuel des amendes pour cette plaque. On écrira également
en fin de traitement le total des montants des amendes pour toutes les
plaques recensées.}

{
Enfin, on écrira un tableau récapitulatif du nombre d’amendes par type
d’amendes. Ce tableau reprendra le libellé et le nombre en commençant
par les plus fréquentes jusqu’au moins fréquentes. }

\liststyleExercice
\setcounter{saveenum}{\value{enumi}}
\begin{enumerate}
\setcounter{enumi}{\value{saveenum}}
\item {\sffamily\bfseries
Bruxelles-national}
\end{enumerate}
{
Votre société d’informatique a signé un contrat avec l’aéroport national
pour l’établissement informatique de statistiques sur le trafic aérien.
Le service informatique de l’aéroport enregistre sans arrêt les
atterrissages et décollages des avions ainsi que d’autres informations.
Pour l’année dernière, toutes les informations sont enregistrées dans
le fichier VOLS.}

{
La structure \textbf{Vol} des enregistrements de ce fichier contient les
champs : }

{
numéro : \ \ chaine donnant le numéro du vol}

{
type  : \ \ \ \ une lettre : ‘A’ pour atterrissage ou ‘D’ pour
décollage}

{
lieu : \ \ \ \ chaine donnant le lieu d’où provient le vol (si le type
vaut ‘A’) }

{
\ \ \ \ ou le lieu de destination du vol (si le type vaut ‘D’)}

{
cie : \ \ \ \ compagnie aérienne propriétaire de l’avion}

{
nb : \ \ \ \ entier donnant le nombre de passagers (pouvant valoir 0)}

{
qté : \ \ \ \ réel donnant le tonnage de marchandise transportée }

{
\ \ \ \ (pouvant \ valoir 0)}

{
date : \ \ \ \ date du vol}

{
moment : \ \ heure de départ ou d’arrivée}

{
Sachant que ce fichier est ordonné croissant en majeur sur la compagnie
et en mineur sur lieu, on vous demande d’établir l’algorithme qui
permette d’établir les statistiques suivantes :}

\liststyleListv
\begin{itemize}
\item {
pour chaque compagnie apparaissant dans VOLS :}
\item {
une ligne titre avec le nom de la compagnie,}
\item {
pour chaque endroit desservi par cette compagnie, une seule ligne avec
son nom, le nombre d’atterrissages d’avions provenant de ce lieu, le
nombre de décollages d’avions vers ce lieu\ \ }
\item {
en fin de chaque compagnie, le nom du lieu d’où provient le plus grand
nombre de passagers transportés par cette compagnie là, ainsi que le
nom du lieu vers lequel la plus grande quantité totale de marchandises
a été expédiée par cette compagnie là.}
\end{itemize}

\bigskip


\bigskip

\liststyleListv
\begin{itemize}
\item {
à la \ fin on écrira : }
\item {
la différence entre le nombre \textbf{total} de passagers arrivés et
partis ;}
\item {
le nombre de compagnies trouvées dans le fichier ;}
\item {
le nom de la compagnie ayant eu le moins de vols ;}
\item {
la tranche horaire où le trafic en nombre de vols est le plus intense
(\textbf{N.B.:} on \ peut utiliser un tableau pour résoudre cette
requête)}
\end{itemize}
{
\textbf{N.B.:} Pour toutes les recherches de maxima ou minima, on les
supposera uniques. La première tranche horaire est 0 et la dernière 23.
}

\liststyleExercice
\begin{enumerate}
\item {\sffamily\bfseries
\textstyleMotCl{Une suite logique}}
\end{enumerate}
{
\textstyleMotCl{\textmd{Voici une petite suite logique :}}}

\liststyleNumberingv
\begin{enumerate}
\item {
\textstyleMotCl{\textmd{Comprenez la logique derrière cette suite et
écrivez la ligne suivante.}}}
\begin{center}
\begin{minipage}{4.748cm}
{
\textstyleMotCl{\textmd{1}}}

{
\textstyleMotCl{\textmd{1 1}}}

{
\textstyleMotCl{\textmd{2 1}}}

{
\textstyleMotCl{\textmd{1 2 1 1}}}

{
\textstyleMotCl{\textmd{1 1 1 2 2 1}}}

{
\textstyleMotCl{\textmd{3 1 2 2 1 1}}}

{
\textstyleMotCl{\textmd{1 3 1 1 2 2 2 1}}}

{
\textstyleMotCl{\textmd{1 1 1 3 2 1 3 2 1 1}}}

{
\textstyleMotCl{\textmd{3 1 1 3 1 2 1 1 1 3 1 2 2 1}}}

{
\textstyleMotCl{\textmd{...}}}
\end{minipage}
\end{center}
\item {
Écrivez un module qui reçoit une ligne (sous forme
d'une Liste d'entiers) et retourne la
ligne suivante (sous forme d'une autre liste
d'entiers). Votre première tâche sera probablement de
comprendre ce que vient faire cet exercice dans le chapitre des
ruptures.}
\item {
\textstylePolicepardfauti{{Écrivez le module qui
reçoit N (un entier) et affiche les N premières lignes de cette suite
logique.}}}
\end{enumerate}
\liststyleExercice
\begin{enumerate}
\item {\sffamily\bfseries
Éliminer les doublons d'une liste.}
\end{enumerate}
{
\textstylePolicepardfauti{Soit une liste
}\textstylePolicepardfauti{\textbf{ordonnée}}\textstylePolicepardfauti{
d'entiers avec des possibles redondances. Écrire un
module qui enlève les redondances de la liste. On vous demande de créer
une }\textstylePolicepardfauti{\textbf{nouvelle
liste}}\textstylePolicepardfauti{ (la liste de départ reste
inchangée)}}

{
\textstylePolicepardfauti{Exemple}\textstylePolicepardfauti{ : Si la
liste est (1, 3, 3, 7, 8, 8, 8) le résultat est (1, 3, 7, 8).}
}
