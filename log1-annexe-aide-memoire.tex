\chapter{Aide-mémoire}

Cet aide-mémoire peut vous accompagner lors d'une
interrogation ou d'un examen. Il vous est permis
d’utiliser ces classes et méthodes sans les développer.
Si vous sentez le besoin d’utiliser un objet ou une méthode qui
n'apparait pas ici, il faudra en écrire explicitement
le contenu et le code.

Pour manipuler les chaines et les caractères

\cadre{
	\begin{liste}
	\item
		\textstyleCodeInsr{long(maChaine)}
				donne la taille (le nombre de caractères) de la chaine \textstyleCodeInsr{maChaine}
	\item
	\textstyleCodeInsr{car(maChaine,pos)}
				donne le caractère en position \textstyleCodeInsr{pos} 
				(à partir de 1) 
				\\dans la chaine \textstyleCodeInsr{maChaine}
	\item
	\textstyleCodeInsr{concat(maChaine1,maChaine2)}
				concatène les chaines \textstyleCodeInsr{maChaine1} 
				et \textstyleCodeInsr{maChaine2}
	\item
	\textstyleCodeInsr{chaine(car)} 
				transforme le caractère \textstyleCodeInsr{car} 
				en une chaine de taille 1
	\item
	\textstyleCodeInsr{estLettre(car)} 
				est vrai si le caractère \textstyleCodeInsr{car} est une lettre
				\\(idem pour \textstyleCodeInsr{estChiffre}, 
				\textstyleCodeInsr{estMajuscule}, 
				\textstyleCodeInsr{estMinuscule})
	\item
	\textstyleCodeInsr{majuscule(car)} 
				donne la majuscule de la lettre \textstyleCodeInsr{car}
				(idem pour \textstyleCodeInsr{minuscule})
	\item
	\textstyleCodeInsr{position(car)} 
				donne la position de \textstyleCodeInsr{car} dans l'alphabet 
	\item
	\textstyleCodeInsr{lettre(num)} 
				l'inverse de la précédente
	\item
	\textstyleCodeInsr{estEspace(car~: caractère)}
			retourne vrai si car est un caractère d’espacement et faux sinon.
	\item
	\textstyleCodeInsr{estPonctuation(car~: caractère)}
			retourne vrai si car est un caractère de ponctuation et faux sinon.
	\end{liste}
	}


\cadre{
\begin{pseudo}
\Class{Date}
\Public
	\ConstrSign{Date}{} \RComment Crée la date du jour
	\ConstrSign{Date}{j, m, a : entiers} 
	\MethodSign{getJour}{}{entier}
	\MethodSign{getMois}{}{entier}
	\MethodSign{getAnnée}{}{entier}
	\MethodSign{égale}{autreDate : Date}{booléen}
	\MethodSign{estAntérieure}{autreDate : Date}{booléen}
\EndClass
\end{pseudo}
}

\cadre{
\begin{pseudo}
\Class{Moment}
\Public
	\ConstrSign{Moment}{} \RComment Crée le moment courant
	\ConstrSign{Moment}{h, m, s : entiers} 
	\MethodSign{getHeure}{}{entier}
	\MethodSign{getMinute}{}{entier}
	\MethodSign{getSeconde}{}{entier}
	\MethodSign{setHeure}{h : entier}{}
	\MethodSign{setMinute}{m : entier}{}
	\MethodSign{setSeconde}{s : entier}{}
	\MethodSign{égal}{autreMoment : Moment}{booléen}
	\MethodSign{estAntérieur}{autreMoment : Moment}{booléen}
\EndClass
\end{pseudo}
}

\cadre{
\begin{pseudo}
\Class{Durée}
\Public
	\ConstrSign{Durée}{secondes : entier} 
	\ConstrSign{Durée}{h, m, s : entiers} 
	\MethodSign{getJour}{}{entier}
	\MethodSign{getHeure}{}{entier}
	\MethodSign{getMinute}{}{entier}
	\MethodSign{getSeconde}{}{entier}
	\MethodSign{getTotalHeure}{}{entier}
	\MethodSign{getTotalMinute}{}{entier}
	\MethodSign{getTotalSeconde}{}{entier}
	\MethodSign{ajouter}{autreDurée : Durée}{}
	\MethodSign{soustraire}{autreDurée : Durée}{}
	\MethodSign{égale}{autreDurée : Durée}{booléen}
	\MethodSign{plusPetit}{autreDurée : Durée}{booléen}
\EndClass
\end{pseudo}
}

\cadre{
\begin{pseudo}
\Class{Liste <T>}
\Public
	\ConstrSign{Liste <T>}{} 
	\MethodSign{get}{pos : entier}{T} \RComment retourne l’élément en position pos
	\MethodSign{set}{pos : entier, valeur : T}{} \RComment modifie l’élément en position pos
	\MethodSign{taille}{}{entier} \RComment retourne la taille de la liste
	\MethodSign{ajouter}{valeur : T}{} \RComment ajoute une valeur en fin de liste
	\MethodSign{insérer}{pos : entier, valeur : T}{} \RComment insère un élément en position pos
	\MethodSign{supprimer}{}{} \RComment supprime le dernier élément
	\MethodSign{supprimerPos}{pos : entier}{} \RComment supprime l’élément en position pos
	\MethodSign{supprimer}{valeur : T}{} \RComment supprime l’élément de valeur donnée
	\MethodSign{vider}{}{} \RComment vide la liste
	\MethodSign{estVide}{}{booléen} \RComment indique si la liste est vide
	\MethodSign{existe}{valeur\In : T, pos\Out : entier}{booléen} \RComment recherche un élément
\EndClass
\end{pseudo}
}

% Une méthode égale pour tous.
% ajouter valeurAbsolue(entier)
