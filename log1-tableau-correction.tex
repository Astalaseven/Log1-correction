\chapter*{Correction des exercices 8.7}

\begin{Emphase}{Exercice 1 -- Somme}

\begin{pseudoN}

\Module{somme}{tabEnt~:~\K{tableau} [1 à n] d’entiers}{entier}

    \Decl i, somme~:~entiers
    \Let somme \Gets 0

    \For{i \From 1 \To n}
        
        \Let somme \Gets somme + tabEnt[i]

    \EndFor

    \Return somme

\EndModule

\end{pseudo}

\end{Emphase}



\begin{Emphase}{Exercice 2 -- Maximum/minimum}

\begin{pseudoN}

    \Module{max}{tabEnt~:~\K{tableau} [1 à n] d’entiers}{entier}

    \Decl max~:~entier
    \Let max \Gets tabEnt[1]

    \For{i \From 2 \To n}
        
        \If{tabEnt[i] > max}

            \Let max \Gets tabEnt[i]

        \EndIf

    \EndFor

    \Return max

\EndModule

\Empty

\Module{min}{tabEnt~:~\K{tableau} [1 à n] d’entiers}{entier}

    \Decl min~:~entier
    \Let min \Gets tabEnt[1]

    \For{i \From 2 \To n}
        
        \If{tabEnt[i] < min}

            \Let min \Gets tabEnt[i]

        \EndIf

    \EndFor

    \Return min

\EndModule

\end{pseudo}

\end{Emphase}


\begin{Emphase}{Exercice 3 -- Indice maximum/minimum}

\begin{pseudoN}

    \Module{indiceMax}{tabEnt~:~\K{tableau} [1 à n] d’entiers}{entier}

    \Decl indiceMax~:~entier
    \Let indiceMax \Gets 1

    \For{i \From 2 \To n}
        
        \If{tabEnt[i] > tabEnt[indiceMax]}

            \Let indiceMax \Gets i

        \EndIf

    \EndFor

    \Return indiceMax

\EndModule

\Empty

\Module{indiceMin}{tabEnt~:~\K{tableau} [1 à n] d’entiers}{entier}

    \Decl indiceMin~:~entier
    \Let indiceMin \Gets 1

    \For{i \From 2 \To n}
        
        \If{tabEnt[i] < tabEnt[indiceMin]}

            \Let indiceMin \Gets i

        \EndIf

    \EndFor

    \Return indiceMin

\EndModule

\Empty

\Module{max}{tabEnt~:~\K{tableau} [1 à n] d’entiers}{entier}

    \Return tabEnt[indiceMax(tabEnt)]

\EndModule

\Empty

\Module{min}{tabEnt~:~\K{tableau} [1 à n] d’entiers}{entier}

    \Return tabEnt[indicemin(tabEnt)]

\EndModule

\end{pseudo}

\end{Emphase}


\begin{Emphase}{Exercice 4 -- Nombre d’éléments d’un tableau}

\begin{pseudoN}

\Module{tailleTableau}{tabRéels~:~tableau [1 à n] de réels}{entier}

    \Return n

\EndModule

\end{pseudo}

\end{Emphase}

\begin{Emphase}{Exercice 5 -- Y a-t-il un pilote dans l’avion~?}

\begin{pseudoN}

\Module{piloteAvion}{avion~:~tableau [1 à n] de chaines}{booléen}

    \Decl i~:~entier
    \Decl trouvé~:~booléen

    \Let i \Gets 1
    \Let trouvé \Gets faux

    \While{i $\le$ n ET NON trouvé}

        \Let trouvé \Gets avion[i] = "pilote"
        \Let i \Gets i +1

    \EndWhile

    \Return trouvé

\EndModule

\end{pseudo}

\end{Emphase}


\begin{Emphase}{Exercice 6 -- Plus grand écart absolu}

\begin{pseudoN}

\Module{plusGrandÉcartAbsolu}{températures~:~tableau [1 à n] de réels}{reél}

    \Decl i, écart, écartMax, max, min~:~réels
    \Let écartMax \Gets 0

    \For{i \From 2 \To n}

        \Let max \Gets max(températures[i-1], températures[i]) 
        \Let min \Gets min(températures[i-1], températures[i]) 

        \Let écart \Gets max - min
        
        \If{écart > écartMax}
            
            \Let écartMax \Gets écart

        \EndIf

    \EndFor

    \Return écartMax

    \EndModule

\end{pseudo}

\end{Emphase}


\begin{Emphase}{Exercice 7 -- Remplacement de valeurs}

\begin{pseudoN}

    \Module{prénomEnMajuscule}{prénoms\In\Out~:~\K{tableau} [1 à n] de chaines}{}

    \Decl i, j~:~caractère
    \Decl prénom~:~chaine

    \Let prénom \Gets ""

    \For{i \From 1 \To n}
        
        \If{estMinuscule(car(prénoms[i], 1))}

            \Let prénom \Gets majuscule(prénoms[i], 1)

            \For{j \From 2 \To n}

                \Let prénom \Gets concat(prénom, car(prénoms[i], j))

            \EndFor

            \Let prénoms[i] \Gets prénom

        \EndIf

    \EndFor

\EndModule

\end{pseudo}

\end{Emphase}

\begin{Emphase}{Exercice 8 -- Tableau ordonné}

\begin{pseudoN}

\Module{estTrié}{valeurs~:~\K{tableau} [1 à n] de chaines}{booléen}

    \Decl i~:~entier
    \Decl trié~:~booléen

    \Let i \Gets 2
    \Let trié \Gets vrai

    \While{i $\le$ n ET trié}

        \Let trié \Gets valeurs[i - 1] < valeurs[i]

    \EndWhile

    \Return trié

\EndModule

\end{pseudo}

\end{Emphase}

\newpage

\begin{Emphase}{Exercice 9 -- Position des maxima}

\begin{pseudoN}

\Module{posMaxima}{cotes~:~\K{tableau} [1 à n] d’entiers}{}

    \Decl i, indiceMax~:~entiers
    \Let indiceMax \Gets indiceMax(cotes)

    \For{i \From 1 \To n}

        \If{tab[i] = tab[indiceMax]}

            \Write tab[i]

        \EndIf

    \EndFor

\EndModule

\end{pseudo}

\end{Emphase}


\begin{Emphase}{Exercice 10 -- Renverser un tableau}

\begin{pseudoN}

    \Module{inverserTab}{tabCar~:~\K{tableau} [1 à n] de
    caractères}{}

    \Decl tabInversé~:~\K{tableau} [1 à n] de caractères
    \Decl i~:~caractère

    \For{i \From 1 \To n DIV 2}

    \Let tabInversé[i] \Gets tabCar[n - i + 1]

    \EndFor

    \Let tabCar \Gets tabInversé

\EndModule

\end{pseudo}

\end{Emphase}

\begin{Emphase}{Exercice 11 -- Tableau symétrique}

\begin{pseudoN}

    \Module{symétrique}{tabCar~:~\K{tableau} [1 à n] de
caractères}{booléen}

    \Decl i~:~entier
    \Decl symétrique~:~booléen

    \Let i \Gets 1

    \Let symétrique \Gets vrai

    \While{i $\le$ n ET symétrique}

        \Let symétrique \Gets tabCar[i] = tabCar[n -1 + 1]
        \Let i \Gets i + 1

    \EndWhile

    \EndModule

\end{pseudo}

\end{Emphase}


\begin{Emphase}{Exercice 12 -- Cumul des ventes}

\begin{pseudoN}

    \Module{cumulVente}{ventes~:~\K{tableau} [1 à 12] d’entiers}{\K{tableau} [1 à 12] d’entiers}

    \Decl cumul~:~\K{tableau} [1 à 12] d’entiers
    \Decl i~:~entier

    \Let cumul \Gets ventes

    \For{i \From 2 \To 12}

        \Let cumul[i] \Gets cumul[i] + cumul[i - 1]

    \EndFor

    \Return cumul

    \EndModule

\end{pseudo}

\end{Emphase}


\begin{Emphase}{Exercice 13 -- Occurrence des chiffres}

\begin{pseudoN}

\Module{occurrence}{nb~:~entier}{}

    \Decl occurrences~:~\K{tableau} [0 à 9] d’entiers

    \Let initialiser(occurrences)
    \Let compterOccurrence(occurrences, nb)
    \Let afficher(occurrences)

\EndModule

    \Empty

\Module{initialiser}{occurrences\InOut~:~\K{tableau} [0 à 9] d’entiers}{}

    \For{i \From 1 \To 9}

        \Let occurrences[i] \Gets 0

    \EndFor

\EndModule

    \Empty

\Module{compterOccurence}{occurrences\InOut~:~\K{tableau} [0 à 9] d’entiers, nb\In~:~entier}{}

    \Decl chiffre~:~entier

    \While{nb > 0}

        \Let chiffre \Gets nb MOD 10
        \Let nb \Gets nb DIV 10
        \Let occurrences[chiffre] \Gets occurrences[chiffre] + 1

    \EndWhile

\EndModule

    \Empty

\Module{afficher}{occurrences~:~\K{tableau} [0 à 9] d’entiers}{}

    \For{i \From 0 \To 9}
        
        \If{occurrences[i] > 0}

            \Write i , "apparait" , occurrences[i] , "fois"

        \EndIf

    \EndFor

\EndModule

\end{pseudoN}

\end{Emphase}


\begin{Emphase}{Exercice 14 -- Palindrome}

\begin{pseudoN}

\Module{palindrome}{phrase~:~\K{tableau} [1 à n] de caractères}{booléen}

    \Decl i~:~caractère
    \Decl chaine, chaineNormalisée~:~chaines
    \Decl tabNormalisé~:~\K{tableau} [1 à n] de caractères

    \Let chaine \Gets tableauVersChaine(phrase)

    \Let chaineNormalisée \Gets versAlpha(chaine)

    \For{i \From 1 \To long(chaineNormalisée)}

        \Let tabNormalisé[i] \Gets car(chaineNormalisée, i)
    \EndFor

    \Return symétrique(tabNormalisé)

\EndModule

    \Empty

\Module{tableauVersChaine}{tab~:~\K{tableau} [1 à n] de caractères}{chaine}

    \Decl chaine~:~chaine
    \Let chaine \Gets ""

    \For{i \From 1 \To n}

        \Let chaine \Gets chaine + i

    \EndFor

    \Return chaine

\EndModule


\end{pseudoN}

\end{Emphase}


\begin{Emphase}{Exercice 15 -- Moyenne d’éléments}

\begin{pseudoN}

\Module{moyenne}{tabEnt~:~\K{tableau} [1 à n] d’entiers}{}

    \Decl i, moyenne, somme, max~:~entiers

    \Let max \Gets max2(indiceMax(tabEnt), indiceMin(tabEnt))
    \Let i \Gets min2(indiceMax(tabEnt), indiceMin(tabEnt))

    \While{i <= max}

        \Let somme \Gets somme + tabEnt[i]

        \Let i \Gets i + 1

    \EndWhile

    \Let moyenne \Gets $\displaystyle \dfrac{\text{somme}}{\text{max - i}}$

    \Write moyenne


\EndModule

\end{pseudoN}

\end{Emphase}


\begin{Emphase}{Exercice 16 -- OXO}

\begin{pseudoN}

\Module{OXO}{oxo~:~\K{tableau} [1 à n] d’entiers}{}

    \Decl i~:~entier

    \Let i \Gets 1

    \While{i $\le$ n - 2}

        \If{oxo[i] = 'O' ET oxo[i + 1] = 'X' ET oxo[i + 2] = 'O'}
            \Let compteur \Gets compteur + 1
            \Let i \Gets i + 2

        \EndIf

        \Let i \Gets i + 1

    \EndWhile

    \Write compteur

\EndModule

\end{pseudoN}

\end{Emphase}


\begin{Emphase}{Exercice 17 -- Les doublons}

\begin{pseudoN}

\Module{doublon}{tabEnt~:~\K{tableau} [1 à n] d’entiers}{booléen}

    \Decl i, j~:~entiers

    \For{i \From 1 \To n}

        \For{j \From 1 \To n}

            \If{tabEnt[i] = tabEnt[j]}
                
                \Return vrai

            \EndIf

        \EndFor

    \EndFor

    \Return faux

\EndModule

\end{pseudoN}

\end{Emphase}

\newpage

\begin{Emphase}{Exercice 18 -- Mastermind}

\begin{pseudoN}

\Module{testerProposition}{proposition\In, solution\In~:~\K{tableau} [1 à n] de Couleur, bienPlacés\Out, malPlacés\Out~:~entiers}{}

    \Decl corrects~:~\K{tableau} [1 à n] de booléens
    \Decl i, j~:~entiers
    \Let initialiser(corrects, faux)
    \Let bienPlacés \Gets 0
    \Let malPlacés \Gets 0

    \For{i \From 1 \To n}

        \If{proposition[i] = solution[i]}

            \Let corrects[i] \Gets vrai
            \Let bienPlacés \Gets bienPlacés + 1

        \Else

            \Let j \Gets 1

            \While{j $\le$ n ET (corrects[i] OU proposition[i] $\ne$ solution[i])}

                \Let j \Gets j + 1

            \EndWhile

            \If{j $\le$ n}

                \Let corrects[j] \Gets vrai
                \Let malPlacés \Gets malPlacés + 1

            \EndIf


        \EndIf

    \EndFor

\EndModule

    \Empty

\Module{initialiser}{tabBool\InOut~:~\K{tableau} [1 à n] de booléens, bool\In~:~booléen}{}

    \Decl i~:~entier

    \For{i \From 1 \To n}

        \Let tabBool[i] \Gets faux

    \EndFor

\EndModule

\end{pseudoN}

\end{Emphase}


\begin{Emphase}{Exercice 19 -- Casser le chiffre de César}

\begin{pseudoN}


    \Module{initialiser}{compteur\InOut~:~\K{tableau} [1 à 26] d’entiers}{}

        \For{i \From 1 \To 26}

            \Let compteur[i] \Gets 0

        \EndFor

    \EndModule

   \Empty

   \Module{casserCésar}{msgChiffré~:~chaine}{}

        \Decl compteur~:~\K{tableau} [1 à 26] d’entiers
        \Decl lettre, décalage~:~entier
        \Decl initialiser(compteur)
        \For{i \From 1 \To long(msgChiffré)}

            \Let lettre \Gets position(car(msgChiffré, i))
            \Let compteur[lettre] \Gets compteur[lettre] + 1 

        \EndFor

        \Let décalage \Gets max(compteur) - 5

        \Write décalage

    \EndModule

\end{pseudoN}
\end{Emphase}
    


