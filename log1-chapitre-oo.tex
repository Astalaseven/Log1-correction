\chapter[L'orienté objet]{\textstyleMotCl{
L'orienté objet}}
\marginicon{objectif}
{
Dans ce chapitre, nous présentons les bases de la programmation orientée
objet. Nous commençons par expliquer les motivations qui ont amené ce
type de programmation avant d'entrer dans le vif du
sujet en explicitant le concept
d'\textit{encapsulation}. Les autres piliers de
l'orienté objet (\textit{héritage} et
\textit{polymorphisme}) ne seront pas vus cette année.}

\section{Motivation}
{
Depuis son apparition, la puissance de l'ordinateur
n'a cessé de croitre exponentiellement. Les tâches qui
lui sont confiées ont fait de même. Ainsi les programmes à écrire sont
de plus en plus gros et de plus en plus complexes.}

{
Face à la complexité, la démarche est toujours la même : découper le
problème en sous-problèmes (qui peuvent à leur tour être découpés) ce
qui permet}

\liststyleListv
\begin{itemize}
\item {
d'attaquer chaque problème séparément en évitant la
surcharge cognitive ;}
\item {
de répartir le travail entre plusieurs personnes ;}
\item {
de pouvoir réutiliser du travail déjà produit si un sous-problème est
déjà apparu dans le cadre d'un autre problème ;}
\item {
de produire un code plus lisible car s'exprimant avec
des termes de plus haut niveau, plus proches du problème à résoudre.
Ainsi, là où un tri devra être fait, on trouvera le mot «~trier~» qui
fera référence à la partie de code qui s’occupe du tri. Cela va dans le
sens d’une plus grande «~abstraction~» du code : un code qui s’éloigne
du langage simpliste compris par le processeur pour s’approcher de la
pensée humaine et des termes du problème à résoudre.}
\end{itemize}
{
Les langages de programmation ont suivi cette approche en permettant
toujours plus d’abstraction. Dans un chapitre précédent, on vous a
présenté la notion de module qui permet de découper la tâche à réaliser
en sous-tâches ainsi que la notion de structure qui permet de regrouper
des données. Il s'agit là de deux approches
dissociées. }

{
C'est cette lacune que se propose de combler
l'orienté objet : permettre de définir des
\textbf{objets} (composés de \textbf{données} et
\textbf{d'instructions}) qui sont proches du problème
à résoudre. Cela va permettre une meilleure lisibilité et une plus
grande concision du code. Ainsi on pourra définir les notions de date,
d'employé, de fournisseur, de plateau de jeu, de pion,
de livre, d'emprunteur, de carte à jouer, de chambre,
de réservation, de vol, de produit, de stock, de ristourne, de facture,
de panier d'achats, de compte en banque, de banque, de
client, de portefeuille d'actions, ...}

\section{La notion d'objet}
\marginicon{definition}
{\sffamily\bfseries\upshape
Définition\footnote{Les définitions sont tirées du livre de Cardon et
Dabancourt (cf. bibliographie)}}

{
Un \textbf{objet} est une entité logicielle qui :}

\liststyleListi
\begin{itemize}
\item {
a une \textbf{identité~}; c'est-à-dire que nous pouvons
identifier un objet par un nom (tout comme une variable possède un
nom).}
\item {
est capable de sauvegarder un \textbf{état}, c'est à
dire un ensemble d'informations dans des variables
internes;}
\item {
répond à des \textbf{messages} précis en déclenchant des activations
internes appropriées qui peuvent changer l'état de
l'objet. Ces opérations sont appelées des
\textbf{méthodes}. Ce sont des fonctions liées à des objets et qui
précisent le \textbf{comportement} de ces objets.}
\end{itemize}
{\sffamily\bfseries\upshape
État}

{
Un objet contient de l'information, des données qui
définissent son état. }

{\bfseries
Exemples}

\liststyleListv
\begin{itemize}
\item {
Pour un produit, l'état peut être :
l'intitulé du produit, son code barre, son prix, ... }
\item {
Pour un employé, on peut avoir : son nom, son prénom, son adresse, sa
date d'embauche, son salaire mensuel, sa fonction, son
téléphone, ... }
\item {
Une carte à jouer a une couleur et une valeur.}
\item {
L'état d'une date est le jour du
calendrier qu'elle représente.}
\item {
L'état d'une heure est le moment de la
journée qu'elle représente.}
\end{itemize}
{
L'état d'un objet est mémorisé via des
variables qu'on appelle des \textit{attributs}.}

\bigskip

\marginicon{definition}
{\sffamily\bfseries\upshape
Attributs}

{
Les \textbf{attributs} d'un objet sont
l'ensemble des informations se présentant sous forme
de variables et permettant de représenter l'état
d'un objet.}

{
Nous verrons plus loin la syntaxe précise pour définir les attributs
d'un objet.}

{
\textbf{Exemples} }

\liststyleListv
\begin{itemize}
\item {
L'intitulé d'un produit peut être
représenté par une chaine. C'est également le cas des
nom et prénom(s) d'un employé.}
\item {
La date d'embauche peut être représentée par un «~objet
date~» (une date est rarement un type primitif du langage utilisé). Un
attribut d'un objet peut être lui même un objet.}
\item {
Un moment de la journée peut aussi être un objet représenté par trois
entiers\footnote{Toutefois, on verra que ce n'est
peut-être pas la meilleure solution.} : les heures, les minutes et les
secondes (en supposant qu'on désire une précision de
l'ordre de la seconde).}
\item {
L'adresse d'un employé peut être
représentée par une seule chaine mais également par un «~objet
adresse~» (qui contiendrait : une rue, un numéro, un code postal,
...).}
\end{itemize}

\marginicon{attention}
{
\textbf{Remarque}\textbf{ : }Certaines parties de
l'état peuvent évoluer au fil du temps.
D'autres parties sont immuables. Ainsi
l'adresse d'une personne peut changer
mais pas sa date de naissance. }

\marginicon{exercice}
{\sffamily\bfseries\scshape
Exercices - attributs}

\liststyleWWviiiNumi
\begin{enumerate}
\item {
Quel(s) attribut(s) prendriez-vous pour représenter
(l'état d') une date ?}
\item {
Et pour un étudiant ?}
\item {
Et pour un dé à 6 faces ?}
\item {
Et pour une télévision ? \ (on peut en trouver vraiment beaucoup !)}
\end{enumerate}

\bigskip

\marginicon{definition}
{\sffamily\bfseries\upshape
Comportement}

{
Le \textbf{comportement} d'un objet est défini par
l'ensemble des messages ou requêtes auxquels il peut
répondre.}

{
Pour ce faire, il exécute une fonction (un module) qui pourra
éventuellement retourner une information à l'émetteur
du message.}

{
Les messages peuvent interroger l'objet, le modifier,
lui demander d'agir sur son environnement (afficher du
texte, modifier un fichier, ...). }

{
\textbf{Exemples} }

\liststyleListv
\begin{itemize}
\item {
Quels «~messages~» peut-on envoyer à une date ? On peut lui demander
(entre autres):}
	\begin{itemize}
	\item {
	des informations sur le jour du mois, le mois, l'année,
	le jour de la semaine ; }
	\item {
	si elle est antérieure ou non à une autre date ;}
	\item {
	si elle fait partie d'une année bissextile ; }
	\item {
	le nombre de jours qui la sépare de la fin de l'année
	;}
	\item {
	de passer au jour suivant, à la semaine suivante, ...}
	\end{itemize}
\item {
Et pour un stock de produits ? On peut }
	\begin{itemize}
	\item {
	lui demander la quantité disponible d'un produit donné
	;}
	\item {
	lui annoncer l'arrivée d'une quantité
	donnée d'un produit donné ;}
	\item {
	lui indiquer qu'un produit n'existe
	plus (à retirer du stock) ;}
	\item {
	lui demander d'enlever une certaine quantité
	d'un produit du stock.}
	\end{itemize}
\item {
Et pour un employé ? On peut}
	\begin{itemize}
	\item {
	lui demander son adresse, son salaire ou sa fonction, ...}
	\item {
	augmenter son salaire ;}
	\item {
	le changer de fonction ;}
	\item {
	le licencier (penser à prévoir une date de départ dans
	l'état !).}
	\end{itemize}
\item {
Pour un moment de la journée on peut demander s'il se
situe le matin ou pas, ...}	
\end{itemize}

\marginicon{exercice}
{\sffamily\bfseries\scshape
Exercice - comportement}
{\begin{enumerate}
	\item {
	Quel comportement voyez-vous pour un téléviseur ?}
	\item {
	Et pour un étudiant ?}
\end{enumerate}}


\bigskip

\marginicon{definition}
{\sffamily\bfseries\upshape
Méthode}

{
Un message lance l'exécution d'un
module appelé \textbf{méthode} dans le jargon de
l'orienté objet. }

{\bfseries
Exemples}

\liststyleListv
\begin{itemize}
	\item {
	Pour permettre à une date de passer au jour suivant, nous allons définir
	un méthode qui incrémente le jour du mois en tenant compte
	d'un possible basculement au mois suivant ou à
	l'année suivante.}
	\item {
	Pour permettre à un moment d'indiquer
	s'il est le matin ou pas, nous allons définir une
	méthode comme celle-ci \ (nous verrons plus tard comment
	l'associer aux objets)}

	\bigskip

	\cadre{
	\begin{pseudo}
		\LComment On suppose que 'heure' est un des attributs utilisés
		\LComment pour représenter l'état (le moment dans la journée)
		\Method{estMatin}{}{booléen}
			\Return heure > 12 // on considère
			que midi est situé l'après-midi
		\EndMethod
	\end{pseudo}
	}
	\\
	\bigskip
	{
	Cet exemple devrait vous sembler familier à deux exceptions près}

	\liststyleListv
	\begin{itemize}
		\item {
		on utilise le mot «~\textstyleMotCl{méthode}~» en lieu et place de
		«~\textstyleMotCl{module}~» ;}
		\item {
		les attributs (l'heure ici) ne sont pas passés en
		paramètre. Un objet connait déjà son état et donc la valeur de ses
		attributs. Nous verrons plus loin la syntaxe précise.}
	\end{itemize}
\end{itemize}

\marginicon{exercice}
{\sffamily\bfseries\scshape
Exercices - méthodes}

\liststyleWWviiiNumi
\begin{enumerate}
\item {
Dans le comportement d'un téléviseur, on retrouve
«~éteindre~» et «~allumer~». À quoi ressemblerait le code de ces
méthodes ?}
\item {
Écrivez la méthode qui permet de passer au jour suivant.}
\end{enumerate}

\bigskip

{\sffamily\bfseries
Activer un comportement}

{
Pour activer un comportement d'un objet, il faut lui
envoyer un message (ou dit autrement, appeler une de ses méthodes). La
syntaxe que nous allons utiliser (c'est la plus
courante) est la notation pointée}

\cadre{
\begin{pseudo}
	\Stmt nomObjet.nomMéthode()
\end{pseudo}
}

{\bfseries
Exemple}

{
Supposons que le nom «~maintenant~» désigne un objet contenant un moment
de la journée (on verra comment réaliser cela). Si on veut savoir si on
est le matin, on peut écrire}

\cadre{
\begin{pseudo}
	\If{maintenant.estMatin()}
		\Stmt ...
	\EndIf
\end{pseudo}
}


\bigskip

\marginicon{exercice}
{\sffamily\bfseries\scshape
Exercice – activer un comportement}
{
Écrire la portion de code qui allume une télévision (désignée par
«~maTélévision~») et puis l'éteint aussitôt après.}


\bigskip

{\sffamily\bfseries\upshape
Les paramètres d'un comportement}

{
Activer un comportement revient à appeler une méthode de
l'objet. Souvent il est nécessaire
d'envoyer à l'objet des informations
complémentaires pour préciser notre demande ce qui se fait via
l'utilisation des paramètres.}

{\bfseries
Exemple}

{
Si on veut modifier le salaire d'un employé, il faut
que notre message contienne le nouveau salaire. Autrement dit, il faut
communiquer ce nouveau salaire à la méthode de changement du salaire.
Ce qui donne la méthode suivante :}

\cadre{
\begin{pseudo}
	\Method{modifierSalaire}{nouveauSalaire : entier}{}
		\Stmt salaire \Gets nouveauSalaire
	\EndMethod
\end{pseudo}
}

\bigskip

\marginicon{exercice}
{\sffamily\bfseries\scshape
Exercices – paramètres du comportement}

\liststyleWWviiiNumi
\begin{enumerate}
	\item {
	Prenons un objet représentant un étudiant. 
	Nous supposerons qu'un étudiant a un \textit{numéro}, un 
	\textit{nom}, une \textit{annéeÉtude}, est ou non 
	\textit{doubleur} et est ou non \textit{ancien}
	Donnez les entêtes des méthodes suivantes qui permettent de
	:}

	\liststyleListv
	\begin{itemize}
		\item {
		demander le numéro d'un étudiant}
		\item {
		demander si un étudiant est doubleur ou non}
		\item {
		modifier l'année détude d'un étudiant}
		\item {
		faire réussir un étudiant}
		\item {
		faire rater un étudiant.}
	\end{itemize}
		
	\item {
	Prenons un objet représentant une date du calendrier grégorien. 
	Donnez les entêtes des méthodes suivantes qui permettent de
	:}

	\liststyleListv
	\begin{itemize}
		\item{
		demander le jour de la semaine d'une date}
		\item {
		savoir si une date est antérieure ou non à une autre}
		\item {
		connaitre le nombre de jours (absolu) séparant deux dates.}
	\end{itemize}
	
	\item {
	Utilisation. Soit «~date1~» et «~date2~», deux dates ; écrivez la
	portion de code qui utilise les méthodes ci-dessus pour}
	
	\liststyleListv
	\begin{itemize}
		\item {
		vérifier quelle date précède l'autre;}
		\item {
		calculer le nombre de jours d'écart entre ces deux
		dates.}
	\end{itemize}
	
	\item {
	Précédemment, vous avez défini l'ensemble du
	comportement d'un téléviseur. Écrivez les entêtes des
	méthodes correspondant à ce comportement ainsi qu'une
	portion de code qui les utilise.}
\end{enumerate}

\bigskip

\section{L'encapsulation}
{
Un objet possède un état qui est représenté par des attributs. Les
bonnes pratiques de la programmation orientée objet préconisent
fortement que les attributs d'un objet soient
invisibles en dehors de l'objet. Ils ne pourront être
accédés qu'au travers du comportement de
l'objet, c'est-à-dire via ses
méthodes.}

\marginicon{definition}
{\sffamily\bfseries\upshape
{
Lorsque les détails de l'implémentation
d'un objet sont masqués aux autres objets, on dit
qu'il y a \textbf{encapsulation} des données et du
comportement des objets.}}

{
Pourquoi une telle recommandation ? Le but est de garantir la cohérence
de l'état de l'objet. Si on pouvait
accéder directement à un attribut (et donc le modifier), on pourrait y
mettre une valeur incohérente. Par exemple, on pourrait dire que les
minutes d'un moment valent -3 ou 75 ou encore que le
jour de la date est 32 !}

{
Dès lors, il nous faudra préciser pour chaque \textbf{membre} (attributs
et méthodes) d'un objet s'il est
\textbf{privé} (inconnu de l'extérieur) ou
\textbf{public} (connu de l'extérieur). }

{
Le bon usage impose que tous les attributs soient rendus privés et que
les méthodes restent publiques. Toutefois, on pourra trouver également
des méthodes privées. Ce sera notamment le cas si plusieurs méthodes
d'un objet ont une partie commune; \ il sera
intéressant de la \textit{factoriser}, c-à-d en faire une méthode
privée (ex : un calcul de maximum).}

{
Puisqu'un attribut est privé, il est courant pour
chacun des attributs de rencontrer une méthode destinée à connaitre la
valeur de cet attribut et une autre qui permet de la modifier.}

{\sffamily\bfseries\upshape
Accesseur et mutateur}

\marginicon{definition}
{\sffamily\bfseries\upshape
{
\textbf{Accesseur}\footnote{\textstyleFootnoteSymbol{On utilise aussi
souvent le mot anglais «~getter~». }} : méthode dont le but est de
fournir la valeur d'un attribut
\textbf{Mutateur}\footnote{\textstyleFootnoteSymbol{On utilise aussi
souvent le mot anglais «~setter~»}} : méthode dont le but est de
modifier la valeur d'un attribut}}

{
Par convention, ces méthodes sont nommées «~getNom~» et «~setNom~» où
«~nom~» est le nom de
l'attribut\footnote{\textstyleFootnoteSymbol{Pour un
attribut booléen, on pourra préférer «~estNom~» au lieu de
«~getNom~»}}. Par facilité, on utilisera parfois le terme «~accesseur~»
pour désigner à la fois les «~accesseurs~» et les «~mutateurs~».}

{\bfseries
Exemple}

{
Écrivons l'accesseur et le mutateur pour
l'attribut «~heure~» d'un moment de
la journée}

\cadre{
\begin{pseudo}
	\Method{getHeure}{}{entier}
		\Return heure
	\EndMethod
\end{pseudo}
}

\bigskip

\cadre{
\begin{pseudo}
	\Method{setHeure}{uneHeure : entier}{}
		\Stmt heure \Gets uneHeure
	\EndMethod
\end{pseudo}
}

\bigskip

{\sffamily\bfseries\upshape
Que faire si le paramètre est invalide ? }

{
Dans l'exemple précédent, que se passerait-il si le
paramètre \textstyleCodeInsr{uneHeure} vaut 25 ? Une valeur aberrante sera
affectée à l'attribut \textstyleCodeInsr{heure}.}

{
Dans le cas de paramètres invalides, la plus mauvaise solution est de ne
rien faire. Le programme continuerait en croyant que tout s’est bien
passé et il court à la catastrophe. Il est préférable qu’un programme
se plante plutôt que de fournir une mauvaise réponse. }

{
Dans certains langages (comme le C), l’usage est que chaque module
retourne un entier indiquant s'il y a eu une erreur
(et laquelle). L’inconvénient est que le module appelant n’est pas
obligé de tenir compte de l’erreur.}

{
Les \textbf{exceptions} sont un mécanisme du même genre mais qui oblige
à fournir un code de traitement de l’erreur. Il ne sera pas étudié en
première année\footnote{\textstyleFootnoteSymbol{En tout cas pas au
cours de Logique mais vous étudierez cette notion au cours de Java.}}.
}

{
Cette année, nous nous contenterons d'indiquer
clairement dans nos codes qu'il
s'agit d'une situation anormale via
la primitive \textstyleMotCl{erreur} qui arrête le déroulement du
programme avec une courte explication du problème.}

{
La syntaxe que nous allons retenir est}

\cadre{
\begin{pseudo}
	\Stmt \K{erreur} "explication de l'erreur"
\end{pseudo}
}

\bigskip

{
Ce qui donne :}

\cadre{
\begin{pseudo}
	\Method{setHeure}{uneHeure : entier}{}
		\If{uneHeure < 0 OU uneHeure > 23}
			\Stmt \K{erreur} "heure invalide"
		\EndIf
		\LComment Pour être précis, il faudrait tenir compte que 24h00 est une
heure valide
		\Stmt heure \Gets uneHeure
	\EndMethod
\end{pseudo}
}

{
\textbf{Remarque} : Accéder à la valeur d’un attribut ne pose pas de
problème de validité. Alors pourquoi rendre l’attribut complètement
privé plutôt que de n’empêcher que des modifications directes (une
sorte d’attribut en \textit{lecture seule}) ? Cela permet de changer de
façon transparente la représentation des données. \ Nous y reviendrons
plus loin.}

\marginicon{exercice}
{\sffamily\bfseries\scshape
Exercice - encapsulation}

{
Sans le savoir, vous avez déjà défini des accesseurs (lisez: accesseurs
et mutateurs) pour le téléviseur. Lesquels ? En suivant la convention
de nom pour les accesseurs, quels noms auraient-ils dû porter ?}

\section[La notion de classe et d'instance]{La notion
de classe et d'instance}
{
Pour pouvoir utiliser des objets nous allons devoir les définir
(expliciter leur état et leur comportement). Cette définition est
commune à tous les objets similaires. Par exemple toutes les dates ont
un même comportement et un même type d'état (une
journée du calendrier).}

\marginicon{definition}
{\sffamily\bfseries\upshape
{
Une \textbf{classe} est un ensemble d'objets qui ont en
commun les mêmes méthodes et qui partagent les mêmes types
d'attributs.}}

{
Une \textbf{instance}\footnote{Vous pouvez considérer les termes
«~instance de classe~» et «~objet~» comme synonymes. }
d'une classe est un objet particulier
d'une classe qui peut activer les méthodes de la
classe et qui a des valeurs particulières pour ses attributs.}

{
On peut établir le parallélisme avec les types de base que vous avez
déjà vus. Définir une classe revient à définir un nouveau type de
données. En gros, on peut dire qu'un\textbf{ }objet
est à une classe ce qu'une variable est à un type.}

{
Comprenons bien que les objets d'une même classe ont le
même «~type~» d'état mais pas le même état proprement
dit. Deux objets «~date~» représentent tous deux une date du calendrier
mais pas (forcement) la même ! Ils auront donc les mêmes attributs mais
avec des valeurs différentes !}

{\sffamily\bfseries\upshape
Définition d'une classe}

{
Nous devons d'abord définir une classe avant de pouvoir
en instancier les objets que nous voulons utiliser. Précisons la
syntaxe utilisée pour définir une classe}

\cadre{
\begin{pseudo}
	\Class{NomDeLaClasse}
		\Private
			\LComment La liste des attributs (donc privés par convention)
		\Public
			\LComment La liste des méthodes publiques
		\Private
			\LComment La liste des méthodes privées
	\EndClass
	\LComment Par souci de lisibilité, on pourra indiquer uniquement 
	les signatures des
	\LComment méthodes et donner le code complet des méthodes à 
	la suite de la classe.
\end{pseudo}
}

\bigskip

{
\textbf{Exemple} : la classe Moment qui représente un moment de la
journée}

\cadre{
\begin{pseudo}
	\Class{Moment}
		\Private
			\Decl heure : entier
			\Decl minute : entier
			\Decl seconde : entier
		\Public
			\MethodSign{getHeure}{}{entier}
			\MethodSign{getMinute}{}{entier}
			\MethodSign{getSeconde}{}{entier}
			\MethodSign{setHeure}{uneHeure : entier}{}
			\MethodSign{setMinute}{uneMinute : entier}{}
			\MethodSign{setSeconde}{uneSeconde : entier}{}
			\MethodSign{estMatin}{}{booléen}
	\EndClass
	
	\Method{estMatin}{}{booléen}
		\Return heure < 12
	\EndMethod
	\LComment + les accesseurs et les mutateurs

\end{pseudo}
}

\bigskip

{\sffamily\bfseries\upshape
Instanciation d'une classe}

{
«~Instancier~» signifie créer un objet d'une classe.
Cela s'écrit avec l'instruction
\textstyleMotCl{nouveau}. Pour lui donner un nom, on
l'assigne à une variable déclarée du type de la
classe.}

\cadre{
\begin{pseudo}
	\Decl nomObjet : nomClasse 
	\RComment déclaration de l'objet
	\Let nomObjet \Gets \K{nouveau} nomClasse() 
	\RComment instanciation de l'objet
\end{pseudo}
}

{
\textbf{Exemple} : pour créer un moment de la journée}

\cadre{
\begin{pseudo}
	\Module{test}{}{}
		\Decl midi : Moment
		\RComment réf. 1
		\Let midi \Gets \K{nouveau} Moment()
		\RComment réf. 2
		\Stmt midi.setHeure( 12 )
		\Stmt midi.setMinute( 0 )
		\Stmt midi.setSeconde( 0 )
		\RComment réf. 3
		\If{midi.estMatin()}
			\Write "Midi est considéré comme
			étant encore le matin"
		\Else 
			\Write "Midi est considéré comme
			étant l'après-midi"
		\EndIf
	\EndModule
\end{pseudo}
}

{
Remarquez qu'il y a une différence importante entre les
objets et les types de bases. Lorsqu'on déclare une
variable d'un type de base, cela alloue
automatiquement un espace mémoire pour cette variable.
C'est différent avec les objets. La déclaration
n'entraine qu'une réservation mémoire
pour une «~référence~» vers un objet. Celui-ci
n'existe pas encore. Il sera créé (et sa mémoire
allouée) via une instruction spécifique (\textstyleMotCl{nouveau}). On
parle de variable «\textit{~dynamique~}». Le nom est alors~une
«~référence~» vers l’objet. Les avantages de cette dissociation seront
évidents lorsque nous parlerons de la notion de \textit{constructeur}.}

{
Après la déclaration (réf. 1), on a :}

\begin{center}
\tablehead{}
\begin{supertabular}{m{2.2089999cm}}
\centering\arraybslash  midi\\\hline
\multicolumn{1}{|m{2.2089999cm}|}{\centering\arraybslash
\itshape rien}\\\hline
\end{supertabular}
\end{center}
{
Après la création (réf. 2), on a :}

\begin{center}
\tablehead{}
\begin{supertabular}{m{2.578cm}m{2.261cm}|m{3.162cm}|}
\centering  midi &
\multicolumn{1}{m{2.261cm}}{~
} &
\multicolumn{1}{m{3.162cm}}{\centering\arraybslash
 Moment}\\\hhline{-~-}
\multicolumn{1}{|m{2.578cm}|}{~
} &
\centering \sffamily → &
\centering\arraybslash  heure = ?\\\hhline{-~~}
~
 &
~
 &
\centering\arraybslash  minute = ?\\
~
 &
~
 &
\centering\arraybslash  seconde = ?\\\hhline{~~-}
\end{supertabular}
\end{center}
{
Après l'action des mutateurs (réf. 3), on a :}


\bigskip

\begin{center}
\tablehead{}
\begin{supertabular}{m{2.578cm}m{2.261cm}|m{3.162cm}|}
\centering  midi &
\multicolumn{1}{m{2.261cm}}{~
} &
\multicolumn{1}{m{3.162cm}}{\centering\arraybslash
 Moment}\\\hhline{-~-}
\multicolumn{1}{|m{2.578cm}|}{~
} &
\centering \sffamily → &
\centering\arraybslash  heure = 12\\\hhline{-~~}
~
 &
~
 &
\centering\arraybslash  minute = 0\\
~
 &
~
 &
\centering\arraybslash  seconde = 0\\\hhline{~~-}
\end{supertabular}
\end{center}

\marginicon{exercice}
{\sffamily\bfseries\scshape
Exercices – classe et instance}

\liststyleWWviiiNumi
\begin{enumerate}
\item {
Pour les étudiants, vous avez déjà écrit les attributs et les en-têtes des
méthodes. Regroupez le tout en une classe \textstyleCodeInsr{Etudiant}
en respectant les notations que vous venez de voir.}
\item {
Écrivez un module qui fait rater un étudiant puis qui le fait réussir.}
\end{enumerate}

\bigskip

\section{Les constructeurs}
{
L'encapsulation nous permet de contrôler
l'état de l'objet et de
l'empêcher de tomber dans un état invalide. Mais
qu'en est-il de l'état de départ ?
Est-il valide ?}

{
Il serait bon, lorsqu'on crée un objet (via
\textstyleMotCl{nouveau}) de pouvoir indiquer l'état
initial de l'objet et que cet état puisse être validé.
C'est le rôle précis des constructeurs.}


\marginicon{definition}
{\sffamily\bfseries\upshape{
Un \textbf{constructeur} est une méthode particulière permettant
d'initialiser les attributs d'un
objet lors de sa création effective. Elle porte le même nom que sa
classe et ne retourne pas de valeur.}}

{
Il peut y avoir plusieurs constructeurs ce qui permet
d'offrir plusieurs possibilités
d'indiquer l'état initial de
l'objet.}

{
Remarquez que cela demande de définir plusieurs méthodes qui portent le
même nom. C'est ce qu'on appelle la
\textbf{surcharge}. Des méthodes éponymes (c-à-d de même nom) doivent
pouvoir être différenciées via leur signature (la liste de leurs
paramètres).}

{\bfseries
Exemple}

{
Écrivons des constructeurs pour un moment de la journée :}

\cadre{
\begin{pseudo}
	\Class{Moment}
		\Private
			\LComment pas de changement
			\Decl heure : entier
			\Decl minute : entier
			\Decl seconde : entier
		\Public
			\ConstrSign{Moment}{uneHeure, uneMinute, uneSeconde : entiers}
			\ConstrSign{Moment}{uneHeure, uneMinute : entiers}
			\RComment 0 seconde par défaut
			\ConstrSign{Moment}{uneHeure : entier}
			\RComment initialiser à une heure pile
			\Empty
			\LComment pas de changement au niveau des méthodes :
			\MethodSign{getHeure}{}{entier}
			\MethodSign{getMinute}{}{entier}
			\MethodSign{getSeconde}{}{entier}
			\MethodSign{setHeure}{uneHeure : entier}{}
			\MethodSign{setMinute}{uneMinute : entier}{}
			\MethodSign{setSeconde}{uneSeconde : entier}{}
			\MethodSign{estMatin}{}{booléen}
	\EndClass
	\end{pseudo}
}

\cadre{
\begin{pseudo}
	\Constr{Moment}{uneHeure, uneMinute, uneSeconde : entiers}
		\Stmt setHeure(uneHeure)
		\Stmt setMinute(uneMinute)
		\Stmt setSeconde(uneSeconde)
		\LComment Il est courant en logique qu'un constructeur 
		appelle les mutateurs 
		\LComment afin d'effectuer les tests sans avoir à les
		dupliquer.
		\LComment Mais c'est une démarche que vous éviterez de
		\LComment faire en Java par exemple.
	\EndConstr
	
	\Empty
	\Constr{Moment}{uneHeure, uneMinute : entiers}
		\Stmt setHeure(uneHeure)
		\Stmt setMinute(uneMinute)
		\Stmt setSeconde(0)
	\EndConstr
	
	\Empty
	\Constr{Moment}{uneHeure : entier}
		\Stmt setHeure(uneHeure)
		\Stmt setMinute(0)
		\Stmt setSeconde(0)
	\EndConstr
		
	\Empty
	\LComment + les accesseurs, les mutateurs et les autres méthodes

\end{pseudo}
}

{
Lorsqu'on instancie un objet, les paramètres
qu'on donne déterminent le constructeur qui est
effectivement utilisé pour initialiser l'état de
l'objet.}

{
\textbf{Exemple} : Instancions quelques moments de la journée}

\cadre{
\begin{pseudo}
	\Let moment11 \Gets \K{nouveau} Moment(14, 23, 56)
	\Let moment11 \Gets \K{nouveau} Moment(9, 30)
	\Let moment11 \Gets \K{nouveau} Moment(17)
\end{pseudo}
}

{
Le fait qu'un objet est instancié via la primitive
\textbf{nouveau} et pas implicitement à la déclaration permet de
postposer sa construction effective au moment où
l'état initial qu'on veut lui donner
sera connu (ce qui peut résulter d'un calcul). On est
ainsi assuré que tous les objets manipulés sont valides ce qui permet
d’éviter les situations où une méthode fait des dégâts suite à la
manipulation d’un objet invalide.}

\marginicon{exercice}
{\sffamily\bfseries\scshape
Exercices - constructeur}

\liststyleWWviiiNumi
\begin{enumerate}
\item {
Écrivez un ou des constructeur(s) pour un \textstyleCodeInsr{Etudiant}}
\item {
Adaptez le module écrit plus haut pour qu'il fasse rater 
puis réussir l'étudiant "Dupont, g12345, en 1ère année,
non doubleur".}
\end{enumerate}

\bigskip

\section{Du choix de la représentation de l'état}
{
Lorsqu'on définit une classe, il faut choisir les
attributs qui vont permettre de représenter l'état des
objets. Cela peut paraitre immédiat mais il n'en est
rien.}

{\bfseries
Exemple}

{
Pour un moment de la journée, nous avons choisi
d'utiliser trois attributs entiers (les heures, les
minutes et les secondes). Nous aurions tout aussi bien pu choisir
d'utiliser un seul entier représentant le nombre de
secondes écoulées depuis minuit.}

{
Ces deux représentations sont tout-à-fait équivalentes en terme de
potentiel mais la grande différence est l'efficacité
du code des méthodes. }

{
Prenons deux méthodes symptomatiques : celle qui donne
l'heure et celle qui compare deux moments de la
journée. La première est beaucoup plus simple à écrire et plus rapide
avec la première représentation alors que la seconde méthode est plus
simple à écrire et plus rapide avec la seconde représentation.}

{
Dès lors, quelle représentation choisir ? Il faut examiner, pour chaque
représentation possible, le nombre de méthodes qui sont efficaces mais
aussi imaginer la fréquence de leur utilisation (ce qui est difficile
et changeant). Heureusement, ce choix n'est pas
définitif. Si on change d'avis, on peut changer la
représentation. Il faudra bien sûr réécrire les méthodes de la classe
mais il ne faudra rien changer au reste du code, c-à-d les lignes du
code utilisant la classe. C’est d’ailleurs là une des grandes forces de
la programmation orientée objet.}

\marginicon{exercice}
{\sffamily\bfseries\scshape
Exercices – représentation de l'état}

\liststyleWWviiiNumi
\begin{enumerate}
\item {
Compléter la classe Moment en écrivant la méthode «~getHeure~» et celle
qui compare deux moments pour les deux représentations imaginées
ci-dessus.}
\item {
Écrire le module qui crée deux moments de la journée et vérifie si le
premier est avant le second. Ce code dépend-il des attributs choisis
pour définir la classe Moment ?}
\end{enumerate}

\bigskip

\marginicon{attention}
{\bfseries
Remarque}

{
Précédemment, nous avons défini un \textbf{accesseur} comme une méthode
permettant d’accéder à la valeur d’un attribut. Mais c’est au
développeur à définir quels sont les attributs ; c’est totalement caché
à l’utilisateur de la classe. On voit donc bien que cette notion
d’accesseur n’a pleinement de sens qu’en interne, pour le développeur
de la classe. Pour l’utilisateur il s’agit d’une méthode comme les
autres.}

\section{La mort d'un objet}
{
On sait que déclarer une variable ou créer un objet réserve de l’espace
en mémoire. On ne s’est jamais demandé quand cet espace mémoire est
libéré.}

{
Pour les variables locales d’un module ou d'une
méthode, la réponse est simple : l’espace mémoire est récupéré
lorsqu’on arrive à la fin du module ou de la méthode.}

{
Pour les objets, c’est un peu plus compliqué. L’espace réservé pour
contenir la référence (voir paragraphe suivant) est bien libéré à la
fin du module puisque la variable cesse d’exister. Par contre l’espace
réservé dynamiquement pour contenir l’objet lui-même (par la primitive
\textstyleMotCl{nouveau}) est toujours là et bien là !}

{
Mais alors, il n’est plus référencé et donc plus utilisable ? Pas
forcément. En effet, il est possible qu’il soit référencé par plusieurs
références. Si certaines sont détruites, il se peut que d’autres
continuent à exister. Ce sera le cas, par exemple, si l’objet constitue
la valeur de retour de la méthode ; sa référence à l’intérieur du
module est détruite mais il sera toujours accessible par une référence
du module appelant.}

{
Mais que faire quand on n’a plus besoin d’un objet ? On trouve
typiquement deux approches dans les langages OO.}

{\sffamily\bfseries\upshape
Destruction explicite de l’objet}

{
Dans cette approche (qui est celle de C++ notamment), c’est au
programmeur lui-même qu’il incombe de détruire explicitement un objet
et ainsi de permettre au système de récupérer l’espace mémoire. }

{
Cette technique offre au programmeur un grand contrôle sur l’utilisation
de la mémoire mais offre malheureusement quelques inconvénients.}

\liststyleListv
\begin{itemize}
\item {
Cela demande une grande attention lors de la programmation afin de
récupérer tout l’espace qui peut l’être. Dans le cas contraire, on
gaspille de la mémoire.}
\item {
Dans l’autre sens, il ne faut pas trop détruire. Si, par mégarde, on
détruit un objet qui est encore référencé et qu'on
utilise cette référence, le comportement du programme est imprévisible
(la mémoire peut avoir été utilisée pour autre chose).}
\item {
Un objet peut contenir des références à d’autres objets. La destruction
est alors un processus non trivial qui peut sensiblement alourdir et
obscurcir le code.}
\end{itemize}
{\sffamily\bfseries\upshape
Utilisation d’un \textit{garbage collector} (ramasse-miettes)}

{
Cette autre approche (choisie notamment par Java) enlève au programmeur
toute (ou presque) responsabilité quant à la gestion de la mémoire. De
temps en temps, ou lorsque le besoin s’en fait sentir, un composant du
système appelé \textit{garbage collector} se met au travail. Son rôle
est justement de récupérer l’espace qui n’est plus utilisé. Pour cela,
il considère que tout objet qui n'est plus accessible
(parce que plus aucune référence ne permet d’y accéder) peut être
détruit.}

{
Par facilité et parce que cela correspond au cours de Java que vous
suivez cette année, nous adopterons dans ce cours cette seconde
approche, c'est à dire qu'il ne faut
pas se préoccuper de ce problème ;-)}

\section{Quelques éléments de syntaxe}
{
Clarifions certaines notations liées aux objets.}

\liststyleListv
\begin{itemize}
\item {
On peut directement afficher un objet. Cela affiche son état,
c'est-à-dire les valeurs de ses attributs dans
l'ordre où ils apparaissent dans la définition de la
classe.
\\
\bigskip
\cadre{
\begin{pseudo}
	\Decl rendezVous : Moment
	\Let rendezVous \Gets \K{nouveau} Moment(14, 23, 56)
	\Write rendezVous 
	\RComment affichera 14, 23 et 56 dans un format lisible quelconque
\end{pseudo}
}
}
\bigskip

\item {
Un nom d'objet est en fait une \textbf{référence} à
l'objet. Ainsi l'assignation ne copie
pas l'objet mais sa référence. Au final, nous avons
deux noms identifiant le même objet}
\\
\bigskip
\cadre{
\begin{pseudo}
	\Decl moment1, moment2 : Moment
	\Let moment1 \Gets \K{nouveau} Moment(14, 23, 56)
	\Let moment2 \Gets moment1
	\RComment moment1 et moment2 désignent le même objet
	\Stmt moment2.setHeure( 12 )
	\Write moment1.getHeure() 
	\RComment affiche 12 !!!
\end{pseudo}
}

\bigskip

\begin{center}
\tablehead{}
\begin{supertabular}{m{2.19cm}m{2.2779999cm}|m{2.975cm}|m{2.34cm}m{2.146cm}}
\hhline{-~-~-}
\multicolumn{1}{|m{2.19cm}|}{\centering  moment1}
&
\centering \sffamily → &
\centering  Heure = \sout{14} 12 &
\multicolumn{1}{m{2.34cm}|}{\centering \sffamily
←} &
\multicolumn{1}{m{2.146cm}|}{\centering\arraybslash
 moment2}\\\hhline{-~~~-}
~
 &
~
 &
\centering  minute = 23 &
~
 &
~
\\
~
 &
~
 &
\centering  seconde = 56 &
~
 &
~
\\\hhline{~~-~~}
\end{supertabular}
\end{center}

\bigskip

\item {
Le signe «~=~» permet de tester que deux noms référencent le même objet.
Pour tester que deux objets différents sont dans le même état, on
utilise la méthode «~égal~».}
\\
\bigskip
\cadre{
\begin{pseudo}
	\Decl moment1, moment2, moment3 : Moment
	\Empty
	\Let moment1 \Gets {nouveau} Moment( 14, 23, 56 )
	\Let moment2 \Gets moment1 
	\RComment moment1 et moment2 désignent le même objet
	\Empty
	\Let moment3 \Gets {nouveau} Moment( 14, 23, 56 )
	\Write moment1 = moment2
	\RComment vrai
	\Write moment1 = moment3 
	\RComment faux
	\Write moment1.égal(moment2) 
	\RComment vrai
	\Write moment1.égal(moment3) 
	\RComment vrai
	\Empty 
	\Stmt moment2.setHeure( 12 )
	\Write moment1.égal(moment2) 
	\RComment vrai
	\Write moment1.égal(moment3) 
	\RComment faux
\end{pseudo}
}
\end{itemize}
\bigskip

\marginicon{exercice}
{\sffamily\bfseries\scshape
Exercice – méthode égal()}
{
Écrire la méthode égal() pour la classe Moment.}

\bigskip

{
N.B. : On supposera par la suite qu'une telle méthode
existe par défaut pour toutes les nouvelles classes.}

\liststyleListv
\begin{itemize}
	\item {
	Un attribut privé n'est pas connu en dehors de la
	classe. Précisons : un attribut privé n'est connu que
	des instances de cette classe, ce qui signifie qu'il
	est également connu par tous les autres objets de la même
	classe.

	\textbf{Exemple} : écrivons la méthode qui teste si un moment précède un
	autre (en supposant que l'état est représenté par un
	seul entier, totalSecondes, le nombre de secondes depuis minuit)
	\\
	\bigskip
	\cadre{
	\begin{pseudo}
		\Method{estAntérieur}{autre : Moment}{booléen}
			\Return totalSecondes < autre.totalSecondes
			\LComment c'est équivalent à \textstyleMotCl{retourner} totalSecondes < autre.getTotalSecondes()
		\EndMethod
	\end{pseudo}
	}
	\bigskip
	}
	\item {
	Lorsqu'il est déclaré, un nom d'objet
	ne référence encore aucun objet. Cela s'indique par la
	valeur «~rien~». On peut aussi utiliser cette valeur pour enlever toute
	référence vers un objet.
	\\
	\bigskip
	\cadre{
	\begin{pseudo}
		\Decl moment : Moment
		\RComment moment = rien
		\Let moment \Gets \K{nouveau} Moment( 14, 23, 56 )
		\RComment moment ${\neq}$ rien
		\Let moment \Gets rien
		\RComment moment = rien
	\end{pseudo}
	}
	}
\end{itemize}
	
\section{Représentation modélisée d'une classe}
{
Un dessin étant souvent plus lisible qu'un texte, on
peut représenter graphiquement une classe. Une notation courante est
celle utilisée en UML\footnote{{Unified
Modeling Langage. }On vous en parlera plus en détail au cours
d'Analyse.}. Pour faire simple, une classe est
représentée par un rectangle composé de 3 zones : la première pour le
nom de la classe, la deuxième pour les attributs et la troisième pour
les méthodes. On indique par un signe «~+~» (resp. «~-~») que le membre
est public (resp. privé)}

{\bfseries
Exemple}

{
Remarquons qu'on indique la signature des méthodes mais
pas le code associé. En fonction du niveau de détail désiré, on
pourrait aussi omettre les paramètres et types de retour. }

\begin{center}
\begin{minipage}{4.905cm}
\begin{center}
\tablehead{}
\begin{supertabular}{|m{4.7050004cm}|}
\hline
\centering\arraybslash \bfseries Date\\\hline
{ {}- jour : entier}

{ {}- mois : entier}

 {}- année : entier\\\hline
{ + getJour() \textsf{→} entier}

{ + setJour( j: entier )}

{ + avancer1Jour()}

 ...\\\hline
\end{supertabular}
\end{center}
\end{minipage}
\end{center}

\bigskip


\bigskip

\section{Un exemple complet : une durée}
{
Examinons un exemple complet pour fixer les notions introduites par ce
chapitre. Lors de l'apprentissage du pseudo-code, vous
avez écrit quelques modules manipulant des heures (conversion du format
HMS en nombre de secondes depuis minuit, conversion inverse, différence
entre 2 heures, …). Il est souvent utile, lorsqu’on développe un
algorithme, d’avoir à sa disposition un tel type de données au même
titre que les types prédéfinis. Faisons-le !}

{\sffamily\bfseries\upshape
Ce que l’on veut vraiment}

{
Avant tout, il faut bien préciser ce que l’on veut décrire. L’«~heure~»
est un concept multifacettes. Parle-t-on de l’heure comme moment dans
la journée ou de l’heure comme représentant une durée ? Dans le premier
cas, elle ne peut dépasser 24h et la différence entre 2 heures n’a pas
de sens (ou plus précisément n’est pas une heure, mais une durée !).
Dans le deuxième cas, on n’a pas ces contraintes. Nous allons ici
adopter la deuxième approche et pour bien la distinguer, nous allons
plutôt appeler cela une \textbf{durée}.}

{\sffamily\bfseries\upshape
Le comportement (les méthodes)}

{
La première question à se poser est celle des services qu’on veut
fournir, c’est-à-dire des méthodes publiques de la classe. On doit
pouvoir \textit{construire} une durée. On doit pouvoir connaitre le
nombre d’heures, minutes ou secondes correspondant à une durée. On doit
pouvoir effectuer des calculs avec des durées (addition, soustraction).
Enfin, on doit pouvoir comparer des durées. Arrêtons-nous là, mais en
pratique, on pourrait trouver encore bon nombre d’autres méthodes qu’il
serait intéressant de fournir. Ce qui nous donne jusqu’à présent}

\cadre{
\begin{pseudo}
	\Class{Durée}
		\Private
			\LComment rien encore
		\Public
			\ConstrSign{Durée}{secondes : entier}
			\ConstrSign{Durée}{heure, minute, seconde : entiers}
			\Empty
			\MethodSign{getJour}{}{entier}
			\RComment nb de jours dans une durée
			\MethodSign{getHeure}{}{entier}
			\RComment entier entre 0 et 23 inclus
			\MethodSign{getMinute}{}{entier}
			\RComment entier entre 0 et 59 inclus
			\MethodSign{getSeconde}{}{entier}
			\RComment entier entre 0 et 59 inclus
			\Empty
			\MethodSign{getTotalHeures}{}{entier}
			\RComment Le nombre total d’heures
			\MethodSign{getTotalMinutes}{}{entier}
			\RComment Le nombre total de minutes
			\MethodSign{getTotalSecondes}{}{entier}
			\RComment Le nombre total de secondes
			\Empty
			\MethodSign{ajouter}{autreDurée : Durée}{}
			\MethodSign{différence}{autreDurée : Durée}{Durée}
			\MethodSign{égale}{autreDurée : Durée}{booléen}
			\MethodSign{plusPetit}{autreDurée : Durée}{booléen}
	\EndClass
\end{pseudo}
}

\bigskip

\marginicon{attention}
{\bfseries
Quelques remarques}

\liststyleListv
\begin{itemize}
	\item {
	On a deux constructeurs, ce qui offre plus de souplesse pour initialiser
	un objet. Ceci est un exemple supplémentaire du concept de
	«\textbf{~surcharge~}».}
	\item {
	Faisons bien la distinction entre les méthodes
	\textstyleCodeInsr{getXXX()} et \textstyleCodeInsr{getTotalXXX()}. Par
	exemple, la méthode \textstyleCodeInsr{getMinute()} retourne la valeur
	de la composante «~minutes~» dans une représentation HMS tandis que la
	méthode \textstyleCodeInsr{getTotalMinutes()} retourne le nombre total
	de minutes pour cette durée. Ex : pour 1h23’12’’,
	\textstyleCodeInsr{getMinute()} retourne 23 et
	\textstyleCodeInsr{getTotalMinutes()} retourne 83. Idem avec les heures
	et les secondes.}
	\item {
	Les méthodes \textstyleCodeInsr{getTotalXXX()} retournent le nombre
	(toujours entier) de XXX contenus dans la durée. Exemple, avec la durée
	0h23’52'’, \textstyleCodeInsr{getTotalMinutes()}
	retourne 23 et pas 24 (autrement dit, il n’y a pas d’arrondi vers le
	haut).}
	\item {
	Il n’y a pas de \textit{mutateur }(\textstyleCodeInsr{setXXX()}). Ce qui
	signifie qu’on ne peut pas changer directement la valeur de l’objet
	après son initialisation. On aurait pu en définir mais nous
	n'avons pas jugé utile de le faire dans ce cas
	précis.}
	\item {
	La méthode \textstyleCodeInsr{ajouter()} ne retourne rien. En effet,
	elle ajoute la durée à l’objet sur lequel est appelée la méthode. C’est
	un choix ; on aurait aussi pu dire que la méthode ne modifie pas
	l’objet mais en retourne un autre qui représente la somme. Dans ce cas,
	on l’aurait plutôt appelée «\textstyleCodeInsr{~plus( )}~».}
	\item {
	La méthode \textstyleCodeInsr{différence()}, elle, renvoie toujours une
	durée (positive).}
\end{itemize}

{\sffamily\bfseries
La représentation de l'état (les attributs)}

{
La question suivante est : «~Comment représenter une durée en interne ?
». Plusieurs possibilités existent. Par exemple :}

\liststyleListv
\begin{itemize}
\item {
Via le nombre d’heures, de minutes et de secondes}
\item {
Via le nombre total de secondes}
\item {
Via une chaine, par exemple au format «~HH :MM :SS~» où HH pourrait
éventuellement excéder 23.}
\end{itemize}
{
Le premier choix semble le plus évident mais réfléchissons-y de plus
près. D’une part, pourquoi se limiter aux heures. On pourrait
introduire un champ ‘\textstyleCodeInsr{jour}’ (après tout on a bien
une méthode \textstyleCodeInsr{getJour()}). }

{
Quel critère doit vraiment nous permettre de décider ? Il faut une
représentation qui soit suffisante (tout est représenté) et qui
permette d’écrire des méthodes lisibles et si possible efficaces
(c'est-à-dire où le calcul est rapide). Selon ces
critères, la deuxième représentation est de loin la meilleure. Ce qui
nous donne}

\cadre{
\begin{pseudo}
	\Class{Durée}
		\Private
			\Decl totalSecondes : entier
		\Public
			\LComment idem
	\EndClass
\end{pseudo}
}

{\sffamily\bfseries\upshape
L'implémentation}

{
On est à présent prêt pour écrire le code des méthodes. Ce qui nous
donne pour la classe dans son entièreté :}

\cadre{
\begin{pseudo}
	\Class{Durée}
		\Private
			\Decl totalSecondes : entier
		\Public
			\ConstrSign{Durée}{secondes : entier}
			\ConstrSign{Durée}{heure, minute, seconde : entiers}
			\Empty
			\MethodSign{getJour}{}{entier}
			\RComment nb de jours dans une durée
			\MethodSign{getHeure}{}{entier}
			\RComment entier entre 0 et 23 inclus
			\MethodSign{getMinute}{}{entier}
			\RComment entier entre 0 et 59 inclus
			\MethodSign{getSeconde}{}{entier}
			\RComment entier entre 0 et 59 inclus
			\Empty
			\MethodSign{getTotalHeures}{}{entier}
			\RComment Le nombre total d’heures
			\MethodSign{getTotalMinutes}{}{entier}
			\RComment Le nombre total de minutes
			\MethodSign{getTotalSecondes}{}{entier}
			\RComment Le nombre total de secondes
			\Empty
			\MethodSign{ajouter}{autreDurée : Durée}{}
			\MethodSign{différence}{autreDurée : Durée}{Durée}
			\MethodSign{égale}{autreDurée : Durée}{booléen}
			\MethodSign{plusPetit}{autreDurée : Durée}{booléen}
	\EndClass
\end{pseudo}
}

\cadre{
\begin{pseudo}
	\Constr{Durée}{secondes : entier}
		\If{secondes < 0}
			\Stmt \K{erreur} "paramètre négatif"
		\EndIf
		\Let totalSecondes \Gets secondes
	\EndConstr
	\Empty
	\Constr{Durée}{heure, minute, seconde : entiers}
		\If{heure < 0 OU minute < 0 OU seconde < 0}
			\Stmt \K{erreur} "un des paramètres est négatif"
		\EndIf
		\Let totalSecondes \Gets 3600*heure + 60*minute + seconde
	\EndConstr
\end{pseudo}
}

\cadre{
\begin{pseudo}
	\LComment Retourne le nombre de jours dans une 
	représentation JJ/HH:MM:SS
	\Method{getJour}{}{entier}
		\Return totalSecondes DIV (3600*24)
	\EndMethod
	\Empty
	\LComment Retourne le nombre d'heures dans une 
	représentation JJ/HH:MM:SS
	\Method{getHeure}{}{entier}
		\LComment On doit enlever les jours éventuels
		\Return (totalSecondes DIV 3600) MOD 24
	\EndMethod
	\Empty
	\LComment Retourne le nombre de minutes dans une 
	représentation JJ/HH:MM:SS
	\Method{getMinute}{}{entier}
		\LComment On doit enlever les heures éventuelles
		\Return (totalSecondes DIV 60) MOD 60
	\EndMethod
	\Empty
	\LComment Retourne le nombre de secondes dans une 
	représentation JJ/HH:MM:SS
	\Method{getSeconde}{}{entier}
		\LComment On doit enlever les minutes éventuelles
		\Return totalSecondes  MOD 60
	\EndMethod
\end{pseudo}
}

\cadre{
\begin{pseudo}

	\LComment Retourne le nombre entier d’heures complètes
	\Method{getTotalHeures}{}{entier}
		\Return totalSecondes DIV 3600
	\EndMethod
	\Empty
	\LComment Retourne le nombre entier de minutes complètes
	\Method{getTotalMinutes}{}{entier}
		\Return totalSecondes DIV 60
	\EndMethod
	\Empty
	\LComment Retourne le nombre entier de secondes complètes
	\Method{getTotalSecondes}{}{entier}
		\Return totalSecondes
	\EndMethod
\end{pseudo}
}

\cadre{
\begin{pseudo}
	\Method{ajouter}{autreDurée : Durée}{}
		\Let totalSecondes \Gets totalSecondes + autreDurée.totalSecondes
	\EndMethod
	\Empty
	\Method{différence}{autreDurée : Durée}{Durée}
		\Return \K{nouvelle} Durée(valeurAbsolue(totalSecondes - autreDurée.totalSecondes))
	\EndMethod
	\Empty
	\Method{égale}{autreDurée : Durée}{booléen}
		\Return totalSecondes = autreDurée.totalSecondes
	\EndMethod
	\Empty
	\Method{plusPetit}{autreDurée : Durée}{booléen}
		\Return totalSecondes < autreDurée.totalSecondes
	\EndMethod
\end{pseudo}
}


{
Et c’est tout ! Chaque méthode est très petite. C’est une constante en
orienté objet : écrire de petites méthodes qui font chacune une et une
seule chose bien précise.}

\section{Ce qu'on n'a pas vu...}
{
Nous n'avons évidemment pas épuisé le sujet de
l'orienté objet. Celui-ci est composé de trois piliers
: l'\textbf{encapsulation},
l'\textbf{héritage} et le \textbf{polymorphisme}.}

{
Nous venons de voir l'essentiel de la partie
«~encapsulation~». Les deux autres notions, plus complexes, seront
abordées au cours de Java mais également au cours de Logique de
deuxième année. En voici un aperçu :}

\marginicon{definition}
{\sffamily\bfseries\upshape
L'héritage}

{
L’héritage permet de définir une classe à partir d’une autre qui lui
sert de base.} 
{On peut alors}

\liststyleListv
\begin{itemize}
\item {
Étendre son état}
\item {
Augmenter ou modifier son comportement}
\end{itemize}
{
Par exemple, on pourra définir un \textstyleCodeInsr{Etudiant} à partir
de la notion de \textstyleCodeInsr{Personne}.}

{
La classe qui sert de point de départ est appelée \textbf{classe de
base}, \textbf{classe mère} ou encore \textbf{super-classe}. La classe
qu’on définit à partir d’une classe mère est appelée \textbf{classe
dérivée}, \textbf{classe fille} ou encore \textbf{sous-classe}.}

{
Cette notion est un préalable pour le pilier suivant, le
\textbf{polymorphisme}.}

\marginicon{definition}
{\sffamily\bfseries\upshape
Le polymorphisme}

{
Le polymorphisme permet d’utiliser un objet fille en lieu et place d’un
objet mère.} 
{Exposé aussi brièvement cela peut paraitre futile mais cela
permet de construire du code ayant une architecture élégante, robuste
et facilement adaptable.}

\section{Exercices}
\liststyleExercice
\begin{enumerate}
\item {\sffamily\bfseries
L'étudiant}
\end{enumerate}
{
Reprenez la classe \textstyleCodeInsr{Etudiant} qui a servi
d'exemple. Identifiez et écrivez les méthodes qui vous
paraissent utiles pour une telle classe.}

\liststyleExercice
\setcounter{saveenum}{\value{enumi}}
\begin{enumerate}
\setcounter{enumi}{\value{saveenum}}
\item {\sffamily\bfseries
Une personne}
\end{enumerate}
{
Créez une classe \textstyleCodeInsr{Personne}, une personne étant
constituée d'un nom, d'un prénom et
d'une date de naissance. Cette classe utilisera la
classe \textstyleCodeInsr{Date} de l'exercice
précédent.}

{
On doit pouvoir construire une personne :}

\liststyleListi
\begin{itemize}
\item {
avec 3 arguments : le nom (\textstyleCodeInsr{chaine}), le prénom
(\textstyleCodeInsr{chaine}) et la date de naissance de la personne
(\textstyleCodeInsr{Date})}
\item {
avec 2 arguments de type chaine : le nom et le prénom de la personne; la
date de naissance est alors initialisée à «\textstyleCodeInsr{~rien~}»}
\end{itemize}
{
Écrivez aussi tous les accesseurs et mutateurs que vous jugez
pertinents. Dans un module principal, créez une personne :}

\liststyleListi
\begin{itemize}
\item {
sans argument}
\item {
avec comme arguments {\textquotedbl}Durant{\textquotedbl} et
{\textquotedbl}Zébulon{\textquotedbl}}
\item {
avec comme arguments {\textquotedbl}Durant{\textquotedbl},
{\textquotedbl}Zébulon{\textquotedbl} et la date de naissance du
01/02/1980}
\end{itemize}
{
Pour réaliser les constructeurs recevant la date de naissance en
paramètre, il faudra tester si cette date n’est pas antérieure à la
date du jour. La date du jour est fournie par le constructeur de
\textstyleCodeInsr{Date} sans paramètre.}

\liststyleExercice
\begin{enumerate}
\item {\sffamily\bfseries
Anniversaire des personnes}
\end{enumerate}
{
Utilisons la classe \textstyleCodeInsr{Personne} écrite plus haut.
Écrire un module qui lit des \textstyleCodeInsr{Personne} (au clavier)
et affiche les noms et le nombre de celles nées ce mois-ci. On suppose
que la lecture de «~rien~» indique la fin des données.}

\liststyleExercice
\setcounter{saveenum}{\value{enumi}}
\begin{enumerate}
\setcounter{enumi}{\value{saveenum}}
\item {\sffamily\bfseries
Le rectangle (version orientée objet)}
\end{enumerate}
{
Nous avons déjà abordé les rectangles dans le chapitre des variables
structurées. Nous reprenons cet exercice sous l’angle de l’orienté
objet.}

{
{Créez une classe
}{\textbf{Rectangle}}{
permettant de définir des rectangles dont les cotés sont parallèles aux
axes des coordonnées dans un plan cartésien. Plusieurs représentations
sont possibles : }}

\liststyleListi
\begin{itemize}
\item {
la position d’un des sommets et les mesures des cotés}
\item {
les positions de deux sommets opposés}
\item {
la position du centre et les demi-mesures des cotés, etc.}
\end{itemize}
{
N’hésitez pas à utiliser la structure point définissant un point dans un
plan cartésien.}

{
Une fois les attributs choisis, écrire divers constructeurs :}

\liststyleListi
\begin{itemize}
\item {
sans arguments : le rectangle est un carré de coté 1 centré en (0,0)}
\item {
avec deux paramètres : les mesures des cotés horizontaux et verticaux,
le rectangle étant centré en (0,0)}
\item {
avec trois paramètres : la position du coin en haut à gauche (structure
Point) et les mesures des cotés horizontaux et verticaux }
\item {
avec deux paramètres de type Point : les positions de deux sommets
opposés}
\end{itemize}
{
Veillez à vérifier la validité des paramètres !}

{
Doter ensuite la classe de méthodes permettant :}

\liststyleListi
\begin{itemize}
\item {
d’obtenir la position du centre}
\item {
d’obtenir la position du coin inférieur droit}
\item {
de calculer le périmètre du rectangle}
\item {
de calculer la surface du rectangle}
\item {
de déplacer le rectangle en donnant l’amplitude du déplacement au niveau
des abscisses et des ordonnées}
\item {
de multiplier les dimensions du rectangle par un facteur k, le centre
restant au même endroit}
\item {
de faire pivoter le rectangle de 90° autour de son centre}
\item {
de vérifier si un rectangle à une intersection avec un autre rectangle}
\end{itemize}
{
Test : écrire un module TestRectangle en vue de tester le bon
fonctionnement de la classe rectangle. Ce module : }

\liststyleListi
\begin{itemize}
\item {
Crée un rectangle R1 par défaut}
\item {
Crée un rectangle R2 de cotés 5 et 7, et centré en (0,0)}
\item {
Crée un rectangle R3 possédant les sommets (-2,3) et (4, -5)}
\item {
Affiche la surface de R1 et le périmètre de R2}
\item {
Déplace R1 d’une unité vers le bas}
\item {
Déplace R2 de 2 unités vers la droite}
\item {
Grossit R3 d’un facteur 3}
\item {
Effectue une rotation de 90° à R2}
\item {
Indique si R2 et R3 possèdent une intersection}
\end{itemize}
{
À présent, choisissez une autre représentation des attributs et récrivez
tout le contenu de la classe (très long, courage !) réécrire ensuite le
module TestRectangle (très rapide !)}
