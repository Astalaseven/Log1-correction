\begin{sol}{3.4}
		\cadre{
		\begin{pseudo}
			\Module{surfaceTriangle}{}{}
				\Decl base, hauteur, surface : réels
				\Read base
				\Read hauteur
				\Let surface \Gets ( base * hauteur ) / 2
				\Write surface
			\EndModule
		\end{pseudo}
		}
	
\end{sol}
\begin{sol}{7.24}
		\cadre{
		\begin{pseudo}
			\LComment{Chiffrer un message en utilisant le chiffre de César}
			\Module{chiffrerCésar}{msgClair\In: chaine, déplacement\In: entier}{chaine}
				\Decl msgChiffré : chaine
				\Decl carClair : caractère
				\Decl i : entier
				\Empty
				\Let msgChiffré \Gets ""
				\For{i \K{de} 1 \K{à} long(msgClair)}
					\Let carClair \Gets car(msgClair,i)
					\Let carChiffré \Gets avancer(carClair, déplacement)
					\Let msgChiffré \Gets concat( msgChiffré, chaine(carChiffré) )
				\EndFor
				\Return msgChiffré
			\EndModule

			\Empty
			\LComment{Calcule la lettre qui est "delta" position plus loin dans l'alphabet (circulairement)}
			\Module{avancer}{lettre\In: caractère, delta\In: entier}{caractère}
				\Return lettre( (numLettre(lettre) + delta - 1) MOD 26 + 1 )
			\EndModule

			\Empty
			\LComment{Déchiffrer un message chiffré avec le chiffre de César}
			\Module{déchiffrerCésar}{msgClair\In: chaine, déplacement\In: entier}{chaine}
				\Return chiffrerCésar(msgClair, 26 - déplacement)
			\EndModule
		\end{pseudo}
		}
	
\end{sol}
