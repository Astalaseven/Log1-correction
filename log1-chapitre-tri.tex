%===============
\chapter{Le tri}
%===============

	\marginicon{objectif}
		Dans ce chapitre nous voyons quelques algorithmes simples pour trier un
		ensemble d'informations~: recherche des maxima, tri
		par insertion et tri bulle dans un tableau. 
		Des algorithmes plus efficaces seront vus
		en deuxième année.


%===================
\section{Motivation}
%===================

	Une application importante de l’informatique est le tri d’informations.
	La conception d’algorithmes efficaces pour ce type de traitement s’est
	imposée avec l’apparition de bases de données de taille importante. On
	citera par exemple la recherche d’un numéro de téléphone ou une adresse
	dans la version informatisée des pages blanches, ou de façon plus
	spectaculaire l’efficacité de certains moteurs de recherche qui
	permettent de retrouver en quelques secondes des mots clés arbitraires
	parmi plusieurs millions de pages sur le web.


	La recherche efficace d’information implique un tri préalable de
	celle-ci. En effet, si les données ne sont pas classées ou triées, le
	seul algorithme possible reviendrait à parcourir entièrement l’ensemble
	des informations. Pour exemple, il suffit d’imaginer un dictionnaire
	dans lequel les mots seraient mélangés de façon aléatoire au lieu
	d’être classés par ordre alphabétique. Pour trouver le moindre mot dans
	ce dictionnaire, il faudrait à chaque fois le parcourir entièrement !
	Il est clair que le classement préalable (ordre alphabétique) accélère
	grandement la recherche.


	Ainsi, recherche et tri sont étroitement liés, et la façon dont les
	informations sont triées conditionne bien entendu la façon de
	rechercher l’information (cf. algorithme de recherche dichotomique).
	Pour exemple, prenons cette fois-ci un dictionnaire des mots croisés
	dans lequel les mots sont d’abord regroupés selon leur longueur et
	ensuite par ordre alphabétique. La façon de rechercher un mot dans ce
	dictionnaire est bien sûr différente de la recherche dans un
	dictionnaire usuel. 

	Le problème central est donc le tri des informations. Celui-ci a pour
	but d’organiser un ensemble d’informations qui ne l’est pas \textit{à
	priori}. On peut distinguer trois grands cas de figure~:

	\liststyleWWviiiNumi
	\begin{enumerate}
		\item 
			D’abord les situations impliquant le classement total d’un ensemble de
			données «~brutes~», c’est-à-dire complètement désordonnées. Prenons
			pour exemple les feuilles récoltées en vrac à l’issue d’un examen ; il
			y a peu de chances que celles-ci soient remises à l’examinateur de
			manière ordonnée ; celui-ci devra donc procéder au tri de l’ensemble
			des copies, par exemple par ordre alphabétique des noms des étudiants,
			ou par numéro de groupe etc.
		\item 
			Ensuite les situations où on s’arrange pour ne jamais devoir trier la
			totalité des éléments d’un ensemble, qui resterait cependant à tout
			moment ordonné. Imaginons le cas d’une bibliothèque~dont les livres
			sont rangés par ordre alphabétique des auteurs~: à l’achat d’un nouveau
			livre, ou au retour de prêt d’un livre, celui-ci est immédiatement
			rangé à la bonne place. Ainsi, l’ordre global de la bibliothèque est
			maintenu par la répétition d’une seule opération élémentaire consistant
			à insérer à la bonne place un livre parmi la collection. C’est la
			situation que nous considérerions dans le cas d'une structure
			où les éléments sont ordonnés.
		\item 
			Enfin, les situations qui consistent à devoir re-trier des données
			préalablement ordonnées sur un autre critère. Prenons l’exemple d’un
			paquet de copies d’examen déjà triées sur l’ordre alphabétique des noms
			des étudiants, et qu’on veut re-trier cette fois-ci sur les numéros de
			groupe. Il est clair qu’une méthode efficace veillera à conserver
			l’ordre alphabétique déjà présent dans la première situation afin que
			les copies apparaissent dans cet ordre dans chacun des groupes.
	\end{enumerate}
	
	Le dernier cas illustre un classement sur une \textbf{clé complexe}
	(ou \textbf{composée}) impliquant la comparaison de plusieurs champs
	d’une même structure~: le premier classement se fait sur le numéro de
	groupe, et à numéro de groupe égal, l’ordre se départage sur le nom de
	l’étudiant. On dira de cet ensemble qu’il est classé en \textbf{majeur}
	sur le numéro de groupe et en \textbf{mineur} sur le nom d’étudiant.

	Notons que certains tris sont dits \textbf{stables} parce
	qu'en cas de tri sur une nouvelle clé, l’ordre de la
	clé précédente est préservé pour des valeurs identiques de la nouvelle
	clé, ce qui évite de faire des comparaisons sur les deux champs à la
	fois. Les méthodes nommées \textbf{tri par insertion}, \textbf{tri
	bulle} et \textbf{tri par recherche de minima successifs }(que nous
	allons aborder dans ce chapitre) sont stables.

	Le tri d’un ensemble d’informations n’offre que des avantages. Outre le
	fait déjà mentionné de permettre une recherche et une obtention rapide
	de l’information, il permet aussi l’application de traitements
	algorithmiques efficaces (comme par exemple celui des ruptures que nous
	verrons plus loin) qui s’avéreraient trop coûteux (en temps, donc en
	argent !) s’ils s’effectuaient sur des ensembles non triés.

	Certains tris sont évidemment plus performants que d’autres. Le choix
	d’un tri particulier dépend de la taille de l’ensemble à trier et de la
	manière dont il se présente, c’est-à-dire déjà plus ou moins ordonné.
	La performance quant à elle se juge sur deux critères~: le nombre de
	tests effectués (comparaisons de valeurs) et le nombre de transferts de
	valeurs réalisés. 
	
	Les algorithmes classiques de tri présentés dans ce chapitre le sont
	surtout à titre pédagogique. Ils ont tous une «~complexité en
	$n^2$~», ce qui veut dire que si $n$ est le nombre
	d’éléments à trier, le nombre d’opérations élémentaires (tests et
	transferts de valeurs) est proportionnel à $n^2$. Ils conviennent
	donc pour des ensembles de taille «~raisonnable~», mais peuvent devenir
	extrêmement lents à l’exécution pour le tri de grands ensembles, comme
	par exemple les données de l’annuaire téléphonique. Plusieurs solutions
	existent, comme la méthode de tri \textbf{Quicksort}. Cet algorithme
	très efficace faisant appel à la récursivité et qui sera étudié en
	deuxième année a une complexité en $n \log(n)$. 

	\marginicon{attention}
	Dans ce chapitre, les algorithmes traiteront du tri dans un
	\textbf{tableau d’entiers à une }\textbf{dimension}. Toute autre
	situation peut bien entendu se ramener à celle-ci moyennant la
	définition de la relation d’ordre propre au type de données utilisé. Ce
	sera par exemple, l’ordre alphabétique pour les chaines de caractères,
	l’ordre chronologique pour des objets \textstyleCodeInsr{Date} ou
	\textstyleCodeInsr{Moment} (que nous verrons
	plus tard), etc. De plus, le seul ordre envisagé sera l’ordre
	\textbf{croissant} des données. Plus loin, on envisagera le tri 
	d'autres structures de données.

	Enfin, dans toutes les méthodes de tri abordées, nous supposerons la
	taille physique du tableau à trier (notée $n$) égale à sa taille
	logique, celle-ci n’étant pas modifiée par l’action de tri.


%==========================
\section{Tri par insertion}
%===========================

	Cette méthode de tri repose sur le principe d’insertion de valeurs dans
	un tableau ordonné. 

	{\sffamily\bfseries\upshape
	Description de l’algorithme}

	Le tableau à trier sera à chaque étape subdivisé en deux sous-tableaux~:
	le premier cadré à gauche contiendra des éléments déjà ordonnés, et le
	second, cadré à droite, ceux qu’il reste à insérer dans le sous-tableau
	trié. Celui-ci verra sa taille s’accroitre au fur et à mesure des
	insertions, tandis que celle du sous-tableau des éléments non triés
	diminuera progressivement.

	Au départ de l’algorithme, le sous-tableau trié est le premier élément
	du tableau. Comme il ne possède qu’un seul élément, ce sous-tableau est
	donc bien ordonné ! Chaque étape consiste ensuite à prendre le premier
	élément du sous-tableau non trié et à l’insérer à la bonne place dans
	le sous-tableau trié.

	Prenons comme exemple un tableau \code{tab} de 20 entiers. 
	
	Au départ, le	sous-tableau trié est formé du premier élément, 
	\code{tab[1]} qui vaut 20~:

	\begin{center}
	\begin{tabular}{*{20}{>{\centering\sffamily\itshape\arraybackslash}m{0.47cm}}}
		 1 &
		 2 &
		 3 &
		 4 &
		 5 &
		 6 &
		 7 &
		 8 &
		 9 &
		 10 &
		 11 &
		 12 &
		 13 &
		 14 &
		 15 & 
		 16 &
		 17 &
		 18 &
		 19 &
		 20
		 \\
	\end{tabular}
	\begin{tabular}{|*{20}{>{\centering\arraybackslash}m{0.46cm}|}}
		\hline
		\multicolumn{1}{|m{0.49700004cm}|}{\cellcolor{gray!25}20} &
		{ 12} &
		{ 18} &
		{ 17} &
		{ 15} &
		{ 14} &
		{ 15} &
		{ 16} &
		{ 18} &
		{ 17} &
		{ 12} &
		{ 14} &
		{ 16} &
		{ 18} &
		{ 15} &
		{ 15} &
		{ 19} &
		{ 11} &
		{ 11} &
		{ 13}\\\hline
	\end{tabular}
	\end{center}

	\bigskip
	
	L’étape suivante consiste à insérer \code{tab[2]}, qui vaut 12, dans ce sous-tableau, de
	taille 2~:

	\begin{center}
	\begin{tabular}{*{20}{>{\centering\sffamily\itshape\arraybackslash}m{0.47cm}}}
		 1 &
		 2 &
		 3 &
		 4 &
		 5 &
		 6 &
		 7 &
		 8 &
		 9 &
		 10 &
		 11 &
		 12 &
		 13 &
		 14 &
		 15 & 
		 16 &
		 17 &
		 18 &
		 19 &
		 20
		 \\
	\end{tabular}
	\begin{tabular}{|*{20}{>{\centering\arraybackslash}m{0.46cm}|}}
		\hline
		\multicolumn{1}{|m{0.49700004cm}|}{\cellcolor{gray!25}12} &
		{\cellcolor{gray!25}20} &
		{18} &
		{ 17} &
		{ 15} &
		{ 14} &
		{ 15} &
		{ 16} &
		{ 18} &
		{ 17} &
		{ 12} &
		{ 14} &
		{ 16} &
		{ 18} &
		{ 15} &
		{ 15} &
		{ 19} &
		{ 11} &
		{ 11} &
		{ 13}\\\hline
	\end{tabular}
	\end{center}

	\bigskip

	Ensuite, c’est au tour de \code{tab[3]}, qui vaut 18, d’être inséré~:

	\begin{center}
	\begin{tabular}{*{20}{>{\centering\sffamily\itshape\arraybackslash}m{0.47cm}}}
		 1 &
		 2 &
		 3 &
		 4 &
		 5 &
		 6 &
		 7 &
		 8 &
		 9 &
		 10 &
		 11 &
		 12 &
		 13 &
		 14 &
		 15 & 
		 16 &
		 17 &
		 18 &
		 19 &
		 20
		 \\
	\end{tabular}
	\begin{tabular}{|*{20}{>{\centering\arraybackslash}m{0.46cm}|}}
		\hline
		{\cellcolor{gray!25}12} &
		{\cellcolor{gray!25}18} &
		{\cellcolor{gray!25}20} &
		{ 17} &
		{ 15} &
		{ 14} &
		{ 15} &
		{ 16} &
		{ 18} &
		{ 17} &
		{ 12} &
		{ 14} &
		{ 16} &
		{ 18} &
		{ 15} &
		{ 15} &
		{ 19} &
		{ 11} &
		{ 11} &
		{ 13}\\\hline
	\end{tabular}
	\end{center}
	
	\bigskip

	Et ainsi de suite jusqu’à insertion du dernier élément dans le
	sous-tableau trié. 

	{\sffamily\bfseries
	Algorithme}

	On combine la recherche de la position d’insertion et le décalage
	concomitant de valeurs.

	\cadre{
	\begin{pseudo}
		\LComment Trie le tableau reçu en paramètre (via un tri par insertion).
		\Module{triInsertion}{tab\InOut~: tableau[1 à n] d’entiers}{}
			\Decl i, j, valAInsérer~: entiers
			\For{i \K{de} 2 \K{à} n}
				\Let valAInsérer \Gets tab[i]
				\LComment recherche de l’endroit où insérer valAInsérer dans le 
				\LComment sous-tableau trié et décalage simultané des éléments
				\Let j \Gets i - 1
				\While{j ${\geq}$ 1 ET valAInsérer < tab[j]}
					\Let tab[j+1] \Gets tab[j]
					\Let j \Gets j – 1
				\EndWhile
				\Let tab[j+1] \Gets valAInsérer
			\EndFor
		\EndModule
	\end{pseudo}
	}

	\bigskip


%================================================
\section{Tri par sélection des minima successifs}
%=================================================
	
	Dans ce tri, on recherche à chaque étape la plus petite valeur de
	l’ensemble non encore trié et on peut la placer immédiatement 
	à sa position	définitive.

	{\sffamily\bfseries\upshape
	Description de l’algorithme}

	Prenons par exemple un tableau de 20 entiers. 
	
	La première étape consiste
	à rechercher la valeur minimale du tableau. Il s’agit de l’élément
	d’indice 10 et de valeur 17.
	
	\begin{center}
	\begin{tabular}{*{20}{>{\centering\sffamily\itshape\arraybackslash}m{0.47cm}}}
		 1 &
		 2 &
		 3 &
		 4 &
		 5 &
		 6 &
		 7 &
		 8 &
		 9 &
		 10 &
		 11 &
		 12 &
		 13 &
		 14 &
		 15 & 
		 16 &
		 17 &
		 18 &
		 19 &
		 20
		 \\
	\end{tabular}
	\begin{tabular}{|*{20}{>{\centering\arraybackslash}m{0.46cm}|}}
		\hline
		{20} &
		{ 52} &
		{ 61} &
		{ 47} &
		{ 82} &
		{ 64} &
		{ 95} &
		{ 66} &
		{ 84} &
		{\cellcolor{gray!25}17} &
		{ 32} &
		{ 24} &
		{ 46} &
		{ 48} &
		{ 75} &
		{ 55} &
		{ 19} &
		{ 61} &
		{ 21} &
		{ 30}\\\hline
	\end{tabular}
	\end{center}

	Celui-ci devrait donc apparaitre en 1\textsuperscript{ère }position du
	tableau. Hors, cette position est occupée par la valeur 20. Comment
	procéder dès lors pour ne perdre aucune valeur et sans faire appel à un
	second tableau ? La solution est simple, il suffit d’échanger le
	contenu des deux éléments d’indices 1 et 10~:
	
	\begin{center}
	\begin{tabular}{*{20}{>{\centering\sffamily\itshape\arraybackslash}m{0.47cm}}}
		 1 &
		 2 &
		 3 &
		 4 &
		 5 &
		 6 &
		 7 &
		 8 &
		 9 &
		 10 &
		 11 &
		 12 &
		 13 &
		 14 &
		 15 & 
		 16 &
		 17 &
		 18 &
		 19 &
		 20
		 \\
	\end{tabular}
	\begin{tabular}{|*{20}{>{\centering\arraybackslash}m{0.46cm}|}}
		\hline
		{\cellcolor{gray!25}17} &
		{ 52} &
		{ 61} &
		{ 47} &
		{ 82} &
		{ 64} &
		{ 95} &
		{ 66} &
		{ 84} &
		{ 20} &
		{ 32} &
		{ 24} &
		{ 46} &
		{ 48} &
		{ 75} &
		{ 55} &
		{ 19} &
		{ 61} &
		{ 21} &
		{ 30}\\\hline
	\end{tabular}
	\end{center}
	
	Le tableau se subdivise à présent en deux sous-tableaux, un sous-tableau
	déjà trié (pour le moment réduit au seul élément $tab[1]$) et le
	sous-tableau des autres valeurs non encore triées (de l’indice 2 à 20).
	On recommence ce processus dans ce second sous-tableau~: le minimum est
	à présent l'élément d’indice 17 et de valeur 19.
	Celui-ci viendra donc en 2\textsuperscript{ème} position, échangeant sa
	place avec la valeur 52 qui s’y trouvait~:

	\begin{center}
	\begin{tabular}{*{20}{>{\centering\sffamily\itshape\arraybackslash}m{0.47cm}}}
		 1 &
		 2 &
		 3 &
		 4 &
		 5 &
		 6 &
		 7 &
		 8 &
		 9 &
		 10 &
		 11 &
		 12 &
		 13 &
		 14 &
		 15 & 
		 16 &
		 17 &
		 18 &
		 19 &
		 20
		 \\
	\end{tabular}
	\begin{tabular}{|*{20}{>{\centering\arraybackslash}m{0.46cm}|}}
		\hline
		{\cellcolor{gray!25}17} &
		{\cellcolor{gray!25}19} &
		{ 61} &
		{ 47} &
		{ 82} &
		{ 64} &
		{ 95} &
		{ 66} &
		{ 84} &
		{ 20} &
		{ 32} &
		{ 24} &
		{ 46} &
		{ 48} &
		{ 75} &
		{ 55} &
		{ 52} &
		{ 61} &
		{ 21} &
		{ 30}\\\hline
	\end{tabular}
	\end{center}

	Le sous-tableau trié est maintenant formé des deux premiers éléments, et
	le sous-tableau non trié par les 18 éléments suivants. 
	
	Pour un tableau de taille $n$, 
	à l’étape $i$ de cet algorithme, on sélectionne donc le minimum du
	sous-tableau non trié (entre les indices $i$ et $n$). Une
	fois le minimum localisé, on l’échange avec l’élément d’indice
	$i$. À chaque étape, la taille du sous-tableau trié augmente de
	1 et celle du sous-tableau non trié diminue de 1. L’algorithme s’arrête
	à la $n$\textsuperscript{ème} étape, lorsque le sous-tableau
	trié correspond au tableau de départ. En pratique l’arrêt se fait après
	l’étape $n - 1$, car lorsque le sous-tableau non trié n’est plus
	que de taille 1, il contient nécessairement le maximum de l’ensemble.

	{\sffamily\bfseries\upshape
	Algorithme}
	
	\cadre{
	\begin{pseudo}
		\LComment Trie le tableau reçu en paramètre (via un tri par sélection des minima successifs).
		\Module{triSélectionMinimaSuccessifs}{tab\InOut~: tableau[1 à n] d’entiers}{}
			\Decl i, indiceMin~: entier
			\For{i \K{de} 1 \K{à} n – 1}
			\RComment i correspond à l’étape de l’algorithme
				\Let indiceMin \Gets positionMin( tab, i, n )
				\Stmt swap( tab[i], tab[indiceMin] )
			\EndFor
		\EndModule
	\end{pseudo}
	}

	\bigskip

	\cadre{
	\begin{pseudo}
		\LComment Retourne de l’indice du minimum entre les indices début et fin du tableau reçu.
		\Module{positionMin}{tab\In~: tableau[1 à n] d’entiers, début, fin~: entiers}{entier}
			\Decl indiceMin, i~: entiers
			\Let indiceMin \Gets début
			\For{i \K{de} début+1 \K{à} fin}
				\If{tab[i] < tab[indiceMin]}
					\Decl indiceMin \Gets i
				\EndIf
			\EndFor
			\Return indiceMin
		\EndModule
		\Empty
		\LComment {Échange le contenu de 2 variables.}
		\Module{swap}{a\InOut, b\InOut~: entiers}{}
			\Decl aux~: entiers
			\Let aux \Gets a
			\Let a \Gets b
			\Let b \Gets aux
		\EndModule
	\end{pseudo}
	}

	

%==================
\section{Tri bulle}
%===================
	
	Il s’agit d’un tri par \textbf{permutations} ayant pour but d’amener à
	chaque étape à la «~surface~» du sous-tableau non trié (on entend par
	là l’élément d’indice minimum) la valeur la plus petite, appelée la
	\textbf{bulle}. La caractéristique de cette méthode est que les
	comparaisons ne se font qu’entre éléments consécutifs du tableau.

	{\sffamily\bfseries\upshape
	Description de l’algorithme}

	Prenons pour exemple un tableau de taille 14. En partant de la fin du
	tableau, on le parcourt vers la gauche en comparant chaque couple de
	valeurs consécutives. Quand deux valeurs sont dans le désordre, on les
	permute. Le premier parcours s’achève lorsqu’on arrive à l’élément
	d’indice 1 qui contient alors la «~bulle~»,
	c'est-à-dire la plus petite valeur du tableau, soit 1~:

	\clearpage
	\begin{center}
	\begin{tabular}{*{14}{>{\centering\sffamily\itshape\arraybackslash}m{0.75cm}}}
		 1 &
		 2 &
		 3 &
		 4 &
		 5 &
		 6 &
		 7 &
		 8 &
		 9 &
		 10 &
		 11 &
		 12 &
		 13 &
		 14
		\\
	\end{tabular}
	\end{center}

	\begin{center}
	\begin{tabular}{|*{14}{>{\centering\arraybackslash}m{0.75cm}|}}
		\hline
		{10} &
		{  5} &
		{ 12} &
		{ 15} &
		{  4} &
		{  8} &
		{  1} &
		{  7} &
		{ 12} &
		{ 11} &
		{  3} &
		{  6} &
		{  5} &
		{\cellcolor{gray!25}4}\\\hline
	\end{tabular}
	\end{center}

	\begin{center}
	\begin{tabular}{|*{14}{>{\centering\arraybackslash}m{0.75cm}|}}
		\hline
		{10} &
		{  5} &
		{ 12} &
		{ 15} &
		{  4} &
		{  8} &
		{  1} &
		{  7} &
		{ 12} &
		{ 11} &
		{  3} &
		{  6} &
		{\cellcolor{gray!25}4} &
		{  5}\\\hline
	\end{tabular}
	\end{center}

	\begin{center}
	\begin{tabular}{|*{14}{>{\centering\arraybackslash}m{0.75cm}|}}
		\hline
		{10} &
		{  5} &
		{ 12} &
		{ 15} &
		{  4} &
		{  8} &
		{  1} &
		{  7} &
		{ 12} &
		{ 11} &
		{  3} &
		{\cellcolor{gray!25}4} &
		{  6} &
		{  5}\\\hline
	\end{tabular}
	\end{center}

	\begin{center}
	\begin{tabular}{|*{14}{>{\centering\arraybackslash}m{0.75cm}|}}
		\hline
		{10} &
		{  5} &
		{ 12} &
		{ 15} &
		{  4} &
		{  8} &
		{  1} &
		{  7} &
		{ 12} &
		{ 11} &
		{\cellcolor{gray!25}3} &
		{  4} &
		{  6} &
		{  5}\\\hline
	\end{tabular}
	\end{center}

	\begin{center}
	\begin{tabular}{|*{14}{>{\centering\arraybackslash}m{0.75cm}|}}
		\hline
		{10} &
		{  5} &
		{ 12} &
		{ 15} &
		{  4} &
		{  8} &
		{  1} &
		{  7} &
		{ 12} &
		{\cellcolor{gray!25}3} &
		{ 11} &
		{  4} &
		{  6} &
		{  5}\\\hline
	\end{tabular}
	\end{center}

	\begin{center}
	\begin{tabular}{|*{14}{>{\centering\arraybackslash}m{0.75cm}|}}
		\hline
		{10} &
		{  5} &
		{ 12} &
		{ 15} &
		{  4} &
		{  8} &
		{  1} &
		{  7} &
		{\cellcolor{gray!25}3} &
		{ 12} &
		{ 11} &
		{  4} &
		{  6} &
		{  5}\\\hline
	\end{tabular}
	\end{center}
	
	\begin{center}
	\begin{tabular}{|*{14}{>{\centering\arraybackslash}m{0.75cm}|}}
		\hline
		{10} &
		{  5} &
		{ 12} &
		{ 15} &
		{  4} &
		{  8} &
		{  1} &
		{\cellcolor{gray!25}3} &
		{  7} &
		{ 12} &
		{ 11} &
		{  4} &
		{  6} &
		{  5}\\\hline
	\end{tabular}
	\end{center}

	\begin{center}
	\begin{tabular}{|*{14}{>{\centering\arraybackslash}m{0.75cm}|}}
		\hline
		{10} &
		{  5} &
		{ 12} &
		{ 15} &
		{  4} &
		{  8} &
		{\cellcolor{gray!25}1} &
		{  3} &
		{  7} &
		{ 12} &
		{ 11} &
		{  4} &
		{  6} &
		{  5}\\\hline
	\end{tabular}
	\end{center}

	\begin{center}
	\tablehead{}
	\begin{tabular}{|*{14}{>{\centering\arraybackslash}m{0.75cm}|}}
		\hline
		{10} &
		{  5} &
		{ 12} &
		{ 15} &
		{  4} &
		{\cellcolor{gray!25}1} &
		{  8} &
		{  3} &
		{  7} &
		{ 12} &
		{ 11} &
		{  4} &
		{  6} &
		{  5}\\\hline
	\end{tabular}
	\end{center}

	\begin{center}
	\begin{tabular}{|*{14}{>{\centering\arraybackslash}m{0.75cm}|}}
		\hline
		{10} &
		{  5} &
		{ 12} &
		{ 15} &
		{\cellcolor{gray!25}1} &
		{  4} &
		{  8} &
		{  3} &
		{  7} &
		{ 12} &
		{ 11} &
		{  4} &
		{  6} &
		{  5}\\\hline
	\end{tabular}
	\end{center}

	\begin{center}
	\begin{tabular}{|*{14}{>{\centering\arraybackslash}m{0.75cm}|}}
		\hline
		{10} &
		{  5} &
		{ 12} &
		{\cellcolor{gray!25}1} &
		{ 15} &
		{  4} &
		{  8} &
		{  3} &
		{  7} &
		{ 12} &
		{ 11} &
		{  4} &
		{  6} &
		{  5}\\\hline
	\end{tabular}
	\end{center}
	
	\begin{center}
	\begin{tabular}{|*{14}{>{\centering\arraybackslash}m{0.75cm}|}}
		\hline
		{10} &
		{  5} &
		{\cellcolor{gray!25}1} &
		{ 12} &
		{ 15} &
		{  4} &
		{  8} &
		{  3} &
		{  7} &
		{ 12} &
		{ 11} &
		{  4} &
		{  6} &
		{  5}\\\hline
	\end{tabular}
	\end{center}
	
	\begin{center}
	\begin{tabular}{|*{14}{>{\centering\arraybackslash}m{0.75cm}|}}
		\hline
		{10} &
		{\cellcolor{gray!25}1} &
		{  5} &
		{ 12} &
		{ 15} &
		{  4} &
		{  8} &
		{  3} &
		{  7} &
		{ 12} &
		{ 11} &
		{  4} &
		{  6} &
		{  5}\\\hline
	\end{tabular}
	\end{center}

	\begin{center}
	\begin{tabular}{|*{14}{>{\centering\arraybackslash}m{0.75cm}|}}
		\hline
		\cellcolor{gray!25}1 & 
		10 & 
		5 & 
		12 & 
		15 &
		4 &
		8 &
		3 &
		7 &
		12 &
		11 &
		4 &
		6 &
		5
		\\\hline
	\end{tabular}
	\end{center}

	Ceci achève le premier parcours. On recommence à présent le même
	processus dans le sous-tableau débutant au deuxième élément, ce qui
	donnera le contenu suivant~:

	\begin{center}
	\begin{tabular}{|*{14}{>{\centering\arraybackslash}m{0.75cm}|}}
		\hline
		1 & 3 & 10 & 5 & 12 & 15 & 4 & 8 & 4 & 7 & 12 & 11 & 5 & 6
		\\\hline
	\end{tabular}
	\end{center}

	Comme pour le tri par recherche des minima successifs, à l’issue de
	l’étape $i$ de l’algorithme, on obtient un sous-tableau trié
	(entre les indices 1 et \textit{i }) et un tableau non encore trié
	(entre les indices $i + 1$ et $n$). Le tri est donc
	achevé à l’issue de l’étape $n - 1$. Cependant, on notera que
	petit à petit, outre le déplacement de la bulle, les éléments plus
	petits évoluent progressivement vers la gauche, et les plus gros vers
	la droite, de sorte qu’il est fréquent que le nombre d’étapes effectif
	est inférieur au nombre d’étapes théorique.

	{\sffamily\bfseries\upshape
	Algorithme}

	Dans cet algorithme, la variable \textstyleCodeInsr{indiceBulle} est à
	la fois le compteur d’étapes et aussi l’indice de l’élément récepteur
	de la bulle à la fin d’une étape donnée.

	\bigskip
	
	\cadre{
	\begin{pseudo}
	\LComment Trie le tableau reçu en paramètre (via un tri bulle).
		\Module{triBulle}{tab\InOut~: tableau[1 à n] d’entiers}{}
			\Decl indiceBulle, i~: entiers
			\For{indiceBulle \K{de} 1 \K{à} n}
				\For{i \K{de} n – 1 \K{à} indiceBulle \K{par} – 1}
					\If{tab[i] > tab[i + 1]}
						\Stmt swap( tab[i], tab[i + 1] )
						\RComment voir algorithme précédent
					\EndIf
				\EndFor
			\EndFor
		\EndModule
	\end{pseudo}
	}

	\bigskip

%=========================
\section{Cas particuliers}
%==========================

	{\sffamily\bfseries\upshape
	Tri partiel}

		Parfois, on n’est pas intéressé par un tri complet de l’ensemble mais on
		veut uniquement les «~$k$~» plus petites (ou plus grandes) valeurs. Le
		tri par recherche des minima et le tri bulle sont particulièrement
		adaptés à cette situation ; il suffit de les arrêter plus tôt. Par
		contre, le tri par insertion est inefficace car il ne permet pas un
		arrêt anticipé.

%--------------------------------------------
% @mcd : le chef demande si on peut virer 
% le paragraphe qui suit ?
% --------------------------------------------
	{\sffamily\bfseries\upshape
	Ensemble (presque) trié}

		Les trois tris que nous avons vus ont la même complexité quadratique
		(voir plus loin) dans le cas général mais comment se comportent-ils
		dans le cas d’un tableau (presque) trié ? Dans ce cas, on peut vérifier
		que le tri par recherche des minima successifs reste quadratique alors
		que les deux autres deviennent linéaires (en $n$).

	{\sffamily\bfseries\upshape
	Sélection des meilleurs}

		Une autre situation courante est la suivante~: un ensemble de valeurs
		sont traitées (lues, calculées, ...) et il ne faut garder que les
		$k$ plus petites (ou plus grandes).
		L'algorithme est plus simple à écrire si on utilise
		une position supplémentaire en fin de tableau. Notons aussi qu’il faut
		tenir compte du cas où il y a moins de valeurs que le nombre de valeurs
		voulues ; c’est pourquoi on ajoute une variable indiquant le nombre
		exact de valeurs dans le tableau.

	{\sffamily\bfseries\scshape
	Exemple~: les $k$ plus petites valeurs}

		On lit un ensemble de valeurs strictement positives (un 0 indique la fin
		de la lecture) et on ne garde que les $k$ plus petites valeurs.

		\bigskip
		
		\cadre{
		\begin{pseudo}
			\Module{meilleurs}{plusPetits\InOut~: tableau[1 à k + 1] d’entiers, nbValeurs\Out~: entier}{}
				\Decl val~: entier
				\Let nbValeurs \Gets 0
				\Read val
				\While{val ${\neq}$ 0}
					\Stmt insérer( val, plusPetits, nbValeurs )
					\Read val
				\EndWhile
			\EndModule
			
			\Empty
			
			\Module{insérer}{val\In~: entier, plusPetits\InOut~: tableau[1 à k + 1] d’entiers, nbValeurs\InOut~: entier}{}
				\Decl i~: entiers
				\Let i \Gets nbValeurs
				\Read val
				\While{i > 0 ET val < plusPetits[i]}
					\Let plusPetits[i + 1] \Gets plusPetits[i]
					\Let i \Gets i - 1
				\EndWhile
				\Let plusPetits[i + 1] \Gets val
				\If{nbValeurs < k}
					\Let nbValeurs \Gets nbValeurs + 1
				\EndIf
			\EndModule
		\end{pseudo}
		}
		

%===============================
\section{Recherche dichotomique}
%================================

	La recherche dichotomique a pour essence de réduire à
	chaque étape la taille de l’ensemble de recherche de moitié, jusqu’à ce
	qu’il ne reste qu’un seul élément dont la valeur devrait être celle
	recherchée, sauf bien entendu en cas d’inexistence de cette valeur dans
	l’ensemble de départ. Nous l’expliquons pour un tableau ordonné mais
	elle peut s’appliquer aussi avec des changements mineurs pour
	d'autres structures (par exemple la liste ordonnée que
	nous étudierons plus tard).

	{\sffamily\bfseries\upshape
	Description de l’algorithme}

	Soit \textstyleCodeInsr{val} la valeur recherchée dans une zone
	délimitée par les indices \textstyleCodeInsr{indiceGauche} et
	\textstyleCodeInsr{indiceDroit}. On commence par déterminer l’élément
	médian, c’est-à-dire celui qui se trouve « au milieu » de la zone de
	recherche ; son indice sera déterminé par la formule

	{\centering
	\textstyleCodeInsr{{indiceMédian}}{
	}\textsf{←}{
	}\textstyleCodeInsr{{(indiceGauche +
	indiceDroit)
	}}\textstyleCodeInsr{{DIV}}\textstyleCodeInsr{{
	2}}
	\par}

	N.B.~: cet élément médian n'est pas tout à fait au milieu dans 
	le cas d'une zone contenant un nombre pair d'éléments.
	On compare alors \textstyleCodeInsr{val}
	avec la valeur de cet élément médian ; il est possible qu’on ait trouvé
	la valeur cherchée; sinon, on partage la zone de recherche en deux
	parties~: une qui \textbf{ne contient certainement pas}
	la valeur cherchée et une qui \textbf{pourrait la
	contenir}. C’est cette deuxième partie qui 
	devient la nouvelle zone de recherche. On
	réitère le processus jusqu’à ce que la valeur cherchée soit trouvée ou
	que la zone de recherche soit réduite à sa plus simple expression,
	c’est-à-dire un seul élément.

	{\sffamily\bfseries
	Algorithme}
	
	\cadre{
		\begin{pseudo}
			\Module{rechercheDichotomique}{tab\InOut~: tableau[1 à n] de T, valeur\In~: T, pos\Out~: entier}{booléen}
				\Decl indiceDroit, indiceGauche, indiceMédian~: entiers
				\Decl candidat~: T
				\Decl trouvé~: booléen
				\Empty
				\Let indiceGauche \Gets 1
				\Let indiceDroit \Gets n
				\Let trouvé \Gets faux
				\Empty
				\While{NON trouvé ET indiceGauche {${\leq}$} indiceDroit}
					\Let indiceMédian \Gets (indiceGauche + indiceDroit) DIV 2
					\Let candidat \Gets tab[indiceMédian]
					\Switch{}
						\Case{candidat = valeur}
							\Let trouvé \Gets vrai
						\Case{candidat < valeur}
							\Let indiceGauche \Gets indiceMédian + 1
							\RComment on garde la partie droite
						\Case{candidat > valeur}
							\Let indiceDroit \Gets indiceMédian – 1
							\RComment on garde la partie gauche
					\EndSwitch
				\EndWhile
				\If{trouvé}
					\Let pos \Gets indiceMédian
				\Else
					\Let pos \Gets indiceGauche
					\RComment dans le cas où la valeur n’est pas trouvée,
					\Empty 
					\RComment on vérifiera que indiceGauche donne la valeur où elle pourrait être insérée.
				\EndIf
				\Return trouvé
			\EndModule
		\end{pseudo}
		}
		
		{\sffamily\bfseries\upshape
		Exemple de recherche fructueuse}

		Supposons que l’on recherche la valeur
		\textbf{23} dans un
		tableau de 20 entiers. La zone ombrée représente à chaque fois la
		partie de recherche, qui est au départ le tableau dans son entièreté.
		Au départ,
		\textstyleCodeInsr{indiceGauche} vaut 1 et
		\textstyleCodeInsr{indiceDroit}	vaut 20.
			
		\begin{center}
		\begin{tabular}{*{20}{>{\centering\sffamily\itshape\arraybackslash}m{0.47cm}}}
			 1 &
			 2 &
			 3 &
			 4 &
			 5 &
			 6 &
			 7 &
			 8 &
			 9 &
			 10 &
			 11 &
			 12 &
			 13 &
			 14 &
			 15 & 
			 16 &
			 17 &
			 18 &
			 19 &
			 20
			 \\
		\end{tabular}
		\begin{tabular}{|*{20}{>{\centering\arraybackslash}m{0.46cm}|}}
			\hline
			{\cellcolor{gray!25} 1} &
			{\cellcolor{gray!25}  3} &
			{\cellcolor{gray!25}  5} &
			{\cellcolor{gray!25}  7} &
			{\cellcolor{gray!25}  7} &
			{\cellcolor{gray!25}  9} &
			{\cellcolor{gray!25}  9} &
			{\cellcolor{gray!25} 10} &
			{\cellcolor{gray!25} 10} &
			{\cellcolor{gray!25} 15} &
			{\cellcolor{gray!25} 16} &
			{\cellcolor{gray!25} 20} &
			{\cellcolor{gray!25} 23} &
			{\cellcolor{gray!25} 23} &
			{\cellcolor{gray!25} 25} &
			{\cellcolor{gray!25} 28} &
			{\cellcolor{gray!25} 28} &
			{\cellcolor{gray!25} 28} &
			{\cellcolor{gray!25} 29} &
			{\cellcolor{gray!25} 29}\\\hline
		\end{tabular}
		\end{center}

		\bigskip

		\textit{Première étape}~:
		\textstyleCodeInsr{indiceMédian} \textsf{←} 
		\textstyleCodeInsr{(1 + 20) DIV	2}, c’est-à-dire 10. Comme la valeur en
		position 10 est 15, la valeur cherchée doit se trouver à sa droite, et
		on prend comme nouvelle zone de recherche celle délimitée par
		\textstyleCodeInsr{indiceGauche} qui vaut 11 et
		\textstyleCodeInsr{indiceDroit} qui vaut 20.
		
		\bigskip
		
		\begin{center}
		\begin{tabular}{*{20}{>{\centering\sffamily\itshape\arraybackslash}m{0.47cm}}}
			 1 &
			 2 &
			 3 &
			 4 &
			 5 &
			 6 &
			 7 &
			 8 &
			 9 &
			 10 &
			 11 &
			 12 &
			 13 &
			 14 &
			 15 & 
			 16 &
			 17 &
			 18 &
			 19 &
			 20
			 \\
		\end{tabular}
		\begin{tabular}{|*{20}{>{\centering\arraybackslash}m{0.46cm}|}}
			\hline
			\multicolumn{1}{|m{0.49700004cm}|}{ 1} &
			{  3} &
			{  5} &
			{  7} &
			{  7} &
			{  9} &
			{  9} &
			{ 10} &
			{ 10} &
			{ 15} &
			{\cellcolor{gray!25} 16} &
			{\cellcolor{gray!25} 20} &
			{\cellcolor{gray!25} 23} &
			{\cellcolor{gray!25} 23} &
			{\cellcolor{gray!25} 25} &
			{\cellcolor{gray!25} 28} &
			{\cellcolor{gray!25} 28} &
			{\cellcolor{gray!25} 28} &
			{\cellcolor{gray!25} 29} &
			{\cellcolor{gray!25} 29}\\\hline
		\end{tabular}
		\end{center}

		\bigskip

		\textit{Deuxième étape}~:
		\textstyleCodeInsr{indiceMédian} \textsf{←} 
		\textstyleCodeInsr{(11 + 20) DIV 2}, c’est-à-dire 15. Comme on y trouve la
		valeur 25, on garde les éléments situées à la gauche de celui-ci ; la
		nouvelle zone de recherche est [11, 14].
		
		
		\begin{center}
		\begin{tabular}{*{20}{>{\centering\sffamily\itshape\arraybackslash}m{0.47cm}}}
			 1 &
			 2 &
			 3 &
			 4 &
			 5 &
			 6 &
			 7 &
			 8 &
			 9 &
			 10 &
			 11 &
			 12 &
			 13 &
			 14 &
			 15 & 
			 16 &
			 17 &
			 18 &
			 19 &
			 20
			 \\
		\end{tabular}
		\begin{tabular}{|*{20}{>{\centering\arraybackslash}m{0.46cm}|}}
			\hline
			\multicolumn{1}{|m{0.49700004cm}|}{ 1} &
			{  3} &
			{  5} &
			{  7} &
			{  7} &
			{  9} &
			{  9} &
			{ 10} &
			{ 10} &
			{ 15} &
			{\cellcolor{gray!25} 16} &
			{\cellcolor{gray!25} 20} &
			{\cellcolor{gray!25} 23} &
			{\cellcolor{gray!25} 23} &
			{ 25} &
			{ 28} &
			{ 28} &
			{ 28} &
			{ 29} &
			{ 29}\\\hline
		\end{tabular}
		\end{center}

		\bigskip

		\textit{Troisième étape}~:
		\textstyleCodeInsr{indiceMédian} \textsf{←} 
		\textstyleCodeInsr{(11 + 14) DIV 2}, c’est-à-dire 12 où se trouve l’élément
		20. La zone de recherche devient
		\textstyleCodeInsr{indiceGauche} vaut 13 et
		\textstyleCodeInsr{indiceDroit} vaut 14.

		
		\begin{center}
		\begin{tabular}{*{20}{>{\centering\sffamily\itshape\arraybackslash}m{0.47cm}}}
			 1 &
			 2 &
			 3 &
			 4 &
			 5 &
			 6 &
			 7 &
			 8 &
			 9 &
			 10 &
			 11 &
			 12 &
			 13 &
			 14 &
			 15 & 
			 16 &
			 17 &
			 18 &
			 19 &
			 20
			 \\
		\end{tabular}
		\begin{tabular}{|*{20}{>{\centering\arraybackslash}m{0.46cm}|}}
			\hline
			\multicolumn{1}{|m{0.49700004cm}|}{ 1} &
			{  3} &
			{  5} &
			{  7} &
			{  7} &
			{  9} &
			{  9} &
			{ 10} &
			{ 10} &
			{ 15} &
			{ 16} &
			{ 20} &
			{\cellcolor{gray!25} 23} &
			{\cellcolor{gray!25} 23} &
			{ 25} &
			{ 28} &
			{ 28} &
			{ 28} &
			{ 29} &
			{ 29}\\\hline
		\end{tabular}
		\end{center}

		\bigskip


		\textit{Quatrième étape~:}
		\textstyleCodeInsr{indiceMédian} \textsf{←} 
		\textstyleCodeInsr{(13 + 14) DIV 2}, c’est-à-dire 13 où se trouve la valeur
		cherchée ; la recherche est terminée.

	{\sffamily\bfseries\upshape
	Exemple de recherche infructueuse}

		Recherchons à présent la valeur \textbf{8}. Les étapes de la recherche 
		vont donner successivement
		
		
		\begin{center}
		\begin{tabular}{*{20}{>{\centering\sffamily\itshape\arraybackslash}m{0.47cm}}}
			 1 &
			 2 &
			 3 &
			 4 &
			 5 &
			 6 &
			 7 &
			 8 &
			 9 &
			 10 &
			 11 &
			 12 &
			 13 &
			 14 &
			 15 & 
			 16 &
			 17 &
			 18 &
			 19 &
			 20
			 \\
		\end{tabular}
		\begin{tabular}{|*{20}{>{\centering\arraybackslash}m{0.46cm}|}}
			\hline
			\multicolumn{1}{|m{0.49700004cm}|}{\cellcolor{gray!25} 1} &
			{\cellcolor{gray!25}  3} &
			{\cellcolor{gray!25}  5} &
			{\cellcolor{gray!25}  7} &
			{\cellcolor{gray!25}  7} &
			{\cellcolor{gray!25}  9} &
			{\cellcolor{gray!25}  9} &
			{\cellcolor{gray!25} 10} &
			{\cellcolor{gray!25} 10} &
			{\cellcolor{gray!25} 15} &
			{\cellcolor{gray!25} 16} &
			{\cellcolor{gray!25} 20} &
			{\cellcolor{gray!25} 23} &
			{\cellcolor{gray!25} 23} &
			{\cellcolor{gray!25} 25} &
			{\cellcolor{gray!25} 28} &
			{\cellcolor{gray!25} 28} &
			{\cellcolor{gray!25} 28} &
			{\cellcolor{gray!25} 29} &
			{\cellcolor{gray!25} 29}\\\hline
		\end{tabular}
		\end{center}

		\bigskip
		
		\begin{center}
		\begin{tabular}{*{20}{>{\centering\sffamily\itshape\arraybackslash}m{0.47cm}}}
			 1 &
			 2 &
			 3 &
			 4 &
			 5 &
			 6 &
			 7 &
			 8 &
			 9 &
			 10 &
			 11 &
			 12 &
			 13 &
			 14 &
			 15 & 
			 16 &
			 17 &
			 18 &
			 19 &
			 20
			 \\
		\end{tabular}
		\begin{tabular}{|*{20}{>{\centering\arraybackslash}m{0.46cm}|}}
			\hline
			\multicolumn{1}{|m{0.49700004cm}|}{\cellcolor{gray!25} 1} &
			{\cellcolor{gray!25}  3} &
			{\cellcolor{gray!25}  5} &
			{\cellcolor{gray!25}  7} &
			{\cellcolor{gray!25}  7} &
			{\cellcolor{gray!25}  9} &
			{\cellcolor{gray!25}  9} &
			{\cellcolor{gray!25} 10} &
			{\cellcolor{gray!25} 10} &
			{ 15} &
			{ 16} &
			{ 20} &
			{ 23} &
			{ 23} &
			{ 25} &
			{ 28} &
			{ 28} &
			{ 28} &
			{ 29} &
			{ 29}\\\hline
		\end{tabular}
		\end{center}

		\bigskip
		
		\begin{center}
		\begin{tabular}{*{20}{>{\centering\sffamily\itshape\arraybackslash}m{0.47cm}}}
			 1 &
			 2 &
			 3 &
			 4 &
			 5 &
			 6 &
			 7 &
			 8 &
			 9 &
			 10 &
			 11 &
			 12 &
			 13 &
			 14 &
			 15 & 
			 16 &
			 17 &
			 18 &
			 19 &
			 20
			 \\
		\end{tabular}
		\begin{tabular}{|*{20}{>{\centering\arraybackslash}m{0.46cm}|}}
			\hline
			\multicolumn{1}{|m{0.49700004cm}|}{ 1} &
			{  3} &
			{  5} &
			{  7} &
			{  7} &
			{\cellcolor{gray!25}  9} &
			{\cellcolor{gray!25}  9} &
			{\cellcolor{gray!25} 10} &
			{\cellcolor{gray!25} 10} &
			{ 15} &
			{ 16} &
			{ 20} &
			{ 23} &
			{ 23} &
			{ 25} &
			{ 28} &
			{ 28} &
			{ 28} &
			{ 29} &
			{ 29}\\\hline
		\end{tabular}
		\end{center}

		\bigskip
		
		\begin{center}
		\begin{tabular}{*{20}{>{\centering\sffamily\itshape\arraybackslash}m{0.47cm}}}
			 1 &
			 2 &
			 3 &
			 4 &
			 5 &
			 6 &
			 7 &
			 8 &
			 9 &
			 10 &
			 11 &
			 12 &
			 13 &
			 14 &
			 15 & 
			 16 &
			 17 &
			 18 &
			 19 &
			 20
			 \\
		\end{tabular}
		\begin{tabular}{|*{20}{>{\centering\arraybackslash}m{0.46cm}|}}
			\hline
			\multicolumn{1}{|m{0.49700004cm}|}{ 1} &
			{  3} &
			{  5} &
			{  7} &
			{  7} &
			{\cellcolor{gray!25}  9} &
			{  9} &
			{ 10} &
			{ 10} &
			{ 15} &
			{ 16} &
			{ 20} &
			{ 23} &
			{ 23} &
			{ 25} &
			{ 28} &
			{ 28} &
			{ 28} &
			{ 29} &
			{ 29}\\\hline
		\end{tabular}
		\end{center}

		\bigskip

		Arrivé à ce stade, la zone de recherche s’est réduite à un seul élément.
		Ce n’est pas celui qu’on recherche mais c’est à cet endroit qu’il se
		serait trouvé ; c’est donc là qu’on pourra éventuellement
		l'insérer. Le paramètre de sortie prend la valeur 6.


%=====================================
\section{Introduction à la complexité}
%======================================

	L’algorithme de recherche dichotomique est beaucoup plus rapide que
	l’algorithme de recherche linéaire. Mais qu’est-ce que cela veut dire
	exactement ? Comment le mesure-t-on ? Quels critères utilise-t-on ?

	On pourrait se dire que pour comparer deux algorithmes, il suffit de les
	traduire tous les deux dans un langage de programmation, de les
	exécuter et de comparer leur temps d’exécution. Cela pose toutefois
	quelques problèmes
			
	\liststyleListv
	\begin{liste}
		\item 
			Il faut que les programmes soient exécutés dans des environnements
			strictement identiques ce qui n’est pas toujours le cas ou facile 
			à vérifier.
		\item 
			Le test ne porte que sur un (voir quelques uns) jeu(x) de tests. Comment
			en tirer un enseignement général ? Que se passerait-il avec des données
			plus importantes ? Avec des données différentes ?
	\end{liste}
	
	En fait, un chiffre précis ne nous intéresse pas. Ce qui est intéressant
	à connaitre, c’est l’évolution de la rapidité de l’algorithme (plus
	précisément, le nombre d’opérations de base) en fonction de la
	\textbf{taille} du problème à résoudre. Comment va-t-il se comporter
	sur de «~gros~» problèmes ?

	Dans le cas de la recherche dans un tableau, 
	la taille du problème est la taille du
	tableau dans lequel on recherche un élément. On peut considérer que
	l’opération de base est la comparaison avec un élément du tableau. La
	question est alors la suivante~: pour un tableau de taille $n$, à combien
	de comparaisons faut-il procéder pour trouver l’élément (ou se rendre
	compte de son absence) ?

	{\bfseries
	Pour la recherche linéaire}

		Cela dépend évidemment de la position de la valeur à trouver. Dans le
		meilleur des cas c’est 1, dans le pire c’est $n$ mais on peut dire, qu’en
		moyenne, cela entraine «~$n/2$~» comparaisons (que ce soit pour la
		trouver ou se rendre compte de son absence).

	{\bfseries
	Pour la recherche dichotomique}

		Ici, la zone de recherche est divisée par 2, à chaque étape. Imaginons
		une liste de 64 éléments~: après 6 étapes, on arrive à un élément
		isolé. Pour une liste de taille $n$, on peut en déduire que le nombre de
		comparaisons est au maximum l'exposant
		qu'il faut donner à 2 pour obtenir $n$, soit
		«~$\log_2(n)$~».

	{\sffamily\bfseries\upshape
	Comparaisons}

	
	Voici un tableau comparatif du nombre de comparaison en fonction de la
	taille $n$

	\begin{center}
	\tablehead{}
	\begin{supertabular}{|m{4.4300003cm}|m{0.507cm}|m{0.71900004cm}|m{0.93100005cm}|m{1.2479999cm}|m{1.4599999cm}|m{1.671cm}|}
	\hline
	\raggedright \bfseries $n$ &
	\raggedleft \bfseries 10 &
	\raggedleft \bfseries 100 &
	\raggedleft \bfseries 1000 &
	\raggedleft \bfseries 10.000 &
	\raggedleft \bfseries 100.000 &
	\raggedleft\arraybslash \bfseries 1
	million\\\hline
	\raggedright \bfseries recherche linéaire &
	\raggedleft  5 &
	\raggedleft  50 &
	\raggedleft  500 &
	\raggedleft  5.000 &
	\raggedleft  50.000 &
	\raggedleft\arraybslash  500.000\\\hline
	\raggedright \bfseries recherche dichotomique &
	\raggedleft  4 &
	\raggedleft  7 &
	\raggedleft  10 &
	\raggedleft  14 &
	\raggedleft  17 &
	\raggedleft\arraybslash  20\\\hline
	\end{supertabular}
	\end{center}
	
	On voit que c’est surtout pour des grands problèmes que la recherche
	dichotomique montre toute son efficacité. On voit aussi que ce qui est
	important pour mesurer la complexité c'est
	l'ordre de grandeur du nombre de comparaisons. 

	
	On dira que la recherche simple est un algorithme de complexité linéaire
	(c'est-à-dire que le nombre
	d'opérations est de l'ordre de $n$ ou
	proportionnel à $n$) ce qu’on note en langage plus mathématique $O(n)$
	(prononcé «~grand $O$ de $n$~»). Pour la recherche dichotomique, la
	complexité est logarithmique, et on le note $O(\log_2(n))$.

	Comparons les complexités les plus courantes.

	\begin{center}
	\begin{tabular}{|m{3.1cm}|m{1.0cm}|m{1.3cm}|m{1.074cm}|m{1.3cm}|m{1.287cm}|m{1.425cm}|m{1.714cm}|}
	\hline
	\centering \bfseries $n$ &
	\raggedleft \bfseries 10 &
	\raggedleft \bfseries 100 &
	\raggedleft \bfseries 1000 &
	\raggedleft \bfseries 10\textsuperscript{4} &
	\raggedleft \bfseries 10\textsuperscript{5} &
	\raggedleft \bfseries 10\textsuperscript{6} &
	\raggedleft\arraybslash \bfseries
	10\textsuperscript{9}\\\hline
	\centering  $O(1)$ &
	\raggedleft  1 &
	\raggedleft  1 &
	\raggedleft  1 &
	\raggedleft  1 &
	\raggedleft  1 &
	\raggedleft  1 &
	\raggedleft\arraybslash  1\\\hline
	\centering  $O(\log_2(n))$ &
	\raggedleft  4 &
	\raggedleft  7 &
	\raggedleft  10 &
	\raggedleft  14 &
	\raggedleft  17 &
	\raggedleft  20 &
	\raggedleft\arraybslash  30\\\hline
	\centering  $O(n)$ &
	\raggedleft  10 &
	\raggedleft  100 &
	\raggedleft  1000 &
	\raggedleft  10.000 &
	\raggedleft  10\textsuperscript{5} &
	\raggedleft  10\textsuperscript{6} &
	\raggedleft\arraybslash 
	10\textsuperscript{9}\\\hline
	\centering  $O(n^2)$ &
	\raggedleft  100 &
	\raggedleft  10.000 &
	\raggedleft  10\textsuperscript{6} &
	\raggedleft  10\textsuperscript{8} &
	\raggedleft  10\textsuperscript{10} &
	\raggedleft  10\textsuperscript{12} &
	\raggedleft\arraybslash 
	10\textsuperscript{18}\\\hline
	\centering  $O(n^3)$ &
	\raggedleft  1000 &
	\raggedleft  10\textsuperscript{6} &
	\raggedleft  10\textsuperscript{9} &
	\raggedleft  10\textsuperscript{12} &
	\raggedleft  10\textsuperscript{15} &
	\raggedleft  10\textsuperscript{18} &
	\raggedleft\arraybslash 
	10\textsuperscript{27}\\\hline
	\centering  $O(2^n)$ &
	\raggedleft  1024 &
	\raggedleft  10\textsuperscript{30} &
	\raggedleft  10\textsuperscript{301} &
	\raggedleft  10\textsuperscript{3010} &
	\raggedleft  10\textsuperscript{30102} &
	\raggedleft  10\textsuperscript{301029} &
	\raggedleft\arraybslash 
	10\textsuperscript{301029995}\\\hline
	\end{tabular}
	\end{center}
	
	On voit ainsi que si on trouve un algorithme correct pour résoudre un
	problème mais que cet algorithme est de complexité exponentielle alors
	cet algorithme ne sert à rien en pratique. Par exemple un algorithme de
	recherche de complexité exponentielle, exécuté sur une machine pouvant
	effectuer un milliard de comparaisons par secondes, prendrait plus de
	dix mille milliards d’années pour trouver une valeur dans un tableau de
	100 éléments.

%====================
\section{	Exercices}
%=====================

	\begin{Exercice}{Sélection des maxima}
		Développez un algorithme similaire consistant à trier un tableau par
		ordre croissant par sélection des \textbf{maxima} successifs. Le
		sous-tableau trié apparaitra donc à droite du tableau, et les maxima
		sélectionnés seront à chaque étape positionnés à droite du sous-tableau
		non trié. 
	\end{Exercice}
	
	\begin{Exercice}{Maxima et minima}
		Développez un algorithme combinant les deux recherches. À chaque étape,
		on sélectionne donc le minimum et le maximum du sous-tableau restant à
		trier et on les positionnera à l’endroit \textit{ad hoc}. 
		Cette méthode apporte-t-elle une amélioration en temps ou
		en simplicité aux deux algorithmes de base ?
	\end{Exercice}

	\begin{Exercice}{Amélioration 1}
			Écrire une amélioration du tri bulle consistant à mémoriser à chaque
			étape l’indice de la dernière permutation, celui-ci délimitant en fait
			la véritable taille du sous-tableau trié à l’issue d’une étape. En lieu
			et place de la boucle \textstyleMotCl{pour} n’incrémentant l’indice
			bulle que de 1 à la fois, on écrira une boucle \textstyleMotCl{tant
			que} à l’issue de laquelle \textstyleCodeInsr{indiceBulle} prendra la
			valeur de l’indice de dernière permutation + 1.
	\end{Exercice}
	
	\begin{Exercice}{Amélioration 2}
			L’amélioration précédente est issue de l’observation du sous-tableau
			déjà trié en début du tableau initial. On peut de même étudier la
			possibilité d’avoir un sous-tableau, trié également, mais cadré à
			droite dans le tableau à trier. Cette symétrie suggère une amélioration
			supplémentaire qui consiste à changer de sens à la fin d’un parcours
			pour entamer le parcours suivant. Lorsqu’on change de sens, on amènera
			l’élément le plus grand (qu’on peut nommer le «~plomb~») au fond du
			tableau non trié. L’association des deux méthodes donne ce qu’on
			appelle le \textbf{tri shaker}, dont le but est de restreindre le
			sous-tableau non trié en augmentant sa borne inférieure et en diminuant
			sa borne supérieure. Écrire l’algorithme qui réalise cette méthode de
			tri.
	\end{Exercice}
	
	\begin{Exercice}{Tri de dates}
			Écrire un algorithme qui trie un tableau de 
			dates par ordre chronologique croissant.
			Les dates sont des structures Date (cfr. chap. 6)
	\end{Exercice}
	
	\begin{Exercice}{Tri de personnes}
			Écrire un algorithme qui trie un tableau de personnes sur leur date
			d'anniversaire (sans tenir compte de
			l'année). Deux personnes nées le même jour seront
			départagées par leur nom.

			Une personne est représentée par la structure 
			
			
			\cadre{
			\begin{pseudo}
				\Struct{Personne}
					\Decl nom~: chaine
					\Decl dateNaissance~: Date
				\EndStruct
			\end{pseudo}
			}
			
	\end{Exercice}
	
	\begin{Exercice}{tableau de Point}
		Écrire un algorithme qui reçoit un tableau de structures Point 
		et trie celui-ci sur l'ordre d'éloignement de ces points par 
		rapport à l'origine du plan. Les premiers éléments seront donc
		ceux les plus proches du point (0, 0) et les derniers seront 
		les points les plus éloignés.
	\end{Exercice}

	\begin{Exercice}{Calcul de complexités}
			
			\begin{enumerate}[label=\alph*)]
				\item 
					Quelle est la complexité d'un algorithme qui recherche
					le maximum d'un tableau de $n$ éléments ?
				\item 
					Quelle est la complexité d'un algorithme qui remplace
					par 0 toutes les occurrences du maximum d'un tableau
					de $n$ éléments ?
				\item 
					Quelle est la complexité d'un algorithme qui vérifie si
					un tableau contient deux éléments égaux ?
				\item 
					Quelle est la complexité de l'algorithme de tri par
					recherche des minima successifs ?
				\item 
					Quelle est la complexité d'un algorithme qui vérifie si
					les éléments d'un tableau forment un palindrome ?
			\end{enumerate}
		\end{Exercice}
		
		\begin{Exercice}{Réflexion}
				L’algorithme de recherche dichotomique est-il toujours à préférer à
				l’algorithme de recherche linéaire ?
		\end{Exercice}


%====================
\section{Références}
%====================

	\liststyleListv
	\begin{liste}
		\item {
			\url{http://interstices.info/jcms/c_6973/les-algorithmes-de-tri}

			\textit{Ce site permet de visualiser les méthodes de tris en
			fonctionnement. Il présente tous les tris vus en première plus quelques
			autres comme le «~tri rapide~» («~quicksort~») que vous verrez en
			deuxième année. }}
	\end{liste}
