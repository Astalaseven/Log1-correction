\chapter*{Correction des exercices 7.4}

\vspace{-2cm}

\begin{Emphase}{Exercice 3 -- Afficher les $n$ premiers}
\cadre{
\begin{pseudo}

\Module{strictementPositifs}{}{}

    \Decl n, i~:~entiers
    \Read n

    \For{i \K{de} 0 \K{à} n}
        \Write i
    \EndFor

\EndModule
\\
\Module{strictementPositifsDécroissants}{}{}

    \Decl n, i~:~entiers
    \Read n

    \For{i \K{de} n \K{à} 0 \K{par} -1}
        \Write i
    \EndFor

\EndModule
\\ 
\Module{carrésParfaits}{}{}

    \Decl n, i~:~entiers
    \Read n

    \For{i \K{de} 0 \K{à} n}
        \Write i$^2$
    \EndFor

\EndModule
\\
\Module{naturelsImpairs}{}{}

    \Decl n, i~:~entiers
    \Read n

    \For{i \K{de} 1 \K{à} n * 2 \K{par} 2}
        \Write i
    \EndFor

\EndModule
\\
\Module{naturelsImpairsInférieurs}{}{}

    \Decl n, i~:~entiers
    \Read n

    \For{i \K{de} 1 \K{à} n/2 \K{par} 2}
        \Write i
    \EndFor

\end{pseudo}
}
\end{Emphase}

\newpage

\begin{Emphase}{Exercice 4 -- Maximum de nombres}
\cadre{
\begin{pseudo}
\Module{maxCote}{}{}

    \Decl cote, max~:~entier
    \Let max \Gets 0
    \Repeat
        \Read cote
        \If{cote > max}
                \Let max \Gets cote
        \EndIf
    \Until{cote = -1}
    \Write max

\EndModule
\end{pseudo}
}
\end{Emphase}


\begin{Emphase}{Exercice 5 -- Afficher les multiples de 3}
\cadre{
\begin{pseudo}
\Module{multiplesDe3}{}{}

    \Decl nombre, multiple3~:~entier
    \Repeat
        \Read nombre
        \If{nombre MOD 3 = 0}
               \Write nombre
               \Let multiple3 \Gets multiple3 + 1
        \EndIf
    \Until{nombre = 0}
    \Write multiple3

\EndModule
\end{pseudo}
}
\end{Emphase}



\begin{Emphase}{Exercice 6 -- Placement d’un capital}
\cadre{
\begin{pseudo}
\Module{placementCapital}{}{}

    \Decl capitalDépart, nbAnnées, tauxPlacement, capitalIntérêt~:~entiers
    \Read capitalDépart, nbAnnées, tauxPlacement
    \For{année \K{de} 2013 \K{à} n}
        \Let capitalIntérêt \Gets capitalDépart + (tauxPlacement *
        capitalDépart)/100
        \Write année, capitalDépart, capitalIntérêt - capitalDépart
    \EndFor

\EndModule
\end{pseudo}
}
\end{Emphase}

\newpage

\begin{Emphase}{Exercice 7 -- Produit de 2 nombres}
\cadre{
\begin{pseudo}
\Module{produit2Nb}{}{}

    \Decl nb1, nb2~:~entiers
    \Read nb1, nb2
    \Return nb1 / (1/nb2)

\EndModule
\end{pseudo}
}
\end{Emphase}

\begin{Emphase}{Exercice 8 -- Génération de suites (1/2)}
\cadre{
\begin{pseudo}

\Module{pasCroissant}{}{}

    \Decl pas~:~entier
    \Let pas \Gets 1

    \For{i \K{de} 1 \K{à} n}
        \Write i + pas - 1
        \Let pas \Gets i
    \EndFor

\EndModule
\\

\Module{boiteuse}{}{}

    \For{i \K{de} 1 \K{à} n}
        \If{i MOD 3 = 1}
            \Write i
        \EndIf
    \EndFor

\EndModule
\\

\Module{suiteDeFibonacci}{}{}

    \Decl i1, i2~:~entiers
    \Let i1 \Gets 0
    \Let i2 \Gets 1

    \If{n = 0}
        \Write i1
    \Else
        \If{n = 1}
            \Write i1,i2
        \Else
            \Write i1,i2
            \While{1 < n}
                \Let i \Gets i1 + i2
                \Let i2 \Gets i1
                \Let i1 \Gets i
                \Write i
            \EndWhile
        \EndIf
    \EndIf

\EndModule
\\

\end{pseudo}
}
\end{Emphase}

\newpage

\begin{Emphase}{Exercice 8 -- Génération de suites (2/2)}
\cadre{
\begin{pseudo}

\Module{processionEchternach}{}{}

    \For{i \K{de} 1 \K{à} n}
        \Write i, i + 1, i + 2, i + 3, i + 2
    \EndFor

\EndModule
\\

\Module{combinaison2Suites}{}{}

    \For{i \K{de} 1 \K{à} n}
        \Write i + (i - 1), i + 1
    \EndFor

\EndModule
\\
\Module{capricieuse}{}{}

\For{i \K{de} 1 \K{à} n}
    \If{i MOD 2 = 1}
        \For{j \K{de} 10 * i - 9 \K{à} 10 * i}
            \Write j
        \EndFor
    \Else
        \For{j \K{de} i * 10 \K{à} i * 10 - 9 \K{par} -1}
            \Write j
        \EndFor
    \EndIf
\EndFor
\EndModule

\end{pseudo}
}
\end{Emphase}

\begin{Emphase}{Exercice 9 -- Factorielle}
\cadre{
\begin{pseudo}
\Module{factorielle}{}{}

    \Decl n, fact~:~entiers
    \Let fact \Gets 1
    \Read n

    \For{i \K{de} n \K{à} 1 \K{par} -1}
        \Let fact \Gets fact * i
    \EndFor

    \Write fact

\EndModule

\end{pseudo}
}
\end{Emphase}

\begin{Emphase}{Exercice 10 -- Somme de chiffres}
\cadre{
\begin{pseudo}
\Module{sommeChiffre}{}{}

    \Decl chiffre, nombre~:~entiers
    \Read nombre

    \While{nombre > 0}
        \Let chiffre \Gets nombre MOD 10
        \Let nombre \Gets nombre / 10 
    \EndWhile

    \Write chiffre

\EndModule

\end{pseudo}
}
\end{Emphase}
