\chapter*{Correction des exercices 4.4}

\vspace{-2cm}

\begin{Emphase}{Simplification d’algorithme}

\begin{minipage}{5cm}
\begin{pseudo}

    \If{ok}
        \Write nombre
    \EndIf

\end{pseudo}
\end{minipage}
\begin{minipage}{5cm}
\begin{pseudo}

    \If{NON ok}
        \Write nombre
    \EndIf

\end{pseudo}
\end{minipage}
\begin{minipage}{5cm}
\begin{pseudo}

    \If{ok1 ET ok2}
        \Write x
    \EndIf

\end{pseudo}
\\\\
\end{minipage}

\begin{minipage}{5cm}
\begin{pseudo}

    \Let ok \Gets condition

\end{pseudo}
\end{minipage}
\begin{minipage}{5cm}
\begin{pseudo}

    \Let ok \Gets a $\ge$ b

\end{pseudo}
\end{minipage}
\end{Emphase}


\begin{Emphase}{Exercice 3 -- Maximum de 2 nombres}
\cadre{
\begin{pseudo}
\Module{max2Nb}{}{}

    \Decl nb1, nb2 : entiers
    \Decl max : entier
    \Read nb1, nb2
    \If{nb2 $\ge$ nb1}
        \Let max \Gets nb2
    \Else
        \Let max \Gets nb1
    \EndIf
    \Write max

\EndModule
\end{pseudo}
}
\end{Emphase}


\begin{Emphase}{Exercice 4 -- Maximum de 3 nombres}
\cadre{
\begin{pseudo}
\Module{max3Nb}{}{}

    \Decl nb1, nb2, nb3 : entiers
    \Decl max : entier
    \Read nb1, nb2, nb3
    \If{nb2 $\ge$ nb1}
        \Let max \Gets nb2
    \Else
        \Let max \Gets nb1
    \EndIf
    \If{nb3 $\ge$ max}
        \Let max \Gets nb3
    \EndIf
    \Write max

\EndModule
\end{pseudo}
}
\end{Emphase}

\newpage

\begin{Emphase}{Exercice 5 -- Signe}
\cadre{
\begin{pseudo}
\Module{signe}{}{}

    \Decl nb : entier
    \Read nb
    \Switch{}
        \Stmt nb > 0 : \K{afficher} "positif"
        \Stmt nb < 0 : \K{afficher} "négatif"
        \Stmt autre : \K{afficher} "nul"
    \EndSwitch

\EndModule
\end{pseudo}
}
\end{Emphase}


\begin{Emphase}{Exercice 6 -- La fourchette}
\cadre{
\begin{pseudo}
\Module{fourchette}{}{}

    \Decl nb1, nb2, nb3, petit, grand : entiers
    \Decl ok : booléen
    \Read nb1, nb2, nb3
    \If{nb2 > nb3}
        \Let petit \Gets nb3
        \Let grand \Gets nb2
    \Else
        \Let petit \Gets nb2
        \Let grand \Gets nb3
    \EndIf

    \Write \Gets nb1 > petit ET nb1 < grand    

\EndModule
\end{pseudo}
}
\end{Emphase}


\begin{Emphase}{Exercice 7 -- Équation du second degré}
\cadre{
\begin{pseudo}
\Module{racinesÉquation}{}{}

    \Decl coeffCarré, coeff, termeIndé : entiers
    \Decl delta : entier
    \Read coeffCarré, coeff, termeIndé
    \Let delta \Gets (coeff)$^2$ - 4 * coeffCarré * termeIndé
    \Switch{}
        \Stmt delta > 0 : \K{afficher} (-coeff $\pm$ $\sqrt{\text{delta}}$)/(2 * coeffCarré)
        \Stmt delta = 0 : \K{afficher} -coeff/(2 * coeffCarré)
        \Stmt autres : afficher "pas de racine"
    \EndSwitch

\EndModule
\end{pseudo}
}
\end{Emphase}

\newpage

\begin{Emphase}{Exercice 8 -- Une petite minute}
\cadre{
\begin{pseudo}
\Module{plusUneMin}{}{}

    \Decl heure, minute : entiers
    \Read heure, minute
    \If{minute = 59}
        \Let minute \Gets 0
        \Let heure \Gets heure + 1
    \Else
        \Let minute \Gets minute + 1
    \EndIf
    \Write heure, minute

\EndModule
\end{pseudo}
}
\end{Emphase}


\begin{Emphase}{Exercice 9 -- Calcul de salaire}
\cadre{
\begin{pseudo}
\Module{salaireNet}{}{}

    \Decl salaireBrut : entier
    \Decl constante RETENUE : 15
    \Decl salaireNet : entier
    \Read salaire
    \If{salaire > 1200}
        \Let salaireNet \Gets salaire - (salaire * RETENUE) / 100
        \Write salaireNet
    \Else
        \Write salaireBrut
    \EndIf

\EndModule
\end{pseudo}
}
\end{Emphase}


\begin{Emphase}{Exercice 10 -- Nombres de jours dans un mois}
\cadre{
\begin{pseudo}
\Module{nbJours}{}{}

    \Decl mois : chaine
    \Decl jours : entier
    \Read mois
    \Switch{mois \K{vaut}}
        \Case{"JANVIER", "MARS", "MAI", "JUILLET", "AOÛT", "OCTOBRE",
        "DÉCEMBRE"}
            \Write 31
        \Case{"AVRIL", "JUIN", "SEPTEMBRE", "NOVEMBRE"}
            \Write 30
        \Case{"FÉVRIER"}
            \Write 28
    \EndSwitch

\EndModule
\end{pseudo}
}
\end{Emphase}

\newpage

\begin{Emphase}{Exercice 11 -- Année bissextile}
\cadre{
\begin{pseudo}
\Module{estBissextile}{}{}

    \Decl annee : entier
    \Read annee
    \Write annee MOD 4 = 0 ET NON(annee MOD 100 = 0) OU annee MOD 400 = 0

\EndModule
\end{pseudo}
}
\end{Emphase}


\begin{Emphase}{Exercice 12 -- Valider une date}
\cadre{
\begin{pseudo}
\Module{dateValide}{}{}

    \Decl annee, mois, jour, jourMois : entiers
    \Decl bissextile : booléen
    \Read jour, mois, annee
    \Let bissextile \Gets annee MOD 4 = 0 ET annee MOD 100 <> 0 OU annee MOD 400 = 0

    \Switch{mois \K{vaut}}
        \Case{1, 3, 5, 7, 8, 10, 12}
            \Let jourMois \Gets 31
        \Case{4, 6, 9, 11}
            \Let jourMois \Gets 30
        \Case{2}
            \If{bissextile}
                \Let jourMois \Gets 29
            \Else
                \Let jourMois \Gets 28
            \EndIf
        \Case{\K{autres }} \Write "mois inconnu"
    \EndSwitch

    \Write 1 $\le$ jour $\le$ jourMois

\EndModule
\end{pseudo}
}
\end{Emphase}

%\newpage

\begin{Emphase}{Exercice 13 -- Le jour de la semaine}
\cadre{
\begin{pseudo}
\Module{jourSemaine}{}{}

    \Decl dateMois : entier
    \Read dateMois
    \Switch{dateMois MOD 7 \K{vaut}}
        \Stmt 0 : \K{afficher} "vendredi"
        \Stmt 1 : \K{afficher} "samedi"
        \Stmt 2 : \K{afficher} "dimanche"
        \Stmt 3 : \K{afficher} "lundi"
        \Stmt 4 : \K{afficher} "mardi"
        \Stmt 5 : \K{afficher} "mercredi"
        \Stmt 6 : \K{afficher} "jeudi"
    \EndSwitch

\EndModule
\end{pseudo}
}
\end{Emphase}


\begin{Emphase}{Exercice 14 -- Quel jour serons-nous~?}
\cadre{
\begin{pseudo}
\Module{jourFutur}{}{}

    \Decl jour : chaine
    \Decl n, jourFutur : entier
    \Read jour, n
    \Switch{jour \K{vaut}}
        \Stmt "lundi"    : jourFutur \Gets 1
        \Stmt "mardi"    : jourFutur \Gets 2
        \Stmt "mercredi" : jourFutur \Gets 3
        \Stmt "jeudi"    : jourFutur \Gets 4
        \Stmt "vendredi" : jourFutur \Gets 5
        \Stmt "samedi"   : jourFutur \Gets 6
        \Stmt "dimanche" : jourFutur \Gets 7
    \EndSwitch

    \Switch{(jourFutur + n) MOD 7 \K{vaut}}
        \Stmt 0 : \K{afficher} "lundi"
        \Stmt 1 : \K{afficher} "mardi"
        \Stmt 2 : \K{afficher} "mercredi"
        \Stmt 3 : \K{afficher} "jeudi"
        \Stmt 4 : \K{afficher} "vendredi"
        \Stmt 5 : \K{afficher} "samedi"
        \Stmt 6 : \K{afficher} "dimanche"
    \EndSwitch

\EndModule
\end{pseudo}
}
\end{Emphase}


\begin{Emphase}{Exercice 15 -- Un peu de trigono}
\cadre{
\begin{pseudo}
\Module{cosinus}{}{}

    \Decl n : entier
    \Decl cosinus : entier
    \Read n
    \If{NON(n MOD 2 = 0)}
        \Let cosinus \Gets 0
    \Else
        \If{(n/2) MOD 2 = 0}
            \Let cosinus \Gets 1
        \Else
            \Let cosinus \Gets -1
        \EndIf
    \EndIf
    \Write cosinus

\EndModule
\end{pseudo}
}
\end{Emphase}

\newpage

\begin{Emphase}{Exercice 16 -- Le stationnement alternatif}
\cadre{
\begin{pseudo}
\Module{bienStationné}{}{}
    
    \Decl dateJour, numMaison : entiers
    \If{1 $\ge$ dateJour $\ge$ 15}
        \Write NON(numMaison MOD 2 = 0)
    \Else
        \Write numMaison MOD 2 = 0
    \EndIf

\EndModule
\end{pseudo}
}
\end{Emphase}


