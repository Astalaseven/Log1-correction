%======================
\chapter{Les tableaux}
%======================

	\marginicon{objectif}
	Dans ce chapitre nous étudions les tableaux, 
	une structure qui peut contenir 
	plusieurs exemplaires de données similaires.

% =============================
\section{Utilité des tableaux}
% =============================

	Nous allons introduire la notion de tableau à partir d’un exemple 
	dans lequel l’utilité de cette structure de données 
	apparaitra de façon naturelle.

	\begin{Emphase}{Exemple~: Statistiques de vente}

		{\itshape
		Un gérant d’une entreprise commerciale souhaite connaitre l’impact d’une
		journée de promotion publicitaire sur la vente de dix de ses produits.
		Pour ce faire, les numéros de ces produits (numérotés de 1 à 10 pour
		simplifier) ainsi que les quantités vendues pendant cette journée de
		promotion sont encodés au fur et à mesure de leurs ventes. En fin de
		journée, le vendeur entrera la valeur 0 pour signaler la fin de
		l’introduction des données. Ensuite, les statistiques des ventes seront
		affichées.}

	\end{Emphase}

	La démarche générale se décompose en trois parties~:

	\begin{liste}
	\item {
		le traitement de début de journée, qui consiste essentiellement à mettre
		les compteurs des quantités vendues pour chaque produit à 0}
		\item {
		le traitement itératif durant toute la journée~: au fur et à mesure des
		ventes, il convient de les enregistrer, c’est-à-dire d’ajouter au
		compteur des ventes d’un produit la quantité vendue de ce produit ; ce
		traitement itératif s’interrompra lorsque la valeur 0 sera introduite}
	\item {
		le traitement final, consistant à communiquer les valeurs des compteurs
		pour chaque produit.}
	\end{liste}

	Vous trouverez sur la page suivante, 
	une version possible de cet algorithme.

%	\begin{algorithm}[H]
%	\caption{Solution sans tableau pour les statistiques de ventes}
%	\label{algo:tab-stat}
	\cadre{
	\begin{pseudo}
	\LComment Ce module calcule et affiche la quantité vendue de 10 produits.
	\Module{statistiquesVentesSansTableau}{}{}
		\Empty
		\Decl cpt1, cpt2, cpt3, cpt4, cpt5~: entiers
		\Decl cpt6, cpt7, cpt8, cpt9, cpt10~: entiers
		\Decl numéroProduit, quantité~: entiers
		\Empty
		\Let cpt1 \Gets 0
		\Let cpt2 \Gets 0
		\Let cpt3 \Gets 0
		\Let cpt4 \Gets 0
		\Let cpt5 \Gets 0
		\Let cpt6 \Gets 0
		\Let cpt7 \Gets 0
		\Let cpt8 \Gets 0
		\Let cpt9 \Gets 0
		\Let cpt10 \Gets 0
		\Empty
		\Write "Introduisez le numéro du produit~:"
		\Read numéroProduit
		\Empty
		\While{numéroProduit > 0}
		\Empty
			\Write "Introduisez la quantité vendue~:"
			\Read quantité
			\Empty
			\Switch{numéroProduit \K{vaut}}
				\Stmt 1~: cpt1 \Gets cpt1 + quantité
				\Stmt 2~: cpt2 \Gets cpt2 + quantité
				\Stmt 3~: cpt3 \Gets cpt3 + quantité
				\Stmt 4~: cpt4 \Gets cpt4 + quantité
				\Stmt 5~: cpt5 \Gets cpt5 + quantité
				\Stmt 6~: cpt6 \Gets cpt6 + quantité
				\Stmt 7~: cpt7 \Gets cpt7 + quantité
				\Stmt 8~: cpt8 \Gets cpt8 + quantité
				\Stmt 9~: cpt9 \Gets cpt9 + quantité
				\Stmt 10~:cpt10 \Gets cpt10 + quantité
			\EndSwitch
			\Empty
			\Write "Introduisez le numéro du produit~:"
			\Read numéroProduit
			\Empty
		\EndWhile
		\Empty
		\Write "quantité vendue de produit 1~:", cpt1
		\Write "quantité vendue de produit 2~:", cpt2
		\Write "quantité vendue de produit 3~:", cpt3
		\Write "quantité vendue de produit 4~:", cpt4
		\Write "quantité vendue de produit 5~:", cpt5
		\Write "quantité vendue de produit 6~:", cpt6
		\Write "quantité vendue de produit 7~:", cpt7
		\Write "quantité vendue de produit 8~:", cpt8
		\Write "quantité vendue de produit 9~:", cpt9
		\Write "quantité vendue de produit 10~:", cpt10
		\Empty
	\EndModule
	\end{pseudo}
	}
%	\end{algorithm}

	Il est clair, à la lecture de cet algorithme, qu’une simplification
	d’écriture s’impose ! Et que ce passerait-il si le nombre de produits à
	traiter était de 20 ou 100 ? Le but de l’informatique étant de
	dispenser l’humain des tâches répétitives, le programmeur peut en
	espérer autant de la part d’un langage de programmation ! La solution
	est apportée par un nouveau type de variables~: les \textbf{variables
	indicées} ou \textbf{tableaux}.

	Au lieu d’avoir à manier dix compteurs distincts
	(\textstyleCodeInsr{cpt1}, \textstyleCodeInsr{cpt2}, etc.), nous allons
	envisager une seule «~grande~» variable \textstyleCodeInsr{cpt} 
	compartimentée en dix «~sous-variables~» qui se distingueront 
	les unes des autres par un indice~: \textstyleCodeInsr{cpt1} 
	deviendrait ainsi \textstyleCodeInsr{cpt[1]}, \textstyleCodeInsr{cpt2} 
	deviendrait	\textstyleCodeInsr{cpt[2]}, et ainsi de suite jusqu’à
	\textstyleCodeInsr{cpt10} qui deviendrait \textstyleCodeInsr{cpt[10]}.

	\begin{center}
		\tablehead{}
		\begin{supertabular}{m{0.588cm}m{0.80cm}m{0.80cm}m{0.80cm}m{0.80cm}m{0.80cm}m{0.80cm}m{0.80cm}m{0.80cm}m{0.80cm}m{1.2989999cm}}
			~ &
			\centering  \textstyleCodeInsr{cpt[1]} &
			\centering  \textstyleCodeInsr{cpt[2]} &
			\centering  \textstyleCodeInsr{cpt[3]} &
			\centering  \textstyleCodeInsr{cpt[4]} &
			\centering  \textstyleCodeInsr{cpt[5]} &
			\centering  \textstyleCodeInsr{cpt[6]} &
			\centering  \textstyleCodeInsr{cpt[7]} &
			\centering  \textstyleCodeInsr{cpt[8]} &
			\centering  \textstyleCodeInsr{cpt[9]} &
			\centering\arraybslash 
			\textstyleCodeInsr{cpt[10]}\\\hhline{~----------}
			\multicolumn{1}{m{0.588cm}|}{\centering 
			\textstyleCodeInsr{cpt}} & 
			\multicolumn{1}{m{0.80cm}|}{~} &
			\multicolumn{1}{m{0.80cm}|}{~} &
			\multicolumn{1}{m{0.80cm}|}{~} &
			\multicolumn{1}{m{0.80cm}|}{~} &
			\multicolumn{1}{m{0.80cm}|}{~} &
			\multicolumn{1}{m{0.80cm}|}{~} &
			\multicolumn{1}{m{0.80cm}|}{~} &
			\multicolumn{1}{m{0.80cm}|}{~} &
			\multicolumn{1}{m{0.80cm}|}{~} &
			\multicolumn{1}{m{1.2989999cm}|}{~
			}\\\hhline{~----------}
		\end{supertabular}
	\end{center}

	Un des intérêts de cette notation est la possibilité de faire apparaitre
	une variable entre les crochets, par exemple
	\textstyleCodeInsr{cpt[i]}, ce qui permet une grande économie de lignes
	de code.
	
	Voici la version avec tableau.

	\cadre{
	\begin{pseudo}
	\label{tableau:tab1DStock10Articles}
	\LComment Ce module calcule et affiche la quantité vendue de 10 produits.
	\Module{statistiquesVentesAvecTableau}{}{}
		\Empty
		\Decl cpt~: \K{tableau} [1 à 10] d’entiers
		\Decl i, numéroProduit, quantité~: entiers
		\Empty
		\For{i \K{de} 1 \K{à} 10}
			\Let cpt[i] \Gets 0
		\EndFor
		\Empty
		\Write "Introduisez le numéro du produit~:"
		\Read numéroProduit
		\Empty
		\While{numéroProduit > 0}
		\Empty
			\Write "Introduisez la quantité vendue~:"
			\Read quantité
			\Empty
			\Let cpt[numéroProduit] \Gets cpt[numéroProduit] + quantité
			\Empty
			\Write "Introduisez le numéro du produit~:"
			\Read numéroProduit
			\Empty
		\EndWhile
		\Empty
		\For{i \K{de} 1 \K{à} 10}
			\Write "quantité vendue de produit ", i, ": ", cpt[i]
		\EndFor
		\Empty
	\EndModule
	\end{pseudo}
	}

% ====================
\section{Définitions}
% ====================

	\marginicon{definition}
	Un \textbf{tableau} est une suite d’éléments de même type 
	portant tous le même nom mais se distinguant 
	les uns des autres par un indice.

	L’\textbf{indice} est un entier 
	donnant la position d’un élément dans la suite. 
	Cet indice varie entre la position du premier élément 
	et la position du dernier élément, 
	ces positions correspondant aux bornes de l’indice.
	Notons qu'il n'y a pas de «~trou~»~: 
	tous les éléments existent entre le premier et le dernier indice.

	\marginicon{definition}
	La \textbf{taille} d’un tableau 
	est le nombre (strictement positif) de ses éléments.
	Attention ! la taille d’un tableau ne peut pas être modifiée pendant
	son utilisation.

	Souvent on utilise un tableau plus grand que
	le nombre utile de ses éléments. 
	Seule une partie du tableau est utilisée. 
	On parle alors de taille \textbf{physique}
	(la taille maximale du tableau) 
	et de taille \textbf{logique}
	(le nombre d'éléments effectivement utilisés)

% ==================
\section{Notations}
% ==================

	\marginicon{definition}
	Pour déclarer un tableau, on écrit~:

	\cadre{
	\begin{pseudo}
	\Decl nomTableau~: \K{tableau} [borneMin à borneMax] de TypeElément
	\end{pseudo}
	}
	
	où \textstyleCodeInsr{TypeElément} est le type des éléments que l’on
	trouvera dans le tableau. Les éléments sont d’un des types élémentaires
	vus précédemment (entier, réel, booléen, chaine, caractère) ou encore
	des variables structurées. À ce propos, remarquons aussi
	qu'un tableau peut être un champ d'une structure. 
	D'autres possibilités apparaitront lors de l'étude de
	l'orienté objet.

	Les bornes apparaissant dans la déclaration sont des constantes ou des
	paramètres ayant une valeur connue lors de la déclaration. Une fois un
	tableau déclaré, seuls les éléments d’indice compris entre
	\textstyleCodeInsr{borneMin} et \textstyleCodeInsr{borneMax} peuvent
	être utilisés. Par exemple, si on déclare~:

	\cadre{
	\begin{pseudo}
	\Decl tabEntiers~: \K{tableau} [1 à 100] d’entiers
	\end{pseudo}
	}
	
	Il est interdit d’utiliser \textstyleCodeInsr{tabEntiers[0]} ou
	\textstyleCodeInsr{tabEntiers[101]}. De plus, chaque élément
	\textstyleCodeInsr{tabEntiers[i]} (avec $1 \leq i \leq 100$) doit
	être manié avec la même précaution qu’une variable simple, c’est-à-dire
	qu’on ne peut utiliser un élément du tableau qui n’aurait pas été
	préalablement affecté ou initialisé.

	N.B.~: Il n'est pas interdit de prendre 0 pour la borne
	inférieure ou même d'utiliser des bornes négatives
	(par exemple~: \textstyleCodeInsr{tabTempératures~: 
	\textbf{tableau} [-20 à 50] de réels}).

	\marginicon{attention}
	En Java, un tableau est défini par sa taille $n$ et les bornes sont
	automatiquement $0$ et $n-1$. Ce n'est pas le cas en
	Logique où on a plus de liberté dans le choix des bornes.

% ==============================================
\section{Tableau statique vs tableau dynamique}
% ==============================================

	Les tableaux étudiés jusqu'ici sont dit \textbf{statiques}.
	Un tableau est dit \textbf{statique} si sa taille 
	est connue lors de l’écriture du programme

	Un tableau est dit \textbf{dynamique}
	si sa taille n’est connue qu’à l’exécution du programme (elle est
	calculée, donnée par l’utilisateur, lue dans un fichier de
	configuration, fournie par une autre partie du programme, \dots)


	\textbf{Exemples}

	\begin{liste}
	\item
		Reprenons l’exemple ci-dessus. 
		S’il y a toujours exactement 10 produits, 
		alors on peut utiliser un tableau statique. 
	\item
		Dans la pratique, il est probable que ce nombre puisse évoluer 
		au cours de l’histoire de l’entreprise.
		On peut au moins demander au gérant le nombre d'articles dans son stock.
		On utilisera alors un tableau dynamique.
	\item 
		Si on désire stocker le chiffre d’affaire d’une entreprise 
		pour une année donnée, mois par mois, un tableau de
		taille 12 est suffisant (tableau statique)
	
	\end{liste}
	
	\textbf{Concrètement, comment choisir?}
	Le choix dépend du langage.

	Certains langages (c’est le cas de Cobol) ne permettent 
	que des tableaux statiques. Dans ce cas, il faudra souvent
	imposer une \textbf{taille maximale} au tableau. 
	Il sera également nécessaire d’utiliser une variable 
	pour retenir la taille utile (ou logique) du tableau, 
	à savoir la partie du tableau réellement utilisée.
	Bien sûr cette solution entraine souvent une perte de place mémoire due
	à une réservation inutilement grande (ou, pire, une saturation du
	tableau qui n’aura pas été prévu assez grand).
	
	D'autres, comme Java, n'ont que des tableaux dynamiques. La taille ne doit 
	pas forcément être connue à la compilation, mais doit être connue à
	l'exécution. Vous verrez les détails dans votre cours de Java.

	\textbf{Exemple statique}

	Reprenons encore l’exemple de la vente de produits. 
	Si on ne dispose pas de tableau dynamique, on peut réserver
	un tableau de taille 1000 par exemple (si on est sûr qu’on ne vendra
	pas plus de 1000 produits différents) et ajouter une variable entière
	(\textstyleCodeInsr{nbProduits}) pour retenir 
	le nombre exact de produits différents actuellement en vente.
	
	\cadre{
	\begin{pseudo}
	\label{tableau:tab1DStock10Articles}
	\LComment Ce module calcule et affiche la quantité vendue de x produits.
	\Module{statistiquesVentes}{}{}
		\Empty
		\Decl cpt~: \K{tableau} [1 à 1000] d’entiers
		\Decl i, numéroProduit, quantité~: entiers
		\Decl nbArticles~: entier
		\Read nbArticles
		\Empty
		\For{i \K{de} 1 \K{à} nbArticles}
			\Let cpt[i] \Gets 0
		\EndFor
		\Empty
		\Write "Introduisez le numéro du produit~:"
		\Read numéroProduit
		\Empty
		\While{numéroProduit > 0 ET numéroProduit < nbArticles}
		\Empty
			\Write "Introduisez la quantité vendue~:"
			\Read quantité
			\Empty
			\Let cpt[numéroProduit] \Gets cpt[numéroProduit] + quantité
			\Empty
			\Write "Introduisez le numéro du produit~:"
			\Read numéroProduit
			\Empty
		\EndWhile
		\Empty
		\For{i \K{de} 1 \K{à} 10}
			\Write "quantité vendue de produit ", i, ": ", cpt[i]
		\EndFor
		\Empty
	\EndModule
	\end{pseudo}
	}

	\textbf{Exemple dynamique}
	
	Pour déclarer un tableau dynamique, on sépare la déclaration du tableau
	(où la taille n’est pas donnée) de sa création effective~(où on donne
	la taille)

	\cadre{
	\begin{pseudo}
	\Decl nomTableau~: \K{tableau} de TypeElément
	\LComment Le code peut déterminer ici les bornes
	\Let nomTableau \Gets \K{nouveau} \K{tableau} [ borneMin à borneMax ] de TypeElément
	\end{pseudo}
	}

	où \textstyleCodeInsr{borneMin} et \textstyleCodeInsr{borneMax} sont des
	expressions entières quelconques.
	
		\cadre{
	\begin{pseudo}
	\label{tableau:tab1DStock10Articles}
	\LComment Ce module calcule et affiche la quantité vendue de 10 produits.
	\Module{statistiquesVentesAvecTableau}{}{}
		\Empty
		\Decl cpt~: \K{tableau} d’entiers
		\Decl i, numéroProduit, quantité~: entiers
		\Decl nbArticles~: entier
		\Read nbArticles
		\Let cpt \Gets \K{nouveau} \K{tableau} [1 à nbArticles] d'entiers
		\Empty
		\For{i \K{de} 1 \K{à} nbArticles}
			\Let cpt[i] \Gets 0
		\EndFor
		\Empty
		\Write "Introduisez le numéro du produit~:"
		\Read numéroProduit
		\Empty
		\While{numéroProduit > 0 ET numéroProduit < nbArticles}
		\Empty
			\Write "Introduisez la quantité vendue~:"
			\Read quantité
			\Empty
			\Let cpt[numéroProduit] \Gets cpt[numéroProduit] + quantité
			\Empty
			\Write "Introduisez le numéro du produit~:"
			\Read numéroProduit
			\Empty
		\EndWhile
		\Empty
		\For{i \K{de} 1 \K{à} 10}
			\Write "quantité vendue de produit ", i, ": ", cpt[i]
		\EndFor
		\Empty
	\EndModule
	\end{pseudo}
	}


% ==============================
\section{Tableau et paramètres}
% ==============================

	Un tableau peut être passé en paramètre à un module mais qu’en est-il de
	sa taille ? Il serait utile de pouvoir appeler le même module avec des
	tableaux de tailles différentes. Pour permettre cela, la taille du
	tableau reçu en paramètre est déclarée avec une variable (qui peut être
	considéré comme un paramètre entrant).

	\textbf{Exemples~:}
	
	\cadre{
	\begin{pseudo}
	\Module{brol}{tab\In~: \K{tableau} [1 à n] d'entiers}{}
		\Write "J’ai reçu un tableau de ", n, " éléments".
	\EndModule
	\end{pseudo}
	}

	Ce $n$ va prendre la taille précise du tableau
	(statique ou dynamique) utilisé à chaque appel et peut être utilisé
	dans le corps du module. Bien sûr il s’agit là de la taille physique du
	tableau. Si une partie seulement du tableau doit être traitée, il
	convient de passer également la taille logique en paramètre.
	
	\cadre{
	\begin{pseudo}
	\Module{brol}{tab\In~: \K{tableau} [1 à n] d'entiers, tailleLogique~: entier}{}
		\Write "J’ai reçu un tableau rempli de ", tailleLogique, " éléments "
		\Write "sur ", n, " éléments au total."
	\EndModule
	\end{pseudo}
	}

	
	Notons qu’on admet également qu’un module retourne un tableau. 
	
	\cadre{
	\begin{pseudo}
	\LComment Ce module crée un tableau statique d'entiers de taille 10, l'initialise à 0 et le retourne.
	\Module{créerTableau}{}{\K{tableau} [1 à 10] d'entiers}
		\Decl tab~: \K{tableau} [1 à 10] d'entiers
		\Decl i~: entier
		\For{i \K{de} 1 \K{à} 10}
			\Let tab[i] \Gets 0
		\EndFor
		\Return tab
	\EndModule
	\Empty
	\Module{principalAppelTableau}{}{}
		\Decl entiers~: \K{tableau} [1 à 10] d'entiers
		\Decl i~: entier
		\Empty
		\Let entiers \Gets créerTableau()
		\Empty
		\For{i \K{de} 1 \K{à} 10}
			\Write entiers[i]
		\EndFor
	\EndModule
	\end{pseudo}
	}
	
	Par contre, on ne peut pas lire ou afficher un tableau en une seule
	instruction; il faut des instructions de lecture ou
	d'affichage individuelles pour chacun de ses éléments.
	

% ==================
\section{Exercices}
% ==================

\subsection{Exercices de base}
% -----------------------------

\begin{Exercice}{Somme}
	\marginicon{java}
	Écrire un module qui reçoit en paramètre le tableau
	\textstyleCodeInsr{tabEnt} de $n$ entiers 
	et qui retourne la somme de ses éléments.
\end{Exercice}

\begin{Exercice}{Maximum/minimum}
	\marginicon{java}
	Écrire un module qui reçoit en paramètre le tableau
	\textstyleCodeInsr{tabEnt} de $n$ entiers et qui
	retourne la plus grande valeur de ce tableau. Idem pour le minimum.
\end{Exercice}

\begin{Exercice}{Indice du maximum/minimum}
	\label{ex:indiceminmax}
	\marginicon{java}
	Écrire un module qui reçoit en paramètre le tableau
	\textstyleCodeInsr{tabEnt} de $n$ entiers et qui
	retourne l’indice de l’élément contenant la plus grande valeur de ce
	tableau. 
	En cas d’ex-æquo, c’est l’indice le plus petit qui sera renvoyé.
	
	Que faut-il changer pour renvoyer l’indice le plus grand ?
	Et pour retourner l’indice du minimum ? 
	Réécrire l’algorithme de l’exercice précédent en utilisant celui-ci.
\end{Exercice}

\begin{Exercice}{Nombre d'éléments d'un tableau}
	Écrire un module qui reçoit en paramètre le tableau
	\textstyleCodeInsr{tabEnt} de $n$ entiers et qui
	retourne le nombre d’éléments du tableau.
\end{Exercice}

\begin{Exercice}{La présence d'un 0}
	Écrire un module qui reçoit en paramètre le tableau
	\textstyleCodeInsr{tabEnt} de $n$ entiers 
	et qui retourne un booléen 
	indiquant s'il contient au moins un 0. 
\end{Exercice}

\begin{Exercice}{Plus grand écart absolu}
	Écrire un module qui reçoit en paramètre le tableau
	\textstyleCodeInsr{tabEnt} de $n$ entiers et qui
	retourne le plus grand écart absolu entre deux éléments consécutifs de
	ce tableau.
	Et si on veut le plus petit écart ?
\end{Exercice}

\begin{Exercice}{Remplacement de valeurs}
	Écrire un module qui reçoit en paramètre le tableau
	\textstyleCodeInsr{tabEnt} de $n$ entiers et qui
	remplace dans ce tableau toutes les valeurs multiples de 3 par 0.
\end{Exercice}

\begin{Exercice}{Tableau ordonné ?}
	\marginicon{java}
	Écrire un module qui reçoit en paramètre le tableau
	\textstyleCodeInsr{tabEnt} de $n$ entiers et qui
	vérifie si ce tableau est ordonné (strictement) croissant sur les
	valeurs. Le module retournera \textbf{vrai} si le tableau est ordonné,
	\textbf{faux} sinon.
\end{Exercice}

\begin{Exercice}{Position des minimums}
	\marginicon{java}
	Écrire un module qui reçoit en paramètre le tableau
	\textstyleCodeInsr{tabEnt} de $n$ entiers et qui
	affiche le ou les indice(s) des éléments contenant la valeur minimale
	du tableau.

	\begin{enumerate}[label=\alph*)]
	\item 
		écrire une première version «~classique~» avec deux parcours de tableau
	\item
		écrire une deuxième version qui ne parcourt qu’une seule fois 
		\textstyleCodeInsr{tabEnt} en
		stockant dans un deuxième tableau les indices des plus petits éléments
		rencontrés (ce tableau étant à chaque fois réinitialisé lorsqu’un
		nouveau minimum est rencontré)
	\end{enumerate}
\end{Exercice}

\begin{Exercice}{Renverser un tableau}
	\marginicon{java}
	Écrire un module qui reçoit en paramètre le tableau
	\textstyleCodeInsr{tabEnt} de $n$ entiers, et qui
	«~renverse~» ce tableau, c’est-à-dire qui permute le premier élément
	avec le dernier, le deuxième élément avec l’avant-dernier et ainsi de
	suite.
\end{Exercice}

\begin{Exercice}{Tableau symétrique ?}
	Écrire un module qui reçoit en paramètre le tableau
	\textstyleCodeInsr{tabEnt} de $n$ entiers et qui
	vérifie si ce tableau est symétrique, c’est-à-dire si le premier
	élément est identique au dernier, le deuxième à l’avant-dernier et
	ainsi de suite.
\end{Exercice}

\begin{Exercice}{Cumul des ventes}
	Soit le tableau \textstyleCodeInsr{ventes~: 
	\textbf{tableau} [1~à 12]} d’entiers où le
	premier élément contient le montant total des ventes pour janvier, le
	second pour février, et ainsi de suite jusqu'à
	décembre. Écrire l’algorithme qui reçoit ce tableau en paramètre, et
	renvoie le tableau \textstyleCodeInsr{cumul~: 
	\textbf{tableau}[1~à 12]} ; chaque élément
	de ce tableau devra contenir le cumul de toutes les ventes depuis le
	début de l’année jusqu’au mois correspondant à
	l'indice de l’élément en question. Le dernier élément
	de cumul devra donc contenir le total des ventes de l’année.
\end{Exercice}

\bigskip
\begin{Exercice}{Occurrence des chiffres}
	\marginicon{java}
	Écrire un module qui reçoit un nombre entier positif ou nul en paramètre
	et qui affiche pour chacun de ses chiffres le nombre de fois qu’il
	apparait dans ce nombre. Ainsi, pour le nombre 10502851125, l’affichage
	mentionnera que le chiffre 0 apparait 2 fois, 1 apparait 3 fois, 2
	apparait 2 fois, 5 apparait 3 fois et 8 apparait une fois (l’affichage
	ne mentionnera donc pas les chiffres qui n’apparaissent pas).
\end{Exercice}

\begin{Exercice}{Palindrome}
	Soit le tableau \textstyleCodeInsr{phrase~: 
	\textbf{tableau}[1 à n]} de caractères 
	(caractère alphabétique, le caractère d’espacement ou un
	caractère de ponctuation). Dans ce tableau, des lettres qui se suivent
	constituent un mot. Ces mots sont séparés les uns des autres par un ou
	plusieurs caractères d’espacement ou de ponctuation. Écrire un module
	qui reçoit en paramètre ce tableau et qui vérifie si la phrase formée
	par les mots de ce tableau est un palindrome. Le résultat sera
	communiqué par le renvoi d’une valeur booléenne. Pour rappel, une
	phrase palindrome est une phrase qui peut se lire dans les deux sens
	sans tenir compte des espacements et de la ponctuation.
	
	Exemple du contenu du tableau phrase~:

	\begin{center}
	\begin{tabular}{|*{21}{>{\small\centering\arraybackslash}m{0.30cm}|}}
	\hline
	  A &
	  ~ &
	  L &
	  ' &
	  E &
	  T &
	  A &
	  P &
	  E &
	  , &
	  ~ &
	  ~ &
	  E &
	  P &
	  A &
	  T &
	  E &
	  - &
	  L &
	  A &
	  ! \\
	\hline
	\end{tabular}
	\end{center}

	\textbf{Aide}~: il est permis d’utiliser les modules suivants~:

	\cadre{
	\begin{pseudo}
	\ModuleSign{estEspace}{car~: caractère}{booléen} 
		\Stmt \RComment retourne vrai si car est un caractère d’espacement.
	\ModuleSign{estPonctuation}{car~: caractère}{booléen} 
		\Stmt \RComment retourne vrai si car est un caractère de ponctuation.
	\end{pseudo}
	}
\end{Exercice}

\begin{Exercice}{Moyenne d'éléments}
	\marginicon{java}
	Soit le tableau \textstyleCodeInsr{tabEnt} contenant
	$n$ entiers \textbf{différents}. Écrire l’algorithme
	qui calcule et affiche la moyenne des éléments situés entre les valeurs
	minimale et maximale du tableau (ces deux valeurs participant au calcul
	de cette moyenne). Dans l’exemple ci-dessous, le maximum est 85, le
	minimum 5 et la moyenne des 8 éléments concernés est 21. Exploitez
	l'exercice \vref{ex:indiceminmax} pour réaliser cet algorithme. 
	
	\begin{center}
	\begin{tabular}{|*{20}{>{\centering\arraybackslash}m{0.35cm}|}}
	\hline
	  12 &
	  \cellcolor{gray!25}85 &
	  \cellcolor{gray!25}21 &
	  \cellcolor{gray!25}17 &
	  \cellcolor{gray!25}8 &
	  \cellcolor{gray!25}6 &
	  \cellcolor{gray!25}10 &
	  \cellcolor{gray!25}16 &
	  \cellcolor{gray!25}5 &
	  74 &
	  64 &
	  29 &
	  41 &
	  11 &
	  73 &
	  72 &
	  28 &
	  66 &
	  55 &
	  44
	\\\hline
	\end{tabular}
	\end{center}
\end{Exercice}

\begin{Exercice}{OXO}
	Soit le tableau \textstyleCodeInsr{oxo} contenant $n$
	caractères. Chaque élément est le caractère ‘O’ ou ‘X’. Écrire
	l’algorithme qui permet de compter et d’afficher le nombre de séquences
	distinctes des valeurs consécutives ‘O’, ‘X’, ‘O’. Lorsqu’un ‘O’ fait
	partie de deux séquences, on ne comptabilise qu’une seule séquence
	(ainsi le ‘O’ en position 6 ci-dessous est le dernier d’une séquence et
	le premier d’une autre~: on ne comptabilisera qu’une seule séquence
	pour ces deux séquences qui se chevauchent). Dans l’exemple ci-dessous,
	il y a donc 3 séquences à comptabiliser~:

	\begin{center}
	\begin{tabular}{*{20}{>{\centering\arraybackslash}m{0.35cm}}}
	 \itshape 1 &
	 \itshape 2 &
	 \itshape 3 &
	 \itshape 4 &
	 \itshape 5 &
	 \itshape 6 &
	 \itshape 7 &
	 \itshape 8 &
	 \itshape 9 &
	 \itshape 10 &
	 \itshape 11 &
	 \itshape 12 &
	 \itshape 13 &
	 \itshape 14 &
	 \itshape 15 &
	 \itshape 16 &
	 \itshape 17 &
	 \itshape 18 &
	 \itshape 19 &
	 \itshape 20
	 \\\hline
	\multicolumn{1}{|>{\centering\arraybackslash}m{0.35cm}|}{ O} &
	\multicolumn{1}{>{\centering\arraybackslash}m{0.35cm}|}{  X} &
	\multicolumn{1}{>{\centering\arraybackslash}m{0.35cm}|}{  X} &
	\multicolumn{1}{>{\centering\arraybackslash}m{0.35cm}|}{ \cellcolor{gray!25} O} &
	\multicolumn{1}{>{\centering\arraybackslash}m{0.35cm}|}{ \cellcolor{gray!25} X} &
	\multicolumn{1}{>{\centering\arraybackslash}m{0.35cm}|}{ \cellcolor{gray!25} O} &
	\multicolumn{1}{>{\centering\arraybackslash}m{0.35cm}|}{  X} &
	\multicolumn{1}{>{\centering\arraybackslash}m{0.35cm}|}{ \cellcolor{gray!25} O} &
	\multicolumn{1}{>{\centering\arraybackslash}m{0.35cm}|}{ \cellcolor{gray!25} X} &
	\multicolumn{1}{>{\centering\arraybackslash}m{0.35cm}|}{ \cellcolor{gray!25} O} &
	\multicolumn{1}{>{\centering\arraybackslash}m{0.35cm}|}{  O} &
	\multicolumn{1}{>{\centering\arraybackslash}m{0.35cm}|}{  O} &
	\multicolumn{1}{>{\centering\arraybackslash}m{0.35cm}|}{  O} &
	\multicolumn{1}{>{\centering\arraybackslash}m{0.35cm}|}{  X} &
	\multicolumn{1}{>{\centering\arraybackslash}m{0.35cm}|}{  X} &
	\multicolumn{1}{>{\centering\arraybackslash}m{0.35cm}|}{ \cellcolor{gray!25} O} &
	\multicolumn{1}{>{\centering\arraybackslash}m{0.35cm}|}{ \cellcolor{gray!25} X} &
	\multicolumn{1}{>{\centering\arraybackslash}m{0.35cm}|}{ \cellcolor{gray!25} O} &
	\multicolumn{1}{>{\centering\arraybackslash}m{0.35cm}|}{  O} &
	\multicolumn{1}{>{\centering\arraybackslash}m{0.35cm}|}{  X}\\
	\hline
	\end{tabular}
	\end{center}
\end{Exercice}

\begin{Exercice}{Les doublons}
	\marginicon{java}
	Vérifiez si un tableau contient au moins 2 éléments égaux.
\end{Exercice}

\bigskip
\bigskip

\begin{Exercice}{Mastermind}
	\marginicon{java}
	Dans le jeu du Mastermind, un joueur A doit trouver une combinaison de k
	pions de couleur, choisie et tenue secrète par un autre joueur B. Cette
	combinaison peut contenir éventuellement des pions de même couleur. À
	chaque proposition du joueur A, le joueur B indique le nombre de pions
	de la proposition qui sont corrects et bien placés et le nombre de
	pions corrects mais mal placés. 
	
	Pour implémenter une simulation de ce jeu, on utilise le type Couleur,
	dont les valeurs possibles sont les couleurs des pions utilisés.
	(Attention, le nombre exact de couleurs n’est pas précisé.) Les seules
	manipulations permises avec ce type sont la comparaison (tester si deux
	couleurs sont identiques ou non) et l’affectation (affecter le contenu
	d’une variable de type Couleur à une autre variable de ce type). Les
	propositions du joueur A, ainsi que la combinaison secrète du joueur B
	sont contenues dans des tableaux de k composantes de type Couleur.
	
	Écrire le module suivant qui renvoie dans les variables
	\textstyleCodeInsr{bienPlacés} et \textstyleCodeInsr{malPlacés}
	respectivement le nombre de pions bien placés et mal placés dans la
	«~proposition~» du joueur A en la comparant à la «~solution~» cachée du
	joueur B.

	\cadre{
	\begin{pseudo}
	\ModuleSign{testerProposition}{
		proposition\In, solution\In~: 
		\K{tableau} [1 à k ] de Couleur,
		\\\hfill bienPlacés\Out, malPlacés\Out: entiers}{}
	\end{pseudo}
	}
\end{Exercice}

\begin{Exercice}{Casser le chiffre de César}
	Dans le chapitre sur les boucles, l'exercice \vref{ex:cesar}
	vous a montré la technique de César pour chiffrer ses messages.
	
	Ce chiffre est en fait assez facile à casser grâce à l'analyse statistique.
	En effet, dans un texte général, les lettres de l'alphabet ne sont pas
	présentes avec la même fréquence. 
	Par exemple, en français, la lettre 'E' est la plus présente.\footnote{Il y a
	évidemment des exceptions, cf. "La disparition" de Georges Perec.}
	Ainsi, si on trouve un message chiffré 
	(et si on sait que c'est du français et que c'est chiffré par le chiffre de César)
	on peut tenter de le comprendre via une analyse statistique. 
	On analyse la fréquence de chaque lettre du message et la plus fréquente remplace
	probablement le 'E'. Par exemple, si la lettre la plus fréquente est le 'B',
	il est probable que le message soit chiffré avec un décalage de 3.
	
	Écrivez un module qui reçoit une chaine, 
	le message chiffré par le chiffre de César
	et retourne une valeur probable pour le déplacement utilisé pour le chiffrer.
\end{Exercice}

